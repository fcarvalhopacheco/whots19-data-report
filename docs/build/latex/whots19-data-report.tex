%% Generated by Sphinx.
\def\sphinxdocclass{report}
\documentclass[a4paper,10pt,english,openany,oneside]{sphinxmanual}
\ifdefined\pdfpxdimen
   \let\sphinxpxdimen\pdfpxdimen\else\newdimen\sphinxpxdimen
\fi \sphinxpxdimen=.75bp\relax
\ifdefined\pdfimageresolution
    \pdfimageresolution= \numexpr \dimexpr1in\relax/\sphinxpxdimen\relax
\fi
%% let collapsible pdf bookmarks panel have high depth per default
\PassOptionsToPackage{bookmarksdepth=5}{hyperref}

\PassOptionsToPackage{warn}{textcomp}
\usepackage[utf8]{inputenc}
\ifdefined\DeclareUnicodeCharacter
% support both utf8 and utf8x syntaxes
  \ifdefined\DeclareUnicodeCharacterAsOptional
    \def\sphinxDUC#1{\DeclareUnicodeCharacter{"#1}}
  \else
    \let\sphinxDUC\DeclareUnicodeCharacter
  \fi
  \sphinxDUC{00A0}{\nobreakspace}
  \sphinxDUC{2500}{\sphinxunichar{2500}}
  \sphinxDUC{2502}{\sphinxunichar{2502}}
  \sphinxDUC{2514}{\sphinxunichar{2514}}
  \sphinxDUC{251C}{\sphinxunichar{251C}}
  \sphinxDUC{2572}{\textbackslash}
\fi
\usepackage{cmap}
\usepackage[T1]{fontenc}
\usepackage{amsmath,amssymb,amstext}
\usepackage{babel}


\usepackage{amsmath,amsfonts,amssymb,amsthm}


\usepackage[Bjornstrup]{fncychap}
\usepackage[,numfigreset=1,mathnumfig]{sphinx}
\sphinxsetup{hmargin={0.7in,0.7in}, vmargin={1in,1in},         verbatimwithframe=true,         TitleColor={rgb}{0,0,0},         HeaderFamily=\rmfamily\bfseries,         InnerLinkColor={rgb}{0,0,1},         OuterLinkColor={rgb}{0,0,1}}
\fvset{fontsize=auto}
\usepackage{geometry}


% Include hyperref last.
\usepackage{hyperref}
% Fix anchor placement for figures with captions.
\usepackage{hypcap}% it must be loaded after hyperref.
% Set up styles of URL: it should be placed after hyperref.
\urlstyle{same}

\addto\captionsenglish{\renewcommand{\contentsname}{Table of Contents}}

\usepackage{sphinxmessages}
\setcounter{tocdepth}{2}


% Jupyter Notebook code cell colors
\definecolor{nbsphinxin}{HTML}{307FC1}
\definecolor{nbsphinxout}{HTML}{BF5B3D}
\definecolor{nbsphinx-code-bg}{HTML}{F5F5F5}
\definecolor{nbsphinx-code-border}{HTML}{E0E0E0}
\definecolor{nbsphinx-stderr}{HTML}{FFDDDD}
% ANSI colors for output streams and traceback highlighting
\definecolor{ansi-black}{HTML}{3E424D}
\definecolor{ansi-black-intense}{HTML}{282C36}
\definecolor{ansi-red}{HTML}{E75C58}
\definecolor{ansi-red-intense}{HTML}{B22B31}
\definecolor{ansi-green}{HTML}{00A250}
\definecolor{ansi-green-intense}{HTML}{007427}
\definecolor{ansi-yellow}{HTML}{DDB62B}
\definecolor{ansi-yellow-intense}{HTML}{B27D12}
\definecolor{ansi-blue}{HTML}{208FFB}
\definecolor{ansi-blue-intense}{HTML}{0065CA}
\definecolor{ansi-magenta}{HTML}{D160C4}
\definecolor{ansi-magenta-intense}{HTML}{A03196}
\definecolor{ansi-cyan}{HTML}{60C6C8}
\definecolor{ansi-cyan-intense}{HTML}{258F8F}
\definecolor{ansi-white}{HTML}{C5C1B4}
\definecolor{ansi-white-intense}{HTML}{A1A6B2}
\definecolor{ansi-default-inverse-fg}{HTML}{FFFFFF}
\definecolor{ansi-default-inverse-bg}{HTML}{000000}

% Define an environment for non-plain-text code cell outputs (e.g. images)
\makeatletter
\newenvironment{nbsphinxfancyoutput}{%
    % Avoid fatal error with framed.sty if graphics too long to fit on one page
    \let\sphinxincludegraphics\nbsphinxincludegraphics
    \nbsphinx@image@maxheight\textheight
    \advance\nbsphinx@image@maxheight -2\fboxsep   % default \fboxsep 3pt
    \advance\nbsphinx@image@maxheight -2\fboxrule  % default \fboxrule 0.4pt
    \advance\nbsphinx@image@maxheight -\baselineskip
\def\nbsphinxfcolorbox{\spx@fcolorbox{nbsphinx-code-border}{white}}%
\def\FrameCommand{\nbsphinxfcolorbox\nbsphinxfancyaddprompt\@empty}%
\def\FirstFrameCommand{\nbsphinxfcolorbox\nbsphinxfancyaddprompt\sphinxVerbatim@Continues}%
\def\MidFrameCommand{\nbsphinxfcolorbox\sphinxVerbatim@Continued\sphinxVerbatim@Continues}%
\def\LastFrameCommand{\nbsphinxfcolorbox\sphinxVerbatim@Continued\@empty}%
\MakeFramed{\advance\hsize-\width\@totalleftmargin\z@\linewidth\hsize\@setminipage}%
\lineskip=1ex\lineskiplimit=1ex\raggedright%
}{\par\unskip\@minipagefalse\endMakeFramed}
\makeatother
\newbox\nbsphinxpromptbox
\def\nbsphinxfancyaddprompt{\ifvoid\nbsphinxpromptbox\else
    \kern\fboxrule\kern\fboxsep
    \copy\nbsphinxpromptbox
    \kern-\ht\nbsphinxpromptbox\kern-\dp\nbsphinxpromptbox
    \kern-\fboxsep\kern-\fboxrule\nointerlineskip
    \fi}
\newlength\nbsphinxcodecellspacing
\setlength{\nbsphinxcodecellspacing}{0pt}

% Define support macros for attaching opening and closing lines to notebooks
\newsavebox\nbsphinxbox
\makeatletter
\newcommand{\nbsphinxstartnotebook}[1]{%
    \par
    % measure needed space
    \setbox\nbsphinxbox\vtop{{#1\par}}
    % reserve some space at bottom of page, else start new page
    \needspace{\dimexpr2.5\baselineskip+\ht\nbsphinxbox+\dp\nbsphinxbox}
    % mimick vertical spacing from \section command
      \addpenalty\@secpenalty
      \@tempskipa 3.5ex \@plus 1ex \@minus .2ex\relax
      \addvspace\@tempskipa
      {\Large\@tempskipa\baselineskip
             \advance\@tempskipa-\prevdepth
             \advance\@tempskipa-\ht\nbsphinxbox
             \ifdim\@tempskipa>\z@
               \vskip \@tempskipa
             \fi}
    \unvbox\nbsphinxbox
    % if notebook starts with a \section, prevent it from adding extra space
    \@nobreaktrue\everypar{\@nobreakfalse\everypar{}}%
    % compensate the parskip which will get inserted by next paragraph
    \nobreak\vskip-\parskip
    % do not break here
    \nobreak
}% end of \nbsphinxstartnotebook

\newcommand{\nbsphinxstopnotebook}[1]{%
    \par
    % measure needed space
    \setbox\nbsphinxbox\vbox{{#1\par}}
    \nobreak % it updates page totals
    \dimen@\pagegoal
    \advance\dimen@-\pagetotal \advance\dimen@-\pagedepth
    \advance\dimen@-\ht\nbsphinxbox \advance\dimen@-\dp\nbsphinxbox
    \ifdim\dimen@<\z@
      % little space left
      \unvbox\nbsphinxbox
      \kern-.8\baselineskip
      \nobreak\vskip\z@\@plus1fil
      \penalty100
      \vskip\z@\@plus-1fil
      \kern.8\baselineskip
    \else
      \unvbox\nbsphinxbox
    \fi
}% end of \nbsphinxstopnotebook

% Ensure height of an included graphics fits in nbsphinxfancyoutput frame
\newdimen\nbsphinx@image@maxheight % set in nbsphinxfancyoutput environment
\newcommand*{\nbsphinxincludegraphics}[2][]{%
    \gdef\spx@includegraphics@options{#1}%
    \setbox\spx@image@box\hbox{\includegraphics[#1,draft]{#2}}%
    \in@false
    \ifdim \wd\spx@image@box>\linewidth
      \g@addto@macro\spx@includegraphics@options{,width=\linewidth}%
      \in@true
    \fi
    % no rotation, no need to worry about depth
    \ifdim \ht\spx@image@box>\nbsphinx@image@maxheight
      \g@addto@macro\spx@includegraphics@options{,height=\nbsphinx@image@maxheight}%
      \in@true
    \fi
    \ifin@
      \g@addto@macro\spx@includegraphics@options{,keepaspectratio}%
    \fi
    \setbox\spx@image@box\box\voidb@x % clear memory
    \expandafter\includegraphics\expandafter[\spx@includegraphics@options]{#2}%
}% end of "\MakeFrame"-safe variant of \sphinxincludegraphics
\makeatother

\makeatletter
\renewcommand*\sphinx@verbatim@nolig@list{\do\'\do\`}
\begingroup
\catcode`'=\active
\let\nbsphinx@noligs\@noligs
\g@addto@macro\nbsphinx@noligs{\let'\PYGZsq}
\endgroup
\makeatother
\renewcommand*\sphinxbreaksbeforeactivelist{\do\<\do\"\do\'}
\renewcommand*\sphinxbreaksafteractivelist{\do\.\do\,\do\:\do\;\do\?\do\!\do\/\do\>\do\-}
\makeatletter
\fvset{codes*=\sphinxbreaksattexescapedchars\do\^\^\let\@noligs\nbsphinx@noligs}
\makeatother


        \input{mystyle.sty}
        \usepackage[notocbib]{apacite}
    

\title{WHOTS-19: Data Report}
\date{Jun 25, 2025}
\release{0.0.1}
\author{Fernando Carvalho Pacheco}
\newcommand{\sphinxlogo}{\sphinxincludegraphics{all_whots_report.pdf}\par}
\renewcommand{\releasename}{Release}
\makeindex
\begin{document}

\pagestyle{empty}
 
        \input{maketitle.sty}
    
\pagestyle{plain}
 
\pagestyle{normal}
\phantomsection\label{\detokenize{index::doc}}



\chapter{Introduction}
\label{\detokenize{1_section:introduction}}\label{\detokenize{1_section::doc}}
\sphinxAtStartPar
In 2003, \sphinxhref{https://www.whoi.edu/profile/rweller/}{Robert Weller} (\sphinxhref{https://www.whoi.edu}{Woods Hole
Oceanographic Institution {[}WHOI{]}})
, \sphinxhref{https://www.whoi.edu/profile/aplueddemann/}{Albert Plueddemann}
(\sphinxhref{https://www.whoi.edu}{WHOI}), and
\sphinxhref{http://www.soest.hawaii.edu/oceanography/faculty/rlukas/}{Roger Lukas}
(\sphinxhref{https://manoa.hawaii.edu}{The University of Hawaii {[}UH{]}}) proposed to establish
a \sphinxhref{http://www.soest.hawaii.edu/whots/}{long\sphinxhyphen{}term surface mooring at the Hawaii Ocean Time\sphinxhyphen{}series (HOT)}
\sphinxhref{https://hahana.soest.hawaii.edu/stationaloha/}{Station ALOHA (22°45’N, 158°W)}
to provide sustained, high\sphinxhyphen{}quality air\sphinxhyphen{}sea fluxes and the associated upper
ocean response as a coordinated part of the HOT program, and as an element of
the global array of ocean reference stations supported by the National Oceanic
and Atmospheric Administration’s (NOAA) Office of Climate Observation.

\sphinxAtStartPar
With support from the NOAA and the National Science Foundation (NSF), the WHOI
HOT Site (WHOTS) surface mooring has been maintained at Station ALOHA since
August 2004. This project aims to record long\sphinxhyphen{}term, high\sphinxhyphen{}quality air\sphinxhyphen{}sea fluxes
as a coordinated part of the HOT program and contribute to the goals of
observing heat, freshwater, and chemical fluxes at a site representative of the
oligotrophic North Pacific Ocean. The approach is to maintain a surface mooring
outfitted for meteorological and oceanographic measurements at a site near
Station ALOHA by successive mooring turnarounds. These observations will be
used to investigate air\sphinxhyphen{}sea interaction processes related to climate
variability.

\sphinxAtStartPar
The original mooring system is described in the mooring deployment and recovery
cruise reports {[}\hyperlink{cite.references:id14}{Plueddemann \sphinxstyleemphasis{et al.}, 2006}, \hyperlink{cite.references:id22}{Whelan \sphinxstyleemphasis{et al.}, 2007}, \hyperlink{cite.references:id23}{Whelan \sphinxstyleemphasis{et al.}, 2008}, \hyperlink{cite.references:id25}{Whelan \sphinxstyleemphasis{et al.}, 2010}, \hyperlink{cite.references:id17}{Santiago\sphinxhyphen{}Mandujano \sphinxstyleemphasis{et al.}, 2024}, \hyperlink{cite.references:id18}{Santiago\sphinxhyphen{}Mandujano \sphinxstyleemphasis{et al.}, 2024}{]}.
Briefly, a Surlyn foam surface buoy is equipped with meteorological
instrumentation, including two complete Air\sphinxhyphen{}Sea Interaction Meteorological
(ASIMET) systems, measuring air and sea surface temperatures, relative humidity,
barometric pressure, wind speed and direction, incoming shortwave and longwave
radiation, and precipitation {[}\hyperlink{cite.references:id9}{Hosom \sphinxstyleemphasis{et al.}, 1995}, \hyperlink{cite.references:id2}{Colbo and Weller, 2009}{]}. Complete surface
meteorological measurements are recorded every minute, as required to compute
air\sphinxhyphen{}sea fluxes of heat, freshwater, and momentum. Each ASIMET system also
transmits hourly averages of the surface meteorological variables via the Argos
satellite system. The mooring line is instrumented to collect time series of
upper ocean temperatures, velocities, and salinities coincident with the
surface forcing record. This mooring includes conductivity, salinity, and
temperature recorders, two Vector Measuring Current Meters (VMCMs), and two
Acoustic Doppler current profilers (ADCPs). See the WHOTS\sphinxhyphen{}19 mooring diagram in
the \hyperref[\detokenize{1_section:diagram}]{Fig.\@ \ref{\detokenize{1_section:diagram}}}.

\sphinxAtStartPar
The subsurface instrumentation is located to resolve the temporal variations of
shear and stratification in the upper pycnocline to support the study of mixed
layer entrainment. Experience with moored profiler measurements near Hawaii
suggests that Richardson number estimates over 10 m scales are adequate.
Salinity is essential to the stratification, as salt\sphinxhyphen{}stratified barrier layers
are observed at HOT and in the region {[}\hyperlink{cite.references:id11}{Kara \sphinxstyleemphasis{et al.}, 2000}{]}. Hence, we use Sea\sphinxhyphen{}Bird
SeaCATs and MicroCATs with vertical separation ranging from 5 to 20 m to
measure temperature and salinity. We use two ADCPs made by Teledyne RD
Instruments to obtain current profiles across the entrainment zone and in the
mixed layer zone. Both ADCPs are in an upward\sphinxhyphen{}looking configuration, one is at
125 m, using 4 m bins, and the other is at 47.5 m using 2 m bins. To provide
near\sphinxhyphen{}surface velocity (where ADCP estimates are less reliable), we deploy two
Vector Measuring Current Meters (VMCMs). The nominal mooring design is a
balance between resolving extremes versus the typical annual cycling of the
mixed layer {[}\hyperlink{cite.references:id14}{Plueddemann \sphinxstyleemphasis{et al.}, 2006}, \hyperlink{cite.references:id15}{Santiago\sphinxhyphen{}Mandujano \sphinxstyleemphasis{et al.}, 2007}{]}. A pair of Sea\sphinxhyphen{}Bird
SeaCATs (SBE\sphinxhyphen{}16) or MicroCATs (SBE\sphinxhyphen{}37) have been included since the WHOTS\sphinxhyphen{}9
deployment (June 2012) to measure near\sphinxhyphen{}bottom temperature and salinity.

\begin{figure}[htbp]
\centering
\capstart

\noindent\sphinxincludegraphics[height=1000\sphinxpxdimen]{{whots19-diagram}.png}
\caption{WHOTS\sphinxhyphen{}19 mooring design}\label{\detokenize{1_section:diagram}}\end{figure}

\sphinxAtStartPar
The WHOTS\sphinxhyphen{}19 mooring was deployed on June 17, 2023
(\sphinxhref{http://www.soest.hawaii.edu/whots/wh19\_dep.html}{WHOTS\sphinxhyphen{}19 cruise})
and was recovered on June 06, 2024
(\sphinxhref{http://www.soest.hawaii.edu/whots/wh20\_dep.html}{WHOTS\sphinxhyphen{}20 cruise}).
The cruises were aboard the R/V Oscar Elton Settle. The WHOTS\sphinxhyphen{}20 mooring was
deployed on June 02, 2024, during the
\sphinxhref{http://www.soest.hawaii.edu/whots/wh20\_dep.html}{WHOTS\sphinxhyphen{}20 cruise}
and is scheduled for recovery in September 2025.

\sphinxAtStartPar
This report documents and describes the oceanographic observations made on the
WHOTS\sphinxhyphen{}19 mooring for nearly one year and from shipboard
measurements during the two cruises when the mooring was deployed and
recovered. Sections
{\hyperref[\detokenize{2_section:description-of-the-whots-19-mooring-cruises}]{\sphinxcrossref{\DUrole{std,std-ref}{II}}}} and
{\hyperref[\detokenize{3_section:description-of-whots-19-mooring}]{\sphinxcrossref{\DUrole{std,std-ref}{III}}}} include a detailed
description of the cruises and the mooring, respectively. Sampling and
processing procedures of the hydrographic casts, thermosalinograph, and
shipboard ADCP data collected during these cruises are described in Section
{\hyperref[\detokenize{4_section:whots-19-20-cruise-shipboard-observations}]{\sphinxcrossref{\DUrole{std,std-ref}{IV}}}}. Section
{\hyperref[\detokenize{5_section:moored-instrument-observations}]{\sphinxcrossref{\DUrole{std,std-ref}{V}}}} includes the processing
procedures for the data collected by the moored instruments:
{\hyperref[\detokenize{5_section:microcat-data-processing-procedures}]{\sphinxcrossref{\DUrole{std,std-ref}{SeaCATs, MicroCATs}}}},
{\hyperref[\detokenize{5_section:acoustic-doppler-current-profiler}]{\sphinxcrossref{\DUrole{std,std-ref}{Moored ADCPs}}}} and
{\hyperref[\detokenize{5_section:vector-measuring-current-meter-vmcm}]{\sphinxcrossref{\DUrole{std,std-ref}{VMCM}}}}. Plots of the
resulting data and preliminary analysis are presented in Section
{\hyperref[\detokenize{6_section:results}]{\sphinxcrossref{\DUrole{std,std-ref}{VI}}}}.


\chapter{Description of the WHOTS\sphinxhyphen{}19 Mooring Cruises}
\label{\detokenize{2_section:description-of-the-whots-19-mooring-cruises}}\label{\detokenize{2_section::doc}}

\section{WHOTS\sphinxhyphen{}19 Cruise: WHOTS\sphinxhyphen{}19 Mooring Deployment}
\label{\detokenize{2_section:whots-19-cruise-whots-19-mooring-deployment}}
\sphinxAtStartPar
The Woods Hole Oceanographic Institution Upper Ocean Processes Group (WHOI/UOP)
, with the UH group’s assistance, conducted the 19 deployment of the WHOTS
mooring onboard the Oscar Elton Settle during the WHOTS\sphinxhyphen{}19 cruise between June
15 and June 22, 2023. The WHOTS\sphinxhyphen{}19 mooring was deployed at Station 50 on June
17, 2023, 02:59 UTC at 22 46.002’N, 157 53.768’W, and the WHOTS\sphinxhyphen{}18 mooring were
recovered on June 18, 2023. The scientific personnel that participated during
the cruise are listed in \hyperref[\detokenize{2_section:table-1}]{Table \ref{\detokenize{2_section:table-1}}}.


\begin{savenotes}\sphinxattablestart
\centering
\sphinxcapstartof{table}
\sphinxthecaptionisattop
\sphinxcaption{Scientific personnel on Ship Oscar Sette during the WHOTS\sphinxhyphen{}19 deployment cruise.}\label{\detokenize{2_section:table-1}}
\sphinxaftertopcaption
\begin{tabulary}{\linewidth}[t]{|T|T|T|}
\hline
\sphinxstyletheadfamily 
\sphinxAtStartPar
\sphinxstylestrong{Name}
&\sphinxstyletheadfamily 
\sphinxAtStartPar
\sphinxstylestrong{Title or function}
&\sphinxstyletheadfamily 
\sphinxAtStartPar
\sphinxstylestrong{Affiliation}
\\
\hline
\sphinxAtStartPar
Plueddeman, Albert
&
\sphinxAtStartPar
Chief Scientist
&
\sphinxAtStartPar
WHOI
\\
\hline
\sphinxAtStartPar
Bigorre, Sebastien
&
\sphinxAtStartPar
Research Specialist
&
\sphinxAtStartPar
WHOI
\\
\hline
\sphinxAtStartPar
Graham, Raymond
&
\sphinxAtStartPar
Research Associate
&
\sphinxAtStartPar
WHOI
\\
\hline
\sphinxAtStartPar
Llanos, Eduardo
&
\sphinxAtStartPar
Research Technician
&
\sphinxAtStartPar
WHOI
\\
\hline
\sphinxAtStartPar
Santiago\sphinxhyphen{}Mandujano, Fernando
&
\sphinxAtStartPar
Research Associate
&
\sphinxAtStartPar
UH
\\
\hline
\sphinxAtStartPar
Fitzgerald, Dan
&
\sphinxAtStartPar
Marine Electronics Technician
&
\sphinxAtStartPar
UH
\\
\hline
\sphinxAtStartPar
Rohrer, Tully
&
\sphinxAtStartPar
Research Associate
&
\sphinxAtStartPar
UH
\\
\hline
\sphinxAtStartPar
Maloney, Kelsey
&
\sphinxAtStartPar
HIMB Housing Coordinator
&
\sphinxAtStartPar
UH
\\
\hline
\sphinxAtStartPar
Adkison, Camille
&
\sphinxAtStartPar
Graduate Volunteer
&
\sphinxAtStartPar
UH
\\
\hline
\sphinxAtStartPar
Dale, Elizabeth
&
\sphinxAtStartPar
Research Technician
&
\sphinxAtStartPar
WHOI
\\
\hline
\end{tabulary}
\par
\sphinxattableend\end{savenotes}

\sphinxAtStartPar
The UH group conducted the shipboard oceanographic observations during the
cruise. A complete description of these operations is available in the
{[}\hyperlink{cite.references:id17}{Santiago\sphinxhyphen{}Mandujano \sphinxstyleemphasis{et al.}, 2024}{]}.

\sphinxAtStartPar
A Sea\sphinxhyphen{}Bird CTD (Conductivity, Temperature, and Depth) system was used to
collect temperature (T), salinity (S), and dissolved oxygen (O\(\sb{\text{2}}\)) profiles
during CTD casts. The time, location, and maximum pressure for each cast are
listed in \{numref\}table\sphinxhyphen{}2. A total of five CTD casts were conducted during the
WHOTS\sphinxhyphen{}19 cruise between June 16 and June 19. CTD profiles were obtained at
Station 20 (while in transit to the WHOTS mooring) and Station 52 (near the
WHOTS\sphinxhyphen{}18 buoy). The cast at Station 20 reached a depth of approximately 1500 m
and included three acoustic releases—two designated for the WHOTS\sphinxhyphen{}19 mooring
and one backup—attached to the rosette frame for functional testing. Four CTD
yo\sphinxhyphen{}yo casts were conducted at Station 52 to obtain repeated upper\sphinxhyphen{}ocean
profiles for comparison with subsurface instruments on the WHOTS\sphinxhyphen{}18 mooring
prior to its recovery. These casts were initiated approximately 0.25 nautical
miles from the buoy and included multiple up\sphinxhyphen{}and\sphinxhyphen{}down cycles between the
surface and depths of 200 to 215 meters, with varying drift during each cast.
CTD modulo errors began to appear during casts 3 and 4, indicating intermittent
communication issues between the CTD and the deck unit. The frequency of these
errors increased substantially during subsequent casts near the WHOTS\sphinxhyphen{}19
mooring, ultimately leading to aborted casts. The issue persisted even after
re\sphinxhyphen{}terminating the CTD cable and replacing the primary CTD with a spare unit,
suggesting the fault likely resided in the CTD wire or the slip\sphinxhyphen{}rings. As a
result, no CTD casts could be completed near the WHOTS\sphinxhyphen{}19 mooring.

\sphinxAtStartPar
Between 2 and 4 water samples were taken from all casts. These samples were to
be analyzed for salinity at UH and used to calibrate the CTD conductivity
sensors.


\begin{savenotes}\sphinxattablestart
\centering
\sphinxcapstartof{table}
\sphinxthecaptionisattop
\sphinxcaption{CTD stations occupied during the WHOTS\sphinxhyphen{}19 cruise (Datetime is in mm/dd/yyyy hh:mm)}\label{\detokenize{2_section:table-2}}
\sphinxaftertopcaption
\begin{tabulary}{\linewidth}[t]{|T|T|T|T|T|}
\hline
\sphinxstyletheadfamily 
\sphinxAtStartPar
\sphinxstylestrong{Station/cast}
&\sphinxstyletheadfamily 
\sphinxAtStartPar
\sphinxstylestrong{Date}
&\sphinxstyletheadfamily 
\sphinxAtStartPar
\sphinxstylestrong{In\sphinxhyphen{}water Time}
&\sphinxstyletheadfamily 
\sphinxAtStartPar
\sphinxstylestrong{Location}
&\sphinxstyletheadfamily 
\sphinxAtStartPar
\sphinxstylestrong{Maximum pressure (dbar)}
\\
\hline
\sphinxAtStartPar
20/1
&
\sphinxAtStartPar
06/16/2023
&
\sphinxAtStartPar
07:01
&
\sphinxAtStartPar
21°28.052´N, 158°21.087´W
&
\sphinxAtStartPar
1489
\\
\hline
\sphinxAtStartPar
52/1
&
\sphinxAtStartPar
06/18/2023
&
\sphinxAtStartPar
16:05
&
\sphinxAtStartPar
22°40.680´N, 157°58.892´W
&
\sphinxAtStartPar
215
\\
\hline
\sphinxAtStartPar
52/2
&
\sphinxAtStartPar
06/18/2023
&
\sphinxAtStartPar
20:02
&
\sphinxAtStartPar
22°40.888´N, 157°58.803´W
&
\sphinxAtStartPar
203
\\
\hline
\sphinxAtStartPar
52/3
&
\sphinxAtStartPar
06/19/2023
&
\sphinxAtStartPar
00:05
&
\sphinxAtStartPar
22°40.707´N, 157°58.954´W
&
\sphinxAtStartPar
202
\\
\hline
\sphinxAtStartPar
52/4
&
\sphinxAtStartPar
06/19/2023
&
\sphinxAtStartPar
05:22
&
\sphinxAtStartPar
22°41.226´N, 157°58.574´W
&
\sphinxAtStartPar
200
\\
\hline
\end{tabulary}
\par
\sphinxattableend\end{savenotes}

\sphinxAtStartPar
Also, continuous ADCP and near\sphinxhyphen{}surface thermosalinograph data were obtained
while underway.

\sphinxAtStartPar
The R/V Oscar Elton Settle was equipped with a RDI Ocean Surveyor 75 kHz ADCP,
set to function in broadband and narrowband configurations. The configuration
information is shown in \hyperref[\detokenize{2_section:table-3}]{Table \ref{\detokenize{2_section:table-3}}}. The ADCP used input from a SAMOS
gyrometer and Furuno GP 150, a GPS receiver, to establish the ship’s heading
and attitude.


\begin{savenotes}\sphinxattablestart
\centering
\sphinxcapstartof{table}
\sphinxthecaptionisattop
\sphinxcaption{Configuration of the Ocean Surveyor 75kHz ADCP on board the Ship Oscar Sette during the WHOTS\sphinxhyphen{}19 cruise}\label{\detokenize{2_section:table-3}}
\sphinxaftertopcaption
\begin{tabulary}{\linewidth}[t]{|T|T|T|}
\hline
\sphinxstyletheadfamily 
\sphinxAtStartPar
\sphinxstylestrong{Parameters}
&\sphinxstyletheadfamily 
\sphinxAtStartPar
\sphinxstylestrong{OS75BB}
&\sphinxstyletheadfamily 
\sphinxAtStartPar
\sphinxstylestrong{OS75NB}
\\
\hline
\sphinxAtStartPar
Sample interval (s)
&
\sphinxAtStartPar
300
&
\sphinxAtStartPar
300
\\
\hline
\sphinxAtStartPar
Number of bins
&
\sphinxAtStartPar
80
&
\sphinxAtStartPar
55
\\
\hline
\sphinxAtStartPar
Bin Length (m)
&
\sphinxAtStartPar
8
&
\sphinxAtStartPar
16
\\
\hline
\sphinxAtStartPar
Transducer depth (m)
&
\sphinxAtStartPar
5
&
\sphinxAtStartPar
5
\\
\hline
\sphinxAtStartPar
Blanking length (m)
&
\sphinxAtStartPar
8
&
\sphinxAtStartPar
8
\\
\hline
\end{tabulary}
\par
\sphinxattableend\end{savenotes}

\sphinxAtStartPar
Near\sphinxhyphen{}surface temperature and salinity data during the WHOTS\sphinxhyphen{}19 cruise were
collected using the thermosalinograph (TSG) system installed aboard the NOAA
Ship Oscar Elton Sette. The system sampled seawater from the ship’s continuous
intake line and included the following instruments: an SBE\sphinxhyphen{}21 thermosalinograph
(SN 3168), an SBE\sphinxhyphen{}45 micro\sphinxhyphen{}thermosalinograph (SN 0290), and an external SBE\sphinxhyphen{}38
temperature sensor (SN 212). The SBE\sphinxhyphen{}21 and SBE\sphinxhyphen{}45 units, each equipped with
internal temperature and conductivity sensors, were located in the ship’s
chemistry lab , approximately 70 meters from the hull intake. The
SBE\sphinxhyphen{}38 was mounted externally at the seawater intake, positioned on the
starboard side of the bow, forward of the bow thruster, at a depth of
approximately 3 meters. All sensors recorded data every second. The
system included a flow meter located in the chemistry lab, which indicated a
flow rate of approximately 1.1 liters per minute throughout the cruise. Among
the instruments, only the SBE\sphinxhyphen{}45 was equipped with a debubbler. To validate and
correct for potential drift in salinity measurements, seawater samples were
collected from the chemistry lab exhaust line every 8 hours. These samples were
stored in 0.25\sphinxhyphen{}liter glass bottles and analyzed post\sphinxhyphen{}cruise at the University
of Hawai‘i laboratory.


\section{WHOTS\sphinxhyphen{}20 Cruise: WHOTS\sphinxhyphen{}19 Mooring Recovery}
\label{\detokenize{2_section:whots-20-cruise-whots-19-mooring-recovery}}
\sphinxAtStartPar
The WHOI/UOP group carried out the mooring turnaround operations during the
WHOTS\sphinxhyphen{}20 cruise, which took place from May 31 to June 8, 2024. The WHOTS\sphinxhyphen{}20
mooring was successfully deployed at Station 52 on June 2, 2024, at 03:54 UTC,
at coordinates 22°40.100′N, 157°56.801′W. The WHOTS\sphinxhyphen{}19 mooring was recovered on
June 6, 2024, at 18:50 UTC. The scientific personnel who participated in the
cruise are listed in \hyperref[\detokenize{2_section:table-4}]{Table \ref{\detokenize{2_section:table-4}}}.


\begin{savenotes}\sphinxattablestart
\centering
\sphinxcapstartof{table}
\sphinxthecaptionisattop
\sphinxcaption{Scientific personnel on Ship Oscar Sette during the WHOTS\sphinxhyphen{}20 deployment cruise.}\label{\detokenize{2_section:table-4}}
\sphinxaftertopcaption
\begin{tabulary}{\linewidth}[t]{|T|T|T|}
\hline
\sphinxstyletheadfamily 
\sphinxAtStartPar
\sphinxstylestrong{Name}
&\sphinxstyletheadfamily 
\sphinxAtStartPar
\sphinxstylestrong{Title or function}
&\sphinxstyletheadfamily 
\sphinxAtStartPar
\sphinxstylestrong{Affiliation}
\\
\hline
\sphinxAtStartPar
Bigorre, Sebastien
&
\sphinxAtStartPar
Chief Scientist
&
\sphinxAtStartPar
WHOI
\\
\hline
\sphinxAtStartPar
Llanos, Nico
&
\sphinxAtStartPar
Engineer
&
\sphinxAtStartPar
WHOI
\\
\hline
\sphinxAtStartPar
Graham, Raymond
&
\sphinxAtStartPar
Engineer
&
\sphinxAtStartPar
WHOI
\\
\hline
\sphinxAtStartPar
Fitzgerald, Dan
&
\sphinxAtStartPar
Marine Electronics Technician
&
\sphinxAtStartPar
UH
\\
\hline
\sphinxAtStartPar
Santiago\sphinxhyphen{}Mandujano, Fernando
&
\sphinxAtStartPar
Research Associate
&
\sphinxAtStartPar
UH
\\
\hline
\sphinxAtStartPar
Rohrer, Tully
&
\sphinxAtStartPar
Research Associate
&
\sphinxAtStartPar
UH
\\
\hline
\sphinxAtStartPar
Shepherd, Merritt
&
\sphinxAtStartPar
Graduate Student
&
\sphinxAtStartPar
UH
\\
\hline
\sphinxAtStartPar
Maloney, Kelsey
&
\sphinxAtStartPar
HIMB Housing Coordinator
&
\sphinxAtStartPar
UH
\\
\hline
\sphinxAtStartPar
Prajna, Jandial
&
\sphinxAtStartPar
Graduate Student
&
\sphinxAtStartPar
UH
\\
\hline
\sphinxAtStartPar
Dirks, Jonah
&
\sphinxAtStartPar
Graduate Student
&
\sphinxAtStartPar
UH
\\
\hline
\end{tabulary}
\par
\sphinxattableend\end{savenotes}

\sphinxAtStartPar
The UH group conducted the shipboard oceanographic observations during the
cruise. A complete description of these operations is available in the WHOTS\sphinxhyphen{}20
cruise report {[}\hyperlink{cite.references:id18}{Santiago\sphinxhyphen{}Mandujano \sphinxstyleemphasis{et al.}, 2024}{]}.

\sphinxAtStartPar
A Sea\sphinxhyphen{}Bird CTD (Conductivity, Temperature, and Depth) system was used to
collect temperature (T), salinity (S), and dissolved oxygen (O\(\sb{\text{2}}\)) profiles
during CTD casts. The time, location, and maximum pressure for each cast are
shown in \hyperref[\detokenize{2_section:table-5}]{Table \ref{\detokenize{2_section:table-5}}}.

\sphinxAtStartPar
Nine CTD casts were conducted during the WHOTS\sphinxhyphen{}20 cruise between June 1 and
June 6, 2024. CTD profiles were collected at Station 20 (en route to the WHOTS
mooring), Station 50 (near the WHOTS\sphinxhyphen{}19 buoy), and Station 52 (near the
WHOTS\sphinxhyphen{}20 buoy). The cast at Station 20 reached a depth of approximately 1500 m
and included three acoustic releases—two intended for the WHOTS\sphinxhyphen{}20 mooring and
one backup—secured to the rosette frame for functionality testing. This cast
exhibited anomalous values in the primary temperature, conductivity, and oxygen
sensors. Post\sphinxhyphen{}cast inspection revealed that the CTD pump cable had been pinched
beneath a hose clamp on one of the Niskin bottles. The secondary sensor suite
was unaffected. Four CTD yo\sphinxhyphen{}yo casts were conducted near the WHOTS\sphinxhyphen{}19 mooring
prior to recovery to provide comparison profiles for the subsurface
instruments, and three additional yo\sphinxhyphen{}yo casts were conducted near the WHOTS\sphinxhyphen{}20
mooring after deployment. All yo\sphinxhyphen{}yo casts began approximately 0.25 nautical
miles from the buoys and consisted of five vertical cycles between 5 m and
205 m depth, with varying drift during each cast.

\sphinxAtStartPar
The first two yo\sphinxhyphen{}yo casts near the WHOTS\sphinxhyphen{}19 mooring (S50C1 and S50C2) displayed
anomalous data at the start of each profile, likely due to air bubbles in the
CTD plumbing system. These artifacts disappeared below 100 m. A Y\sphinxhyphen{}shaped
plastic “de\sphinxhyphen{}bubbler” was subsequently installed in both CTD plumbing lines to
mitigate this issue. All subsequent casts returned high\sphinxhyphen{}quality data.

\sphinxAtStartPar
One deep CTD cast was conducted approximately 2 nautical miles from the
WHOTS\sphinxhyphen{}19 buoy to obtain a profile for comparison with the near\sphinxhyphen{}bottom MicroCAT
sensors. This cast reached a maximum depth of nearly 4550 m. However, due to
the absence of a functioning altimeter onboard, it was not possible to safely
approach the depth of the deepest MicroCATs, located at approximately 4659 m.

\sphinxAtStartPar
Between four and five water samples were collected during each cast using
0.25\sphinxhyphen{}liter glass bottles. These samples were returned to the University of
Hawai‘i for salinity analysis and were used to calibrate the CTD
conductivity sensors.


\begin{savenotes}\sphinxattablestart
\centering
\sphinxcapstartof{table}
\sphinxthecaptionisattop
\sphinxcaption{CTD stations during the WHOTS\sphinxhyphen{}20 cruise (WHOTS\sphinxhyphen{}19 mooring recovery). Datetime is in UTC (mm/dd/yy hh:mm).}\label{\detokenize{2_section:table-5}}
\sphinxaftertopcaption
\begin{tabulary}{\linewidth}[t]{|T|T|T|T|T|}
\hline
\sphinxstyletheadfamily 
\sphinxAtStartPar
\sphinxstylestrong{Station/cast}
&\sphinxstyletheadfamily 
\sphinxAtStartPar
\sphinxstylestrong{Date}
&\sphinxstyletheadfamily 
\sphinxAtStartPar
\sphinxstylestrong{In\sphinxhyphen{}water Time}
&\sphinxstyletheadfamily 
\sphinxAtStartPar
\sphinxstylestrong{Location}
&\sphinxstyletheadfamily 
\sphinxAtStartPar
\sphinxstylestrong{Maximum pressure (dbar)}
\\
\hline
\sphinxAtStartPar
20/1
&
\sphinxAtStartPar
6/1/2024
&
\sphinxAtStartPar
02:55
&
\sphinxAtStartPar
21°17.112′N, 158°19.541′W
&
\sphinxAtStartPar
1529
\\
\hline
\sphinxAtStartPar
50/1
&
\sphinxAtStartPar
6/3/2024
&
\sphinxAtStartPar
16:07
&
\sphinxAtStartPar
22°46.880′N, 157°55.793′W
&
\sphinxAtStartPar
204
\\
\hline
\sphinxAtStartPar
50/2
&
\sphinxAtStartPar
6/3/2024
&
\sphinxAtStartPar
20:04
&
\sphinxAtStartPar
22°46.221′N, 157°56.073′W
&
\sphinxAtStartPar
202
\\
\hline
\sphinxAtStartPar
50/3
&
\sphinxAtStartPar
6/4/2024
&
\sphinxAtStartPar
00:08
&
\sphinxAtStartPar
22°46.396′N, 157°56.165′W
&
\sphinxAtStartPar
205
\\
\hline
\sphinxAtStartPar
50/4
&
\sphinxAtStartPar
6/4/2024
&
\sphinxAtStartPar
04:02
&
\sphinxAtStartPar
22°46.339′N, 157°56.148′W
&
\sphinxAtStartPar
202
\\
\hline
\sphinxAtStartPar
52/1
&
\sphinxAtStartPar
6/4/2024
&
\sphinxAtStartPar
20:37
&
\sphinxAtStartPar
22°40.237′N, 157°59.375′W
&
\sphinxAtStartPar
203
\\
\hline
\sphinxAtStartPar
52/2
&
\sphinxAtStartPar
6/4/2024
&
\sphinxAtStartPar
23:58
&
\sphinxAtStartPar
22°40.237′N, 157°59.271′W
&
\sphinxAtStartPar
202
\\
\hline
\sphinxAtStartPar
52/3
&
\sphinxAtStartPar
6/5/2024
&
\sphinxAtStartPar
04:04
&
\sphinxAtStartPar
22°40.430′N, 157°59.022′W
&
\sphinxAtStartPar
204
\\
\hline
\sphinxAtStartPar
50/5
&
\sphinxAtStartPar
6/6/2024
&
\sphinxAtStartPar
00:09
&
\sphinxAtStartPar
22°47.311′N, 157°57.818′W
&
\sphinxAtStartPar
4625
\\
\hline
\end{tabulary}
\par
\sphinxattableend\end{savenotes}

\sphinxAtStartPar
Also, continuous ADCP and near\sphinxhyphen{}surface thermosalinograph data were obtained
while underway.

\sphinxAtStartPar
The NOAA Ship Oscar Elton Sette was equipped with a Teledyne RDI Ocean Surveyor
75 kHz Acoustic Doppler Current Profiler (ADCP), configured to operate in
narrowband mode. The broadband mode was non\sphinxhyphen{}functional during this cruise.
Configuration details are provided in \hyperref[\detokenize{2_section:table-6}]{Table \ref{\detokenize{2_section:table-6}}}. The ADCP relied on input
from a SAMOS gyrometer and a Furuno GP\sphinxhyphen{}170 GPS receiver to determine the ship’s
heading and attitude.


\begin{savenotes}\sphinxattablestart
\centering
\sphinxcapstartof{table}
\sphinxthecaptionisattop
\sphinxcaption{Configuration of the Ocean Surveyor 75kHz ADCP on board the Ship Oscar Sette during the WHOTS\sphinxhyphen{}20 cruise}\label{\detokenize{2_section:table-6}}
\sphinxaftertopcaption
\begin{tabulary}{\linewidth}[t]{|T|T|}
\hline
\sphinxstyletheadfamily 
\sphinxAtStartPar
\sphinxstylestrong{Parameters}
&\sphinxstyletheadfamily 
\sphinxAtStartPar
\sphinxstylestrong{OS75NB}
\\
\hline
\sphinxAtStartPar
Sample interval (s)
&
\sphinxAtStartPar
300
\\
\hline
\sphinxAtStartPar
Number of bins
&
\sphinxAtStartPar
55
\\
\hline
\sphinxAtStartPar
Bin Length (m)
&
\sphinxAtStartPar
16
\\
\hline
\sphinxAtStartPar
Transducer depth (m)
&
\sphinxAtStartPar
5
\\
\hline
\sphinxAtStartPar
Blanking length (m)
&
\sphinxAtStartPar
8
\\
\hline
\end{tabulary}
\par
\sphinxattableend\end{savenotes}

\sphinxAtStartPar
Near\sphinxhyphen{}surface salinity data during the WHOTS\sphinxhyphen{}20 cruise were collected using the
thermosalinograph (TSG) system installed aboard the NOAA Ship Oscar Elton
Sette. The system sampled water from the ship’s continuous seawater supply and
consisted of a Sea\sphinxhyphen{}Bird SBE\sphinxhyphen{}45 micro\sphinxhyphen{}thermosalinograph (SN 0290), equipped with
internal temperature and conductivity sensors located in the ship’s chemistry
lab, approximately 70 meters from the hull intake.

\sphinxAtStartPar
A second TSG unit, the Sea\sphinxhyphen{}Bird SBE\sphinxhyphen{}21, was installed but was non\sphinxhyphen{}functional
throughout the cruise. Similarly, the SBE\sphinxhyphen{}38 remote temperature sensor, mounted
at the water intake near the ship’s bow, was also inoperative. As a result, sea
surface temperature data could not be collected during the cruise.

\sphinxAtStartPar
The SBE\sphinxhyphen{}45 recorded data at 1 Hz and included a built\sphinxhyphen{}in debubbler. The
seawater intake is located at the bow of the vessel, forward of the
starboard\sphinxhyphen{}side bow thruster, at a depth of 3 meters. Flow through the system
was monitored via a flow meter in the chemistry lab, which registered a rate of
approximately 1.5 liters per minute during the cruise.

\sphinxAtStartPar
To correct for potential drift in conductivity measurements, discrete salinity
samples were collected every 8 hours from the exhaust line in the chemistry lab
using 0.25\sphinxhyphen{}liter glass bottles. These samples were returned to the University
of Hawai‘i laboratory for post\sphinxhyphen{}cruise salinity analysis.


\chapter{Description of WHOTS\sphinxhyphen{}19 Mooring}
\label{\detokenize{3_section:description-of-whots-19-mooring}}\label{\detokenize{3_section::doc}}
\sphinxAtStartPar
The WHOTS\sphinxhyphen{}19 mooring was deployed on June 17, 2023, from the NOAA Ship \sphinxstyleemphasis{Oscar
Elton Sette} and recovered on June 6, 2024. The mooring consisted of a surface
buoy equipped with two complete sets of Air\textendash{}Sea Interaction Meteorological
(ASIMET) sensors and a suite of subsurface oceanographic instruments mounted
along the bridle and mooring line down to 155 m, with additional sensors near
the bottom. For a detailed technical description of the mooring design, see
{[}\hyperlink{cite.references:id17}{Santiago\sphinxhyphen{}Mandujano \sphinxstyleemphasis{et al.}, 2024}, \hyperlink{cite.references:id18}{Santiago\sphinxhyphen{}Mandujano \sphinxstyleemphasis{et al.}, 2024}{]}.

\sphinxAtStartPar
The WHOTS\sphinxhyphen{}19 mooring was designed to collect high\sphinxhyphen{}quality time series of
surface meteorology and upper\sphinxhyphen{}ocean temperature, salinity, and biogeochemical
variables in support of long\sphinxhyphen{}term air\sphinxhyphen{}sea flux studies at Station ALOHA.


\section{Surface Components}
\label{\detokenize{3_section:surface-components}}
\sphinxAtStartPar
The buoy included the following instrumentation mounted on a blue hull with a
white tower and yellow deck (\hyperref[\detokenize{3_section:table-surface-components}]{Table \ref{\detokenize{3_section:table-surface-components}}}):


\begin{savenotes}\sphinxattablestart
\centering
\sphinxcapstartof{table}
\sphinxthecaptionisattop
\sphinxcaption{Surface meteorological instrumentation on the WHOTS\sphinxhyphen{}19 buoy, organized by system side (port = System 1, starboard = System 2). All sensors were mounted on the buoy tower. Heights are reported relative to the buoy deck.}\label{\detokenize{3_section:table-surface-components}}
\sphinxaftertopcaption
\begin{tabulary}{\linewidth}[t]{|T|T|}
\hline
\sphinxstyletheadfamily 
\sphinxAtStartPar
\sphinxstylestrong{Instrument Type}
&\sphinxstyletheadfamily 
\sphinxAtStartPar
\sphinxstylestrong{IDs / Details}
\\
\hline
\sphinxAtStartPar
\sphinxstylestrong{Data Loggers}
&
\sphinxAtStartPar
9 (System 1, port side); ID 42 (System 2, starboard side)
\\
\hline
\sphinxAtStartPar
\sphinxstylestrong{Relative Humidity and Air Temperature (HRH) Sensors}
&
\sphinxAtStartPar
269 and 247, mounted at 231 cm
\\
\hline
\sphinxAtStartPar
\sphinxstylestrong{Barometric Pressure Recorders (BPR)}
&
\sphinxAtStartPar
210 and 206, mounted at 242 cm
\\
\hline
\sphinxAtStartPar
\sphinxstylestrong{Wind Sensors (RM Young)}
&
\sphinxAtStartPar
225 and 701, mounted at 264 cm
\\
\hline
\sphinxAtStartPar
\sphinxstylestrong{Precipitation Sensors (PRC)}
&
\sphinxAtStartPar
235 (System 1 \textendash{} top broken before recovery); ID 506, both mounted at 256 cm
\\
\hline
\sphinxAtStartPar
\sphinxstylestrong{Longwave Radiation Sensors (LWR)}
&
\sphinxAtStartPar
214 and 219, mounted at 281 cm
\\
\hline
\sphinxAtStartPar
\sphinxstylestrong{Shortwave Radiation Sensors (SWR)}
&
\sphinxAtStartPar
233 and 373, mounted at 281 cm
\\
\hline
\sphinxAtStartPar
\sphinxstylestrong{Sea Surface Temperature (SST)}
&
\sphinxAtStartPar
1727 and 5996, mounted at \textendash{}155 cm
\\
\hline
\sphinxAtStartPar
\sphinxstylestrong{Iridium Satellite Systems}
&
\sphinxAtStartPar
IMEIs: 300234063855630, 300234063160290
\\
\hline
\sphinxAtStartPar
\sphinxstylestrong{GPS \textendash{} Melo}
&
\sphinxAtStartPar
IMEI: 300034013707580
\\
\hline
\sphinxAtStartPar
\sphinxstylestrong{GPS \textendash{} Rover}
&
\sphinxAtStartPar
1034 and 724; IMEIs: 300434064530400, 300434063547190
\\
\hline
\sphinxAtStartPar
\sphinxstylestrong{Other Sensors}
&
\sphinxAtStartPar
Vaisala WXT (ID 204, 264 cm); Airmar (ID 6079L253, 273 cm, hanging); HC2A (ID WI9, 232 cm); SBE39AT (ID 719, 228 cm)
\\
\hline
\end{tabulary}
\par
\sphinxattableend\end{savenotes}


\section{Subsurface Instrumentation}
\label{\detokenize{3_section:subsurface-instrumentation}}
\sphinxAtStartPar
Subsurface instruments were mounted along the bridle and at a fixed depth on
the mooring line, as follows (\hyperref[\detokenize{3_section:table-subsurface}]{Table \ref{\detokenize{3_section:table-subsurface}}}):


\begin{savenotes}\sphinxattablestart
\centering
\sphinxcapstartof{table}
\sphinxthecaptionisattop
\sphinxcaption{Additional shared surface instrumentation on the WHOTS\sphinxhyphen{}19 buoy, including GPS units, Iridium satellite modems, and auxiliary sensors.}\label{\detokenize{3_section:table-subsurface}}
\sphinxaftertopcaption
\begin{tabulary}{\linewidth}[t]{|T|T|T|T|}
\hline
\sphinxstyletheadfamily 
\sphinxAtStartPar
\sphinxstylestrong{Instrument Type}
&\sphinxstyletheadfamily 
\sphinxAtStartPar
\sphinxstylestrong{ID}
&\sphinxstyletheadfamily 
\sphinxAtStartPar
\sphinxstylestrong{Depth}
&\sphinxstyletheadfamily 
\sphinxAtStartPar
\sphinxstylestrong{Notes}
\\
\hline
\sphinxAtStartPar
MAPCO\(\sb{\text{2}}\) System
&
\sphinxAtStartPar
26
&
\sphinxAtStartPar
155
&
\sphinxAtStartPar
With equilibration tube
\\
\hline
\sphinxAtStartPar
CTD (SBE16)
&
\sphinxAtStartPar
6832
&
\sphinxAtStartPar
155
&
\sphinxAtStartPar

\\
\hline
\sphinxAtStartPar
Oxygen Sensor
&
\sphinxAtStartPar
1381
&
\sphinxAtStartPar
155
&
\sphinxAtStartPar

\\
\hline
\sphinxAtStartPar
Fluorometer
&
\sphinxAtStartPar
2597
&
\sphinxAtStartPar
155
&
\sphinxAtStartPar
Includes bio\sphinxhyphen{}wiper
\\
\hline
\sphinxAtStartPar
pH Sensor (SAMI)
&
\sphinxAtStartPar
P246
&
\sphinxAtStartPar
155
&
\sphinxAtStartPar

\\
\hline
\sphinxAtStartPar
Span Gas Canister
&
\sphinxAtStartPar
JB03812
&
\sphinxAtStartPar
—
&
\sphinxAtStartPar
For sensor calibration
\\
\hline
\sphinxAtStartPar
Xeos Kilo Transmitter
&
\sphinxAtStartPar
1494
&
\sphinxAtStartPar
—
&
\sphinxAtStartPar
IMEI: 300234062727610
\\
\hline
\end{tabulary}
\par
\sphinxattableend\end{savenotes}

\sphinxAtStartPar
Five internally logging Sea\sphinxhyphen{}Bird SBE\sphinxhyphen{}56 temperature sensors were bolted to the
buoy hull’s underside, measuring sea surface temperature and salinity.
The SBE\sphinxhyphen{}56s measured SST once every 60 sec between 80\sphinxhyphen{}110 cm below the surface.
Two SBE\sphinxhyphen{}37 MicroCATs were at 1.55m measuring at every 300s (See
\hyperref[\detokenize{3_section:table-7}]{Table \ref{\detokenize{3_section:table-7}}}).


\begin{savenotes}\sphinxattablestart
\centering
\sphinxcapstartof{table}
\sphinxthecaptionisattop
\sphinxcaption{WHOTS\sphinxhyphen{}19 MicroCAT and SBE\sphinxhyphen{}56 Temperature Sensor Information.}\label{\detokenize{3_section:table-7}}
\sphinxaftertopcaption
\begin{tabulary}{\linewidth}[t]{|T|T|T|T|}
\hline
\sphinxstyletheadfamily 
\sphinxAtStartPar
\sphinxstylestrong{Instrument}
&\sphinxstyletheadfamily 
\sphinxAtStartPar
\sphinxstylestrong{SN}
&\sphinxstyletheadfamily 
\sphinxAtStartPar
\sphinxstylestrong{Depth (m)}
&\sphinxstyletheadfamily 
\sphinxAtStartPar
\sphinxstylestrong{Sample Interval (sec)}
\\
\hline
\sphinxAtStartPar
SBE\sphinxhyphen{}56
&
\sphinxAtStartPar
6410
&
\sphinxAtStartPar
0.8
&
\sphinxAtStartPar
60
\\
\hline
\sphinxAtStartPar
SBE\sphinxhyphen{}56
&
\sphinxAtStartPar
6239
&
\sphinxAtStartPar
0.8
&
\sphinxAtStartPar
60
\\
\hline
\sphinxAtStartPar
SBE\sphinxhyphen{}56
&
\sphinxAtStartPar
6412
&
\sphinxAtStartPar
1.0
&
\sphinxAtStartPar
60
\\
\hline
\sphinxAtStartPar
SBE\sphinxhyphen{}56
&
\sphinxAtStartPar
6983
&
\sphinxAtStartPar
1.1
&
\sphinxAtStartPar
60
\\
\hline
\sphinxAtStartPar
SBE\sphinxhyphen{}56
&
\sphinxAtStartPar
7211
&
\sphinxAtStartPar
0.8
&
\sphinxAtStartPar
60
\\
\hline
\sphinxAtStartPar
SBE\sphinxhyphen{}37
&
\sphinxAtStartPar
1727
&
\sphinxAtStartPar
1.55
&
\sphinxAtStartPar
300
\\
\hline
\sphinxAtStartPar
SBE\sphinxhyphen{}37
&
\sphinxAtStartPar
5996
&
\sphinxAtStartPar
1.55
&
\sphinxAtStartPar
300
\\
\hline
\end{tabulary}
\par
\sphinxattableend\end{savenotes}

\sphinxAtStartPar
Instrumentation provided by UH for the WHOTS\sphinxhyphen{}19 mooring included 20 SBE\sphinxhyphen{}37
Microcats, and two upward\sphinxhyphen{}looking RDI Workhorse ADCPs, transmitting in 300 kHz
and 600 kHz,respectively. The Microcats all measured temperature and
conductivity, with ten of them measuring pressure. All MicroCATs were deployed
with antifoulant capsules. In addition to the buoy instrumentation, WHOI
provided two Vector Measuring Current Meters (VMCMs), four deep Microcats (
SBE\sphinxhyphen{}37) installed at 1875 m and near the bottom of the mooring, and all
required subsurface mooring hardware.

\sphinxAtStartPar
The \hyperref[\detokenize{3_section:mooring-subsurface}]{Table \ref{\detokenize{3_section:mooring-subsurface}}} provides a listing of the WHOTS\sphinxhyphen{}19 subsurface
instrumentation at their nominal depths on the mooring, along with serial
numbers, sampling rates, and other pertinent information. A cold water spike
was induced to the UH MicroCATs before deployment
(\hyperref[\detokenize{3_section:mooring-subsurface}]{Table \ref{\detokenize{3_section:mooring-subsurface}}}) and after recovery \hyperref[\detokenize{3_section:table-8}]{Table \ref{\detokenize{3_section:table-8}}} by placing
an ice pack in contact with their temperature sensor to check for any drift in
their internal clock. To produce a spike in the ADCP data, each instrument’s
transducer was rubbed gently by hand for 20 seconds (\hyperref[\detokenize{3_section:table-9}]{Table \ref{\detokenize{3_section:table-9}}}
, and \hyperref[\detokenize{3_section:table-10}]{Table \ref{\detokenize{3_section:table-10}}}).

\begin{landscape}


\begin{savenotes}\sphinxattablestart
\centering
\sphinxcapstartof{table}
\sphinxthecaptionisattop
\sphinxcaption{WHOTS\sphinxhyphen{}19 mooring subsurface instrument deployment. All times are in UTC (MM/DD/YY hh:mm:ss) for the year 2023. Columns: SN = Serial Number, Inst = Instrument, Z (m) = Depth in meters, P SN = Pressure Sensor Serial Number, Δt (s) = Sample Interval in seconds, AR = Acoustic Receiver, MC = MicroCAT, VMCM = Vector Measuring Current Meter, ADCP = Acoustic Doppler Current Profiler.}\label{\detokenize{3_section:mooring-subsurface}}
\sphinxaftertopcaption
\begin{tabulary}{\linewidth}[t]{|T|T|T|T|T|T|T|T|T|}
\hline
\sphinxstyletheadfamily 
\sphinxAtStartPar
SN
&\sphinxstyletheadfamily 
\sphinxAtStartPar
Inst
&\sphinxstyletheadfamily 
\sphinxAtStartPar
Z (m)
&\sphinxstyletheadfamily 
\sphinxAtStartPar
P SN
&\sphinxstyletheadfamily 
\sphinxAtStartPar
Δt (s)
&\sphinxstyletheadfamily 
\sphinxAtStartPar
Start
&\sphinxstyletheadfamily 
\sphinxAtStartPar
Spike Start
&\sphinxstyletheadfamily 
\sphinxAtStartPar
Spike End
&\sphinxstyletheadfamily 
\sphinxAtStartPar
In Water
\\
\hline
\sphinxAtStartPar
3382
&
\sphinxAtStartPar
MC
&
\sphinxAtStartPar
7
&
\sphinxAtStartPar
N/A
&
\sphinxAtStartPar
180
&
\sphinxAtStartPar
06/13 23:59
&
\sphinxAtStartPar
06/16 00:00
&
\sphinxAtStartPar
06/16 00:30
&
\sphinxAtStartPar
06/16 21:00
\\
\hline
\sphinxAtStartPar
0035
&
\sphinxAtStartPar
VMCM
&
\sphinxAtStartPar
10
&
\sphinxAtStartPar
N/A
&
\sphinxAtStartPar
60
&
\sphinxAtStartPar
06/08 00:21
&
\sphinxAtStartPar
06/16 02:03
&
\sphinxAtStartPar
N/A
&
\sphinxAtStartPar
06/16 20:28
\\
\hline
\sphinxAtStartPar
6892
&
\sphinxAtStartPar
MC
&
\sphinxAtStartPar
15
&
\sphinxAtStartPar
2651234
&
\sphinxAtStartPar
75
&
\sphinxAtStartPar
06/13 23:59
&
\sphinxAtStartPar
06/16 00:00
&
\sphinxAtStartPar
06/16 00:30
&
\sphinxAtStartPar
06/16 20:19
\\
\hline
\sphinxAtStartPar
4663
&
\sphinxAtStartPar
MC
&
\sphinxAtStartPar
25
&
\sphinxAtStartPar
N/A
&
\sphinxAtStartPar
180
&
\sphinxAtStartPar
06/13 23:59
&
\sphinxAtStartPar
06/16 00:00
&
\sphinxAtStartPar
06/16 00:30
&
\sphinxAtStartPar
06/16 20:19
\\
\hline
\sphinxAtStartPar
112497
&
\sphinxAtStartPar
AR
&
\sphinxAtStartPar
25
&
\sphinxAtStartPar
N/A
&
\sphinxAtStartPar
N/A
&
\sphinxAtStartPar
06/12 19:00
&
\sphinxAtStartPar
N/A
&
\sphinxAtStartPar
N/A
&
\sphinxAtStartPar
06/16 20:19
\\
\hline
\sphinxAtStartPar
0058
&
\sphinxAtStartPar
VMCM
&
\sphinxAtStartPar
30
&
\sphinxAtStartPar
N/A
&
\sphinxAtStartPar
60
&
\sphinxAtStartPar
06/08 00:24
&
\sphinxAtStartPar
06/16 02:13
&
\sphinxAtStartPar
N/A
&
\sphinxAtStartPar
06/16 19:32
\\
\hline
\sphinxAtStartPar
3533
&
\sphinxAtStartPar
MC
&
\sphinxAtStartPar
35
&
\sphinxAtStartPar
N/A
&
\sphinxAtStartPar
180
&
\sphinxAtStartPar
06/13 23:59
&
\sphinxAtStartPar
06/16 00:00
&
\sphinxAtStartPar
06/16 00:30
&
\sphinxAtStartPar
06/16 20:12
\\
\hline
\sphinxAtStartPar
3381
&
\sphinxAtStartPar
MC
&
\sphinxAtStartPar
40
&
\sphinxAtStartPar
N/A
&
\sphinxAtStartPar
180
&
\sphinxAtStartPar
06/13 23:59
&
\sphinxAtStartPar
06/16 00:00
&
\sphinxAtStartPar
06/16 00:30
&
\sphinxAtStartPar
06/16 20:09
\\
\hline
\sphinxAtStartPar
3668
&
\sphinxAtStartPar
MC
&
\sphinxAtStartPar
45
&
\sphinxAtStartPar
5579
&
\sphinxAtStartPar
180
&
\sphinxAtStartPar
06/16 00:00
&
\sphinxAtStartPar
06/16 01:04
&
\sphinxAtStartPar
06/16 02:00
&
\sphinxAtStartPar
06/16 21:03
\\
\hline
\sphinxAtStartPar
13917
&
\sphinxAtStartPar
ADCP 600kHz
&
\sphinxAtStartPar
47.5
&
\sphinxAtStartPar
N/A
&
\sphinxAtStartPar
600
&
\sphinxAtStartPar
06/15 23:59
&
\sphinxAtStartPar
See Table
&
\sphinxAtStartPar
See Table
&
\sphinxAtStartPar
06/16 21:23
\\
\hline
\sphinxAtStartPar
3619
&
\sphinxAtStartPar
MC
&
\sphinxAtStartPar
50
&
\sphinxAtStartPar
N/A
&
\sphinxAtStartPar
180
&
\sphinxAtStartPar
06/13 23:59
&
\sphinxAtStartPar
06/16 00:00
&
\sphinxAtStartPar
06/16 00:30
&
\sphinxAtStartPar
06/16 21:04
\\
\hline
\sphinxAtStartPar
3620
&
\sphinxAtStartPar
MC
&
\sphinxAtStartPar
55
&
\sphinxAtStartPar
N/A
&
\sphinxAtStartPar
180
&
\sphinxAtStartPar
06/13 23:59
&
\sphinxAtStartPar
06/16 00:00
&
\sphinxAtStartPar
06/16 00:30
&
\sphinxAtStartPar
06/16 21:27
\\
\hline
\sphinxAtStartPar
3621
&
\sphinxAtStartPar
MC
&
\sphinxAtStartPar
65
&
\sphinxAtStartPar
N/A
&
\sphinxAtStartPar
180
&
\sphinxAtStartPar
06/13 23:59
&
\sphinxAtStartPar
06/16 00:00
&
\sphinxAtStartPar
06/16 00:30
&
\sphinxAtStartPar
06/16 21:29
\\
\hline
\sphinxAtStartPar
3632
&
\sphinxAtStartPar
MC
&
\sphinxAtStartPar
75
&
\sphinxAtStartPar
N/A
&
\sphinxAtStartPar
180
&
\sphinxAtStartPar
06/13 23:59
&
\sphinxAtStartPar
06/16 00:00
&
\sphinxAtStartPar
06/16 00:30
&
\sphinxAtStartPar
06/16 21:30
\\
\hline
\sphinxAtStartPar
4699
&
\sphinxAtStartPar
MC
&
\sphinxAtStartPar
85
&
\sphinxAtStartPar
10209
&
\sphinxAtStartPar
240
&
\sphinxAtStartPar
06/13 23:59
&
\sphinxAtStartPar
06/16 00:00
&
\sphinxAtStartPar
06/16 00:30
&
\sphinxAtStartPar
06/16 21:32
\\
\hline
\sphinxAtStartPar
3791
&
\sphinxAtStartPar
MC
&
\sphinxAtStartPar
95
&
\sphinxAtStartPar
N/A
&
\sphinxAtStartPar
180
&
\sphinxAtStartPar
06/16 00:00
&
\sphinxAtStartPar
06/16 02:06
&
\sphinxAtStartPar
06/16 02:30
&
\sphinxAtStartPar
06/16 21:34
\\
\hline
\sphinxAtStartPar
2796
&
\sphinxAtStartPar
MC
&
\sphinxAtStartPar
105
&
\sphinxAtStartPar
12348705
&
\sphinxAtStartPar
240
&
\sphinxAtStartPar
06/13 23:59
&
\sphinxAtStartPar
06/16 00:00
&
\sphinxAtStartPar
06/16 00:30
&
\sphinxAtStartPar
06/16 21:36
\\
\hline
\sphinxAtStartPar
25352
&
\sphinxAtStartPar
MC (New)
&
\sphinxAtStartPar
120
&
\sphinxAtStartPar
59843944
&
\sphinxAtStartPar
240
&
\sphinxAtStartPar
06/13 23:59
&
\sphinxAtStartPar
06/16 00:00
&
\sphinxAtStartPar
06/16 00:30
&
\sphinxAtStartPar
06/16 21:47
\\
\hline
\sphinxAtStartPar
7537
&
\sphinxAtStartPar
ADCP 300kHz
&
\sphinxAtStartPar
125
&
\sphinxAtStartPar
N/A
&
\sphinxAtStartPar
300
&
\sphinxAtStartPar
06/15 23:59
&
\sphinxAtStartPar
See Table
&
\sphinxAtStartPar
See Table
&
\sphinxAtStartPar
06/16 21:47
\\
\hline
\sphinxAtStartPar
25444
&
\sphinxAtStartPar
MC (New)
&
\sphinxAtStartPar
135
&
\sphinxAtStartPar
12512941
&
\sphinxAtStartPar
240
&
\sphinxAtStartPar
06/13 23:59
&
\sphinxAtStartPar
06/16 00:00
&
\sphinxAtStartPar
06/16 00:30
&
\sphinxAtStartPar
06/16 21:50
\\
\hline
\sphinxAtStartPar
13571
&
\sphinxAtStartPar
MC
&
\sphinxAtStartPar
155
&
\sphinxAtStartPar
10211
&
\sphinxAtStartPar
240
&
\sphinxAtStartPar
06/13 23:59
&
\sphinxAtStartPar
06/16 00:00
&
\sphinxAtStartPar
06/16 00:30
&
\sphinxAtStartPar
06/16 21:51
\\
\hline
\sphinxAtStartPar
11380
&
\sphinxAtStartPar
MC
&
\sphinxAtStartPar
39m off
&
\sphinxAtStartPar
2146835
&
\sphinxAtStartPar
300
&
\sphinxAtStartPar
06/13 23:59
&
\sphinxAtStartPar
06/16 00:00
&
\sphinxAtStartPar
06/16 00:30
&
\sphinxAtStartPar
06/17 00:02
\\
\hline
\sphinxAtStartPar
11381
&
\sphinxAtStartPar
MC
&
\sphinxAtStartPar
39m off
&
\sphinxAtStartPar
2146836
&
\sphinxAtStartPar
300
&
\sphinxAtStartPar
06/16 00:00
&
\sphinxAtStartPar
06/16 00:00
&
\sphinxAtStartPar
06/16 00:30
&
\sphinxAtStartPar
06/17 00:36
\\
\hline
\end{tabulary}
\par
\sphinxattableend\end{savenotes}

\end{landscape}


\begin{savenotes}\sphinxattablestart
\centering
\sphinxcapstartof{table}
\sphinxthecaptionisattop
\sphinxcaption{WHOTS\sphinxhyphen{}19 mooring C\sphinxhyphen{}T and ADCP Instruments recovery information. All times are in UTC (mm/dd/yy hh:mm:ss). Sea\sphinxhyphen{}Bird 37 Serial Number (SN).}\label{\detokenize{3_section:table-8}}
\sphinxaftertopcaption
\begin{tabulary}{\linewidth}[t]{|T|T|T|T|T|T|T|T|}
\hline
\sphinxstyletheadfamily 
\sphinxAtStartPar
\sphinxstylestrong{Depth (m)}
&\sphinxstyletheadfamily 
\sphinxAtStartPar
\sphinxstylestrong{SN}
&\sphinxstyletheadfamily 
\sphinxAtStartPar
\sphinxstylestrong{Time out of water}
&\sphinxstyletheadfamily 
\sphinxAtStartPar
\sphinxstylestrong{Time of Spike}
&\sphinxstyletheadfamily 
\sphinxAtStartPar
\sphinxstylestrong{Time of End Spike}
&\sphinxstyletheadfamily 
\sphinxAtStartPar
\sphinxstylestrong{Time Logging Stopped}
&\sphinxstyletheadfamily 
\sphinxAtStartPar
\sphinxstylestrong{Samples Logged}
&\sphinxstyletheadfamily 
\sphinxAtStartPar
\sphinxstylestrong{Data Quality}
\\
\hline
\sphinxAtStartPar
7
&
\sphinxAtStartPar
3617
&
\sphinxAtStartPar
8/29/21 3:17
&
\sphinxAtStartPar
8/29/21 6:48
&
\sphinxAtStartPar
8/29/21 7:48
&
\sphinxAtStartPar
1/31/21 10:42
&
\sphinxAtStartPar
233016
&
\sphinxAtStartPar
Good
\\
\hline
\sphinxAtStartPar
15
&
\sphinxAtStartPar
6893
&
\sphinxAtStartPar
8/29/21 3:18
&
\sphinxAtStartPar
8/29/21 6:48
&
\sphinxAtStartPar
8/29/21 7:48
&
\sphinxAtStartPar
5/8/21 12:58
&
\sphinxAtStartPar
838859
&
\sphinxAtStartPar
Good
\\
\hline
\sphinxAtStartPar
25
&
\sphinxAtStartPar
6894
&
\sphinxAtStartPar
8/29/21 3:22
&
\sphinxAtStartPar
8/29/21 6:48
&
\sphinxAtStartPar
8/29/21 7:48
&
\sphinxAtStartPar
5/8/21 12:58
&
\sphinxAtStartPar
838859
&
\sphinxAtStartPar
Good
\\
\hline
\sphinxAtStartPar
35
&
\sphinxAtStartPar
6895
&
\sphinxAtStartPar
8/29/21 3:31
&
\sphinxAtStartPar
8/29/21 6:48
&
\sphinxAtStartPar
8/29/21 7:48
&
\sphinxAtStartPar
5/8/21 12:58
&
\sphinxAtStartPar
838859
&
\sphinxAtStartPar
Good
\\
\hline
\sphinxAtStartPar
40
&
\sphinxAtStartPar
6896
&
\sphinxAtStartPar
8/29/21 3:31
&
\sphinxAtStartPar
8/29/21 6:48
&
\sphinxAtStartPar
8/29/21 7:48
&
\sphinxAtStartPar
5/8/21 12:58
&
\sphinxAtStartPar
838859
&
\sphinxAtStartPar
Good
\\
\hline
\sphinxAtStartPar
45
&
\sphinxAtStartPar
6887
&
\sphinxAtStartPar
8/29/21 3:32
&
\sphinxAtStartPar
8/29/21 6:48
&
\sphinxAtStartPar
8/29/21 7:48
&
\sphinxAtStartPar
1/31/21 10:47
&
\sphinxAtStartPar
559239
&
\sphinxAtStartPar
Good
\\
\hline
\sphinxAtStartPar
47.5
&
\sphinxAtStartPar
ADCP 1825
&
\sphinxAtStartPar
8/29/21 2:12
&
\sphinxAtStartPar
N/A
&
\sphinxAtStartPar
See \hyperref[\detokenize{3_section:table-10}]{Table \ref{\detokenize{3_section:table-10}}}
&
\sphinxAtStartPar
1/21/20
&
\sphinxAtStartPar
15823
&
\sphinxAtStartPar
Failed on 01/21/20
\\
\hline
\sphinxAtStartPar
50
&
\sphinxAtStartPar
6897
&
\sphinxAtStartPar
8/29/21 2:12
&
\sphinxAtStartPar
8/29/21 6:48
&
\sphinxAtStartPar
8/29/21 7:48
&
\sphinxAtStartPar
5/8/21 12:58
&
\sphinxAtStartPar
838859
&
\sphinxAtStartPar
Good
\\
\hline
\sphinxAtStartPar
55
&
\sphinxAtStartPar
6898
&
\sphinxAtStartPar
8/29/21 1:55
&
\sphinxAtStartPar
8/29/21 6:48
&
\sphinxAtStartPar
8/29/21 7:48
&
\sphinxAtStartPar
5/8/21 12:58
&
\sphinxAtStartPar
838859
&
\sphinxAtStartPar
Good
\\
\hline
\sphinxAtStartPar
65
&
\sphinxAtStartPar
6899
&
\sphinxAtStartPar
8/29/21 1:53
&
\sphinxAtStartPar
8/29/21 6:48
&
\sphinxAtStartPar
8/29/21 7:48
&
\sphinxAtStartPar
5/8/21 12:58
&
\sphinxAtStartPar
838859
&
\sphinxAtStartPar
Good
\\
\hline
\sphinxAtStartPar
75
&
\sphinxAtStartPar
3618
&
\sphinxAtStartPar
8/29/21 1:52
&
\sphinxAtStartPar
8/29/21 6:48
&
\sphinxAtStartPar
8/29/21 7:48
&
\sphinxAtStartPar
1/31/21 10:45
&
\sphinxAtStartPar
233016
&
\sphinxAtStartPar
Good
\\
\hline
\sphinxAtStartPar
85
&
\sphinxAtStartPar
3634
&
\sphinxAtStartPar
8/29/21 1:51
&
\sphinxAtStartPar
8/29/21 6:48
&
\sphinxAtStartPar
8/29/21 7:48
&
\sphinxAtStartPar
1/31/21 10:42
&
\sphinxAtStartPar
233016
&
\sphinxAtStartPar
Good
\\
\hline
\sphinxAtStartPar
95
&
\sphinxAtStartPar
3670
&
\sphinxAtStartPar
8/29/21 1:50
&
\sphinxAtStartPar
8/29/21 6:48
&
\sphinxAtStartPar
8/29/21 7:48
&
\sphinxAtStartPar
3/16/21 13:52
&
\sphinxAtStartPar
190650
&
\sphinxAtStartPar
Good
\\
\hline
\sphinxAtStartPar
105
&
\sphinxAtStartPar
6889
&
\sphinxAtStartPar
8/29/21 1:49
&
\sphinxAtStartPar
8/29/21 6:48
&
\sphinxAtStartPar
8/29/21 7:48
&
\sphinxAtStartPar
1/31/21 10:47
&
\sphinxAtStartPar
559239
&
\sphinxAtStartPar
Good
\\
\hline
\sphinxAtStartPar
120
&
\sphinxAtStartPar
6890
&
\sphinxAtStartPar
8/29/21 1:48
&
\sphinxAtStartPar
8/29/21 6:48
&
\sphinxAtStartPar
8/29/21 7:48
&
\sphinxAtStartPar
1/31/21 10:47
&
\sphinxAtStartPar
559239
&
\sphinxAtStartPar
Good
\\
\hline
\sphinxAtStartPar
125
&
\sphinxAtStartPar
ADCP 4891
&
\sphinxAtStartPar
8/29/21 1:44
&
\sphinxAtStartPar
N/A
&
\sphinxAtStartPar
See \hyperref[\detokenize{3_section:table-10}]{Table \ref{\detokenize{3_section:table-10}}}
&
\sphinxAtStartPar
7/6/21
&
\sphinxAtStartPar
92393
&
\sphinxAtStartPar
Good
\\
\hline
\sphinxAtStartPar
135
&
\sphinxAtStartPar
6888
&
\sphinxAtStartPar
8/29/21 1:43
&
\sphinxAtStartPar
8/29/21 6:48
&
\sphinxAtStartPar
8/29/21 7:48
&
\sphinxAtStartPar
1/31/21 10:47
&
\sphinxAtStartPar
559239
&
\sphinxAtStartPar
Good
\\
\hline
\sphinxAtStartPar
155
&
\sphinxAtStartPar
6891
&
\sphinxAtStartPar
8/29/21 1:41
&
\sphinxAtStartPar
8/29/21 6:48
&
\sphinxAtStartPar
8/29/21 7:48
&
\sphinxAtStartPar
1/31/21 10:47
&
\sphinxAtStartPar
559239
&
\sphinxAtStartPar
Good
\\
\hline
\sphinxAtStartPar
1875
&
\sphinxAtStartPar
3639
&
\sphinxAtStartPar
8/29/21 0:52
&
\sphinxAtStartPar
8/29/21 6:48
&
\sphinxAtStartPar
8/29/21 7:48
&
\sphinxAtStartPar
7/21/21 0:19
&
\sphinxAtStartPar
190650
&
\sphinxAtStartPar
Good
\\
\hline
\sphinxAtStartPar
1875
&
\sphinxAtStartPar
12242
&
\sphinxAtStartPar
8/29/21 0:52
&
\sphinxAtStartPar
8/29/21 6:48
&
\sphinxAtStartPar
8/29/21 7:48
&
\sphinxAtStartPar
8/31/21 19:40
&
\sphinxAtStartPar
202689
&
\sphinxAtStartPar
Good
\\
\hline
\sphinxAtStartPar
38 mab
&
\sphinxAtStartPar
11391
&
\sphinxAtStartPar
8/28/21 21:45
&
\sphinxAtStartPar
8/29/21 6:48
&
\sphinxAtStartPar
8/29/21 7:48
&
\sphinxAtStartPar
8/31/21 17:00
&
\sphinxAtStartPar
202657
&
\sphinxAtStartPar
Good
\\
\hline
\sphinxAtStartPar
38 mab
&
\sphinxAtStartPar
12241
&
\sphinxAtStartPar
8/28/21 21:45
&
\sphinxAtStartPar
8/29/21 6:48
&
\sphinxAtStartPar
8/29/21 7:48
&
\sphinxAtStartPar
8/31/21 16:00
&
\sphinxAtStartPar
202645
&
\sphinxAtStartPar
Good
\\
\hline
\end{tabulary}
\par
\sphinxattableend\end{savenotes}


\begin{savenotes}\sphinxattablestart
\centering
\sphinxcapstartof{table}
\sphinxthecaptionisattop
\sphinxcaption{WHOTS\sphinxhyphen{}19 mooring ADCP deployment and configuration information. All times are in UTC (mm/dd/yy hh:mm:ss).}\label{\detokenize{3_section:table-9}}
\sphinxaftertopcaption
\begin{tabulary}{\linewidth}[t]{|T|T|T|}
\hline
\sphinxstyletheadfamily 
\sphinxAtStartPar
\sphinxstylestrong{\sphinxhyphen{}}
&\sphinxstyletheadfamily 
\sphinxAtStartPar
\sphinxstylestrong{ADCP S/N 1825}
&\sphinxstyletheadfamily 
\sphinxAtStartPar
\sphinxstylestrong{ADCP S/N 4891}
\\
\hline
\sphinxAtStartPar
\sphinxstylestrong{Frequency (kHz)}
&
\sphinxAtStartPar
600
&
\sphinxAtStartPar
300
\\
\hline
\sphinxAtStartPar
\sphinxstylestrong{Number of Depth Cells}
&
\sphinxAtStartPar
25
&
\sphinxAtStartPar
30
\\
\hline
\sphinxAtStartPar
\sphinxstylestrong{Depth Cell Size (m)}
&
\sphinxAtStartPar
2 m
&
\sphinxAtStartPar
4 m
\\
\hline
\sphinxAtStartPar
\sphinxstylestrong{Pings per Ensemble}
&
\sphinxAtStartPar
80
&
\sphinxAtStartPar
40
\\
\hline
\sphinxAtStartPar
\sphinxstylestrong{Time per Ensemble (min)}
&
\sphinxAtStartPar
10 min
&
\sphinxAtStartPar
10 min
\\
\hline
\sphinxAtStartPar
\sphinxstylestrong{Time per Ping (sec)}
&
\sphinxAtStartPar
2 sec
&
\sphinxAtStartPar
4 sec
\\
\hline
\sphinxAtStartPar
\sphinxstylestrong{Time of First Ping}
&
\sphinxAtStartPar
10/04/19, 00:00:00
&
\sphinxAtStartPar
10/04/19, 00:00:00
\\
\hline
\sphinxAtStartPar
\sphinxstylestrong{Transducer 1 Spike Time}
&
\sphinxAtStartPar
10/05/19, 03:30:00
&
\sphinxAtStartPar
10/05/19, 03:31:00
\\
\hline
\sphinxAtStartPar
\sphinxstylestrong{Transducer 2 Spike Time}
&
\sphinxAtStartPar
10/05/19, 03:30:15
&
\sphinxAtStartPar
10/05/19, 03:31:15
\\
\hline
\sphinxAtStartPar
\sphinxstylestrong{Transducer 3 Spike Time}
&
\sphinxAtStartPar
10/05/19, 03:30:30
&
\sphinxAtStartPar
10/05/19, 03:31:30
\\
\hline
\sphinxAtStartPar
\sphinxstylestrong{Transducer 4 Spike Time}
&
\sphinxAtStartPar
10/05/19, 03:30:45
&
\sphinxAtStartPar
10/05/19, 03:31:45
\\
\hline
\sphinxAtStartPar
\sphinxstylestrong{Time in Water}
&
\sphinxAtStartPar
10/05/19, 19:43:00
&
\sphinxAtStartPar
10/05/19, 20:04:00
\\
\hline
\sphinxAtStartPar
\sphinxstylestrong{Depth (m)}
&
\sphinxAtStartPar
47.5 m
&
\sphinxAtStartPar
125 m
\\
\hline
\end{tabulary}
\par
\sphinxattableend\end{savenotes}


\begin{savenotes}\sphinxattablestart
\centering
\sphinxcapstartof{table}
\sphinxthecaptionisattop
\sphinxcaption{WHOTS\sphinxhyphen{}19 mooring ADCP recovery information. All times are in UTC (mm/dd/yy hh:mm:ss).}\label{\detokenize{3_section:table-10}}
\sphinxaftertopcaption
\begin{tabulary}{\linewidth}[t]{|T|T|T|}
\hline

\sphinxAtStartPar

&\sphinxstyletheadfamily 
\sphinxAtStartPar
\sphinxstylestrong{ADCP S/N 1825}
&\sphinxstyletheadfamily 
\sphinxAtStartPar
\sphinxstylestrong{ADCP S/N 4891}
\\
\hline
\sphinxAtStartPar
\sphinxstylestrong{Transducer 1 Spike Time}
&
\sphinxAtStartPar
N/A
&
\sphinxAtStartPar
N/A
\\
\hline
\sphinxAtStartPar
\sphinxstylestrong{Transducer 2 Spike Time}
&
\sphinxAtStartPar
N/A
&
\sphinxAtStartPar
N/A
\\
\hline
\sphinxAtStartPar
\sphinxstylestrong{Transducer 3 Spike Time}
&
\sphinxAtStartPar
N/A
&
\sphinxAtStartPar
N/A
\\
\hline
\sphinxAtStartPar
\sphinxstylestrong{Transducer 4 Spike Time}
&
\sphinxAtStartPar
N/A
&
\sphinxAtStartPar
N/A
\\
\hline
\sphinxAtStartPar
\sphinxstylestrong{Time out of Water}
&
\sphinxAtStartPar
08/29/21, 02:12:00
&
\sphinxAtStartPar
08/29/21 01:44:00
\\
\hline
\end{tabulary}
\par
\sphinxattableend\end{savenotes}

\sphinxAtStartPar
The RDI 300 kHz Workhorse Sentinel ADCP, SN 4891, was deployed at 125 m with
transducers facing upwards with an additional external battery pack. This
instrument was set to ping at 4\sphinxhyphen{}second intervals for 160 seconds every 10
minutes, and the burst sampling was designed to minimize aliasing by occasional
large ocean swell orbital motions. The bin size was set for 4 m. The total
number of ensemble records was 92,393. The first ensemble was on 10/04/2019 at
00:00:00Z, and the last was on 07/06/2021 at 14:39.59Z (see \hyperref[\detokenize{3_section:table-9}]{Table \ref{\detokenize{3_section:table-9}}},
\hyperref[\detokenize{3_section:table-10}]{Table \ref{\detokenize{3_section:table-10}}}, and {\hyperref[\detokenize{appendices:whots-19-300-khz-serial-4891}]{\sphinxcrossref{\DUrole{std,std-ref}{WHOTS\sphinxhyphen{}19 300 kHz \sphinxhyphen{} Serial 4891}}}} for
more configuration). This instrument also measured temperature.

\sphinxAtStartPar
The RDI 600 kHz Workhorse Sentinel ADCP, SN 1825, was deployed at 47.5 m with
transducers facing upwards with an additional external battery pack. The
instrument was set to ping at 2\sphinxhyphen{}second intervals for 160 seconds every 10
minutes, and the burst sampling was designed to minimize aliasing by occasional
large ocean swell orbital motions. The bin size was set for 2 m. The total
number of ensemble records was 15,823. The first ensemble was on 10/04/2019 at
00:00:00Z, and the last was on 01/21/2020 at 20:49:59Z (see \hyperref[\detokenize{3_section:table-9}]{Table \ref{\detokenize{3_section:table-9}}}
, \hyperref[\detokenize{3_section:table-10}]{Table \ref{\detokenize{3_section:table-10}}}, and {\hyperref[\detokenize{appendices:whots-19-600-khz-serial-1825}]{\sphinxcrossref{\DUrole{std,std-ref}{WHOTS\sphinxhyphen{}19 600 kHz \sphinxhyphen{} Serial 1825}}}}
for more configuration). This instrument also measured temperature.

\sphinxAtStartPar
The two VMCMs, SN 2042 and 2032, were deployed at 10 m and 30 m depth,
respectively. The instruments were prepared for deployment by the WHOI/UOP
group and set to sample at 1\sphinxhyphen{}minute interval. These instruments also
measured temperature.

\sphinxAtStartPar
All WHOTS\sphinxhyphen{}19 instruments were successfully recovered; recovery information for
the C\sphinxhyphen{}T instruments is shown in \hyperref[\detokenize{3_section:table-8}]{Table \ref{\detokenize{3_section:table-8}}}. Most of the instruments had
some degree of biofouling, with the most substantial fouling near the surface.
The fouling extended down to the ADCP at 125 m, although it was minor at that
level.

\sphinxAtStartPar
All MicroCATs were in good condition after recovery, although with more
biofouling than usual given that they were in the water for nearly 2 years.
MicroCAT SN 6899 (65 m) had a barnacle partially blocking the top of its
conductivity cell.

\sphinxAtStartPar
After recovery and before stopping recording, a bag of ice was placed in
contact with each MicroCAT temperature sensor, to produce a spike in the data
as a reference point to check the instrument’s clock. However, it was later
found that all the instruments had stopped logging data due to battery
drainage, except for the deep instruments. The data from all instruments were
downloaded on board the ship, the instruments returned data records ending as
early as January 31, 2021. \hyperref[\detokenize{3_section:table-8}]{Table \ref{\detokenize{3_section:table-8}}} has an initial evaluation of the
data quality; more details are in
{\hyperref[\detokenize{5_section:microcat-data-processing-procedures}]{\sphinxcrossref{\DUrole{std,std-ref}{MicroCAT Data Processing Procedures}}}}, and
{\hyperref[\detokenize{6_section:microcat-data}]{\sphinxcrossref{\DUrole{std,std-ref}{MicroCAT Data}}}}.

\sphinxAtStartPar
The data from the upward\sphinxhyphen{}looking 300 kHz ADCP at 125 m were good; the
instrument was not pinging upon recovery and the data record indicates that it
stopped recording on July 6, 2021. There appears to be no obviously
questionable data from this ADCP at this time, apart from near\sphinxhyphen{}surface
side\sphinxhyphen{}lobe interference. The 600 kHz instrument (47.5 m) was recovered with its battery
wire cut and separated from the instrument. The data record indicates that it
stopped recording on January 21, 2020.


\chapter{WHOTS (19\sphinxhyphen{}20) Cruise Shipboard Observations}
\label{\detokenize{4_section:whots-19-20-cruise-shipboard-observations}}\label{\detokenize{4_section::doc}}
\sphinxAtStartPar
The hydrographic profile observations made during the WHOTS cruises were
obtained with a Sea\sphinxhyphen{}Bird CTD package with dual temperature, salinity, and
oxygen sensors. This CTD was installed on a rosette\sphinxhyphen{}sampler with 5 L Niskin
sampling bottles for calibration water samples. Furthermore, the ship Oscar
Sette came equipped with a thermosalinograph system that provided a continuous
depiction of the near\sphinxhyphen{}surface layer’s temperature and salinity. Horizontal
currents over the depth range of 30\sphinxhyphen{}700 m were measured from the shipboard 75
kHz Ocean Surveyor (OS75) ADCP (narrowband) with a vertical resolution of 16m
for the WHOTS\sphinxhyphen{}19 and WHOTS\sphinxhyphen{}20 cruises. Broadband mode for the OS75 provided
additional current data over the range upper 200 m with a vertical resolution
of 8m.

\sphinxAtStartPar
Data gaps occurred when the system was shut down temporarily during
communications with the acoustic releases used for the moorings during both
cruises. Periods of missing data between 300 and 450 m in the broadband ADCP
were apparent due to the lack of scattering material in the water.


\section{Conductivity, Temperature, and Depth (CTD) Profiling}
\label{\detokenize{4_section:conductivity-temperature-and-depth-ctd-profiling}}
\sphinxAtStartPar
Continuous measurements of temperature, conductivity, dissolved oxygen, and
pressure were made with the UH Sea\sphinxhyphen{}Bird SBE\sphinxhyphen{}9/11Plus CTD underwater units
\#91361 and \#850 during WHOTS\sphinxhyphen{}19 and WHOTS\sphinxhyphen{}20 cruises respectively. The CTD was
equipped with an internal Digiquartz pressure sensor and pairs of external
temperature, conductivity, and oxygen sensors.

\sphinxAtStartPar
Each temperature\sphinxhyphen{}conductivity sensor pair used a Sea\sphinxhyphen{}Bird TC duct, which
circulated seawater through independent pump and plumbing installations. The
CTD configuration also included two oxygen sensors, installed in the plumbing
for each sensor set. In both cruises, the CTD was mounted in a vertical
position in the lower part of a rosette sampler, with the sensors’ water
intakes located at the bottom of the rosette.

\sphinxAtStartPar
The package was deployed on a conducting cable, which allowed for real\sphinxhyphen{}time
data acquisition and display. The deployment procedure consisted of lowering
the package to approximately 10 dbar and waiting until the CTD pumps started
operating. The CTD was then raised until the sensors were close to the surface
to begin the CTD cast. The time and position of each cast were obtained via a
GPS connection to the CTD deck box. Four salinity samples were taken on each
cast for calibration of the conductivity sensors.


\subsection{Data Acquisition and Processing}
\label{\detokenize{4_section:data-acquisition-and-processing}}
\sphinxAtStartPar
CTD data were acquired at the instrument’s highest sampling rate of 24 samples
per second. Digital data were stored on a laptop computer, and, for redundancy,
the analog signal was recorded on a separate computer using a sound card and
Audacity (TM) software. Backups of CTD data were made onto USB storage cards.

\sphinxAtStartPar
The raw CTD data were quality controlled and screened for spikes described in
the WHOTS Data Report 1 {[}\hyperlink{cite.references:id15}{Santiago\sphinxhyphen{}Mandujano \sphinxstyleemphasis{et al.}, 2007}{]}. Data alignment,
averaging, correction, and reporting were done as described in
{[}\hyperlink{cite.references:id20}{Tupas \sphinxstyleemphasis{et al.}, 1993}{]}. Spikes in the data occur when the CTD samples the disturbed
water of its wake. Therefore, the downcast samples were rejected when the CTD
was moving upward or when its acceleration exceeded 0.5 m s\sphinxhyphen{}2 in magnitude. The
data were subsequently averaged into 2\sphinxhyphen{}dbar pressure bins after calibrating the
CTD conductivity with the bottle salinities.

\sphinxAtStartPar
The data were additionally screened by comparing the T\sphinxhyphen{}C sensor pairs. These
differences permitted the identification of problems with the sensors. The data
from only one T\sphinxhyphen{}C pair, whichever was deemed most reliable, is reported here.
Only data from the downcast are reported, as wake effects from the rosette
commonly contaminate upcast data.

\sphinxAtStartPar
Temperature is reported on the ITS\sphinxhyphen{}90 scale. Salinity and all derived units
were calculated using the UNESCO (1981) routines; salinity is reported in the
Practical Salinity(SA) scale (PSS\sphinxhyphen{}78). Oxygen is reported in umol kg\sphinxhyphen{}1.


\subsection{CTD Sensor Calibration and Corrections}
\label{\detokenize{4_section:ctd-sensor-calibration-and-corrections}}

\subsubsection{Pressure}
\label{\detokenize{4_section:pressure}}
\sphinxAtStartPar
The pressure calibration strategy for CTD pressure transducers \#75434 and \#1430
used during WHOTS\sphinxhyphen{}19 and WHOTS\sphinxhyphen{}20 cruises respectively employed a high\sphinxhyphen{}quality
quartz pressure transducer as a transfer standard. Periodic recalibrations of
this lab standard were performed with a primary pressure standard. The only
corrections applied to the CTD pressures were a constant offset determined when
the CTD first enters the water on each cast. Also, a span correction determined
from bench tests on the sensor against the transfer standard was applied. These
procedures and corrections are thoroughly documented in the HOT\sphinxhyphen{}2019 data
report {[}{]}.


\subsubsection{Temperature/Conductivity}
\label{\detokenize{4_section:temperature-conductivity}}
\sphinxAtStartPar
Sea\sphinxhyphen{}Bird SBE\sphinxhyphen{}3\sphinxhyphen{}Plus temperature and SBE 4C conductivity transducers were used
during WHOTS\sphinxhyphen{}19 and \sphinxhyphen{}20 cruises. These sensors’ history and performance have
been monitored during HOT cruises, and calibrations and drift corrections
applied during WHOTS cruises are thoroughly documented in the HOT\sphinxhyphen{}2019 data
report {[}{]}.


\subsubsection{Dissolved Oxygen}
\label{\detokenize{4_section:dissolved-oxygen}}
\sphinxAtStartPar
Sea\sphinxhyphen{}Bird SBE\sphinxhyphen{}43 oxygen sensors were used during the WHOTS\sphinxhyphen{}19 and \sphinxhyphen{}20 cruises.
The WHOTS\sphinxhyphen{}19 oxygen data were calibrated using calibration coefficients
obtained during the HOT\sphinxhyphen{}315 cruise conducted on 3\sphinxhyphen{}7 September 2019, before the
WHOTS\sphinxhyphen{}19 cruise, which used the same oxygen sensors. The CTD empirical
calibration was performed using oxygen water samples and the procedure from
{[}\hyperlink{cite.references:id13}{Owens and Millard, 1985}{]}. See {[}\hyperlink{cite.references:id19}{Tupas \sphinxstyleemphasis{et al.}, 1996}{]} for details on these calibrations
procedures. The oxygen data from WHOTS\sphinxhyphen{}20 were calibrated using calibration
coefficients obtained during the HOT\sphinxhyphen{}327 cruise conducted on 15\sphinxhyphen{}19 February
2021, before the WHOTS\sphinxhyphen{}20 cruise, which used the same oxygen sensors.


\section{Water Sampling and Analysis}
\label{\detokenize{4_section:water-sampling-and-analysis}}

\subsection{Salinity}
\label{\detokenize{4_section:salinity}}
\sphinxAtStartPar
Salinity samples were collected by a rosette sampler during CTD casts at
selected depths during WHOTS\sphinxhyphen{}19 and \sphinxhyphen{}20, and then sub\sphinxhyphen{}sampled in 250 ml glass
bottles. The top of each bottle and thimble were thoroughly dried before being
tightly capped to prevent water from being trapped between the cap or thimble
and the bottle’s mouth. It has been observed that residual water trapped in
this way increases its salinity due to evaporation, and it can leak into the
sample when the bottle is opened for measuring. Samples from each cruise were
measured after the cruise in the UH laboratory using a Guildline Autosal 8400B
SN 70168 for WHOTS\sphinxhyphen{}19 and SN 73647 for WHOTS\sphinxhyphen{}20. International Association for
Physical Sciences of the Ocean (IAPSO) standard seawater samples were measured
to standardize the Autosal, and samples from a large batch of “secondary
standard” (substandard) seawater were measured after every 24\sphinxhyphen{}48 samples to
detect drift in the Autosal. Standard deviations of the secondary standard
measurements were less than \(\pm\) 0.001 for WHOTS\sphinxhyphen{}19 and \sphinxhyphen{}20 cruises
\hyperref[\detokenize{4_section:table-11}]{Table \ref{\detokenize{4_section:table-11}}}.

\sphinxAtStartPar
The substandard water was collected by a rosette sampler from 1020 m at station
ALOHA during HOT cruises and drained into a 50\sphinxhyphen{}liter Nalgene plastic carboy. In
the laboratory, the water was then thoroughly mixed in a glass carboy for 20
minutes by manually shaking, rolling, and tilting the carboy vigorously, after
which a 2\sphinxhyphen{}inch protective layer of white oil was added on top to deter
evaporation. The substandard water was allowed to stand for approximately three
days before it was used and was stored in the same temperature\sphinxhyphen{}controlled room
as the Autosal, protecting it from the light with black plastic bags to inhibit
biological growth. Substandard seawater batch \#67 was prepared on August 18,
2019, and it was used for WHOTS\sphinxhyphen{}19. The batch \#71 was prepared on August 27,
2021, and it was used for WHOTS\sphinxhyphen{}20.

\sphinxAtStartPar
Samples from the WHOTS\sphinxhyphen{}19 were measured on October 28, 2019 and samples from
WHOTS\sphinxhyphen{}20 were measured on September 13, 2021. \hyperref[\detokenize{4_section:table-11}]{Table \ref{\detokenize{4_section:table-11}}} shows the
precision measurements of the secondary sub\sphinxhyphen{}standards.


\begin{savenotes}\sphinxattablestart
\centering
\sphinxcapstartof{table}
\sphinxthecaptionisattop
\sphinxcaption{The precision of salinity measurements of secondary lab standards.}\label{\detokenize{4_section:table-11}}
\sphinxaftertopcaption
\begin{tabulary}{\linewidth}[t]{|T|T|T|T|T|}
\hline
\sphinxstyletheadfamily 
\sphinxAtStartPar
\sphinxstylestrong{Cruise}
&\sphinxstyletheadfamily 
\sphinxAtStartPar
\sphinxstylestrong{Mean Salinity +/\sphinxhyphen{} SD}
&\sphinxstyletheadfamily 
\sphinxAtStartPar
\sphinxstylestrong{\# Samples}
&\sphinxstyletheadfamily 
\sphinxAtStartPar
\sphinxstylestrong{Substandard Batch}
&\sphinxstyletheadfamily 
\sphinxAtStartPar
\sphinxstylestrong{IAPSO Batch}
\\
\hline
\sphinxAtStartPar
\sphinxstylestrong{WHOTS\sphinxhyphen{}19}
&
\sphinxAtStartPar
3x.xxxx \(\pm\) 0.000x
&
\sphinxAtStartPar
x
&
\sphinxAtStartPar
xx
&
\sphinxAtStartPar
P1xx
\\
\hline
\sphinxAtStartPar
\sphinxstylestrong{WHOTS\sphinxhyphen{}20}
&
\sphinxAtStartPar
3x.xxxx \(\pm\) 0.000x
&
\sphinxAtStartPar
xx
&
\sphinxAtStartPar
xx
&
\sphinxAtStartPar
P1xx
\\
\hline
\end{tabulary}
\par
\sphinxattableend\end{savenotes}


\section{Thermosalinograph Data Acquisition and Processing}
\label{\detokenize{4_section:thermosalinograph-data-acquisition-and-processing}}

\subsection{WHOTS\sphinxhyphen{}19 Cruise}
\label{\detokenize{4_section:whots-19-cruise}}
\sphinxAtStartPar
Near\sphinxhyphen{}surface temperature and salinity data during the WHOTS\sphinxhyphen{}19 cruise were
acquired from the thermosalinograph (TSG) system installed on the NOAA Ship
Oscar Sette. The sensors were sampling water from the continuous seawater
system running through the ship. They comprised one thermosalinograph
model SBE\sphinxhyphen{}21 (SN 3168) and a micro\sphinxhyphen{}thermosalinograph model SBE\sphinxhyphen{}45 (SN 0290),
both with (internal) temperature and conductivity sensors located in the ship’s
chemistry lab, about 70 m from the hull intake; and an SBE\sphinxhyphen{}38 (SN 266) external
temperature sensor located at the entrance of the water intake. All instruments
recorded data every second. The water intake is located at the ship’s bow,
forward from the starboard side bow thruster at a depth of 3 m. The system has
a flow meter in the chemistry lab, showing a flow rate of about 1.1
liters/minute during the cruise. Only the SBE\sphinxhyphen{}45 has a debubbler. Salinity
water samples were taken every 8 hours from the exhaust in the Chemistry lab
using 0.25\sphinxhyphen{}liter glass bottles, to be measured in the UH lab to correct any
drift in the thermosalinograph conductivities.


\subsubsection{Temperature Calibration}
\label{\detokenize{4_section:temperature-calibration}}
\sphinxAtStartPar
External temperature data from the SBE\sphinxhyphen{}38 sensor (last calibrated at Sea\sphinxhyphen{}Bird
on December 29, 2019) were used to measure the seawater temperature. These
data were compared to the data collected during CTD casts.


\subsubsection{Nominal Conductivity Calibration}
\label{\detokenize{4_section:nominal-conductivity-calibration}}
\sphinxAtStartPar
Data from the SBE\sphinxhyphen{}45 conductivity and temperature sensors were used to
calculate the intake seawater salinity. These sensors were last calibrated at
Sea\sphinxhyphen{}Bird on February 19th, 2019. All conductivity data from the
thermosalinograph were nominally calibrated with coefficients from this
calibration. However, all the final salinity data reported here were calibrated
against bottle data, as explained below.


\subsubsection{Data Processing}
\label{\detokenize{4_section:data-processing}}
\sphinxAtStartPar
Daily files containing navigation data recorded every second were concatenated
with the thermosalinograph data. The thermosalinograph data were then screened
for gross errors, with upper and lower bounds of 18°C and 35°C for
temperature and 3 and 6 Siemens/m for conductivity. There were 92 points
outside the valid temperature range and no points outside the valid
conductivity range.

\sphinxAtStartPar
A 5\sphinxhyphen{}point running median filter was used to detect one\sphinxhyphen{} or two\sphinxhyphen{}point
temperature and conductivity glitches in the thermosalinograph data. Glitches
in temperature and conductivity detected by the 5\sphinxhyphen{}point median filter were
immediately replaced by the median. Threshold values of 0.3°C for temperature
and 0.1 Siemens/m for conductivity were used for the median filter. After
running the filter, there were 13 internal temperature, 100 external
temperature, and 913 conductivity points replaced with the median.

\sphinxAtStartPar
A 3\sphinxhyphen{}point triangular running mean filter was used to smooth the temperature and
conductivity data after passing the glitch detection.

\sphinxAtStartPar
The thermosalinograph aboard the Ship Oscar Sette was set to record data every
second. The system had to be secured on the last day of the cruise due to the
bad weather because it kept shutting down due to air going into the plumbing,
causing the pumps to stop working.

\sphinxAtStartPar
Data were visually scanned to flag spikes likely caused by contamination due to
the introduction of bubbles to the flow\sphinxhyphen{}through system during transits or rough
conditions. Of 69,311,136 data points, 141,365 conductivity data points were
flagged as bad.


\subsubsection{Bottle salinity and CTD Salinity Comparisons}
\label{\detokenize{4_section:bottle-salinity-and-ctd-salinity-comparisons}}
\sphinxAtStartPar
The thermosalinograph salinity was calibrated by comparing it to bottle
salinity samples drawn from a water intake next to the thermosalinograph every
8 hours throughout the cruise. Of the twenty\sphinxhyphen{}one thermosalinograph bottles
sampled, bottle \#18, \#19, \#20, and \#21 were identified as a conductivity
outlier and were discarded from the analysis. Samples were analyzed as
described in {\hyperref[\detokenize{4_section:water-sampling-and-analysis}]{\sphinxcrossref{\DUrole{std,std-ref}{Water Sampling and Analysis}}}}. The comparison
was made in conductivity to eliminate the effects of temperature. The
conductivity of each bottle sample was computed using the salinity of the
bottle, thermosalinograph temperature, and a pressure of 10 dbar, which
includes the pressure of the flow\sphinxhyphen{}through system’s pump.

\sphinxAtStartPar
Salinity samples were drawn from the flow\sphinxhyphen{}through system, located less than 0.5
m from the SBE\sphinxhyphen{}45. Consequently, there should be virtually no delay between
when the water passes through the thermosalinograph and sampled. A 90\sphinxhyphen{}second
average centered on the sample draw time was chosen for processing purposes.

\sphinxAtStartPar
In order to make the comparison in conductivity units, the CTD conductivity was
calculated using the 4 dbar downcast CTD salinity, the internal
thermosalinograph temperature, and a pump pressure of 10 dbar. There were
eleven CTD casts conducted during WHOTS\sphinxhyphen{}19 while the thermosalinograph was
running. Casts 4, 5, 6, 7, and 9 were removed from the analysis as temperature
and conductivity outliers.

\sphinxAtStartPar
A cubic spline was fit to the time series of the differences between the bottle
and TSG conductivity, and a correction was obtained for the TSG conductivities.
Salinity was calculated using these corrected conductivities, the
thermosalinograph temperatures, and 10 dbar pressure. After applying
corrections, the mean difference between the bottle and thermosalinograph
salinities was \sphinxhyphen{}0.00004 psu with a standard deviation of 0.001886 psu. The mean
CTD \sphinxhyphen{} thermosalinograph difference was \sphinxhyphen{}0.0028 psu with a standard deviation of
0.001118 psu.


\subsubsection{CTD Temperature Comparisons}
\label{\detokenize{4_section:ctd-temperature-comparisons}}
\sphinxAtStartPar
There were 11 CTD casts conducted during WHOTS\sphinxhyphen{}19, one of which was a test cast
offshore Honolulu (Station 20) and five at Station 50 (WHOTS\sphinxhyphen{}19), and five at
Station 52 (WHOTS\sphinxhyphen{}18), respectively.
The 4 dbar downcast CTD temperature data from those casts were used to compare
with the thermosalinograph data at the time of the casts. This comparison gives
an estimate of the quality of the thermosalinograph measurements. Of the 11
casts, five were identified as temperature outliers after comparing it against
the thermosalinograph data and removed from the analysis. The mean difference
between the CTD and the internal temperature sensor was 0.058°C, with a
standard deviation of \(\pm\) 0.056°C.


\subsection{WHOTS\sphinxhyphen{}20 Cruise}
\label{\detokenize{4_section:whots-20-cruise}}
\sphinxAtStartPar
Near\sphinxhyphen{}surface temperature and salinity data during the WHOTS\sphinxhyphen{}20 cruise were
acquired from the thermosalinograph (TSG) system installed on the NOAA Ship
Oscar Sette. The sensors were sampling water from the continuous seawater
system running through the ship, and comprised one thermosalinograph model
SBE\sphinxhyphen{}21 (SN 3168) and a micro\sphinxhyphen{}thermosalinograph model SBE\sphinxhyphen{}45 (SN 0290), both
with (internal) temperature and conductivity sensors located in the ship’s
chemistry lab, about 70 m from the hull intake; and an SBE\sphinxhyphen{}38 (SN 266)
external temperature sensor located at the entrance of the water intake. All
instruments recorded data every second. The water intake is located at the bow
of the ship, forward from the starboard side bow thruster at a depth of 3 m.
The system has a flow meter in the chemistry lab, showing a flow rate of about
1.1 liter/minute during the cruise. Only the SBE\sphinxhyphen{}45 has a debubbler. Salinity
water samples were taken every 8 hours from the exhaust in the Chemistry lab
using 0.25 litter glass bottles, to be measured in the UH lab to correct for
any drift in the thermosalinograph conductivities.


\subsubsection{Temperature Calibration}
\label{\detokenize{4_section:id7}}
\sphinxAtStartPar
External temperature data from the SBE\sphinxhyphen{}38 sensor (last calibrated at Sea\sphinxhyphen{}Bird
on November 26, 2020) were used to measure the seawater temperature. These
data were compared to the data collected during CTD casts.


\subsubsection{Nominal Conductivity Calibration}
\label{\detokenize{4_section:id8}}
\sphinxAtStartPar
Data from the SBE\sphinxhyphen{}45 conductivity and temperature sensors were used to
calculate the intake seawater salinity. These sensors were last calibrated at
Sea\sphinxhyphen{}Bird on November 17, 2020. All conductivity data from the thermosalinograph
were nominally calibrated with coefficients from this calibration. However, all
the final salinity data reported here were calibrated against bottle data, as
explained below.


\subsubsection{Data Processing}
\label{\detokenize{4_section:id9}}
\sphinxAtStartPar
Daily files containing navigation data recorded every second were concatenated
with the thermosalinograph data. The thermosalinograph data were then screened
for gross errors, with upper and lower bounds of 18°C and 35°C for
temperature and 3 and 6 Siemens m \sphinxhyphen{}1 for conductivity. There were 488 points
outside the valid temperature range and no points outside the valid
conductivity range.

\sphinxAtStartPar
A 5\sphinxhyphen{}point running median filter was used to detect one\sphinxhyphen{} or two\sphinxhyphen{}point
temperature and conductivity glitches in the thermosalinograph data. Glitches
in temperature and conductivity detected by the 5\sphinxhyphen{}point median filter were
immediately replaced by the median. Threshold values of 0.3°C for temperature
and 0.1 Siemens m\sphinxhyphen{}1 for conductivity were used for the median filter. After
running the filter, there were 283 internal temperature, 1998 external
temperature, and 341 conductivity points replaced by the median. A 3\sphinxhyphen{}point
triangular running mean filter was used to smooth the temperature and
conductivity data after passing the glitch detection.

\sphinxAtStartPar
The thermosalinograph aboard the Ship Oscar Sette was set to record data every
second. Both thermosalinographs exhibited a number of conductivity and
temperature glitches due to air going into the plumbing. In addition, the
system had a drainage problem according to the ship’s technician. The data
between August 26 at 13:30 and 27 at 07:00 are particularly bad because it was
during transit back to Oahu to disembark a crew member with medical problems,
and the flow through the system was stopped during that time.

\sphinxAtStartPar
Data were visually scanned to flag spikes likely caused by contamination due to
the introduction of bubbles to the flow\sphinxhyphen{}through system during transits or rough
conditions. Of 649,826 data points, 133,851 conductivity data points were
flagged as bad.


\subsubsection{Bottle salinity and CTD Salinity Comparisons}
\label{\detokenize{4_section:id10}}
\sphinxAtStartPar
The thermosalinograph salinity was calibrated by comparing it to bottle
salinity samples drawn from a water intake next to the thermosalinograph every
8 hours throughout the cruise. Of the sixteen thermosalinograph bottles
sampled, bottle \#1 was identified as a conductivity outlier and were discarded
from the analysis. Samples were analyzed as described in
{\hyperref[\detokenize{4_section:water-sampling-and-analysis}]{\sphinxcrossref{\DUrole{std,std-ref}{Water Sampling and Analysis}}}}. The comparison was made in
conductivity to eliminate the effects of temperature. The conductivity of each
bottle sample was computed using the salinity of the bottle, thermosalinograph
temperature, and a pressure of 10 dbar, which includes the pressure of the
flow\sphinxhyphen{}through system’s pump.

\sphinxAtStartPar
Salinity samples were drawn from the flow\sphinxhyphen{}through system, located less than 0.5
m from the SBE\sphinxhyphen{}45. Consequently, there should be virtually no delay between
when the water passes through the thermosalinograph and sampled. A 90\sphinxhyphen{}second
average centered on the sample draw time was chosen for processing purposes.

\sphinxAtStartPar
In order to make the comparison in conductivity units, the CTD conductivity was
calculated using the 4 dbar downcast CTD salinity, the internal
thermosalinograph temperature, and a pump pressure of 10 dbar. There were ten
CTD casts conducted during WHOTS\sphinxhyphen{}20 while the thermosalinograph was running.
Casts 1, and 10 were removed from the analysis as temperature and conductivity
outliers.

\sphinxAtStartPar
A cubic spline was fit to the time series of the differences between the bottle
and TSG conductivity, and a correction was obtained for the TSG conductivities.
Salinity was calculated using these corrected conductivities, the
thermosalinograph temperatures, and ten dbar pressure. After applying
corrections, the mean difference between the bottle and thermosalinograph
salinities was less than 1 mpsu with a standard deviation of 0.000662 psu. The
mean CTD \sphinxhyphen{} thermosalinograph difference was \sphinxhyphen{}0.00018 psu with a standard
deviation of 0.00124 psu.


\subsubsection{CTD Temperature Comparisons}
\label{\detokenize{4_section:id11}}
\sphinxAtStartPar
There were ten CTD casts conducted during WHOTS\sphinxhyphen{}20, one of which was a test
cast offshore Honolulu (Station 20), one at Aloha Station (Station 2), five at
Station 52 (WHOTS\sphinxhyphen{}20), and
two at Station 50 (WHOTS\sphinxhyphen{}19). The 4 dbar downcast
CTD temperature data from those casts were used to compare with the
thermosalinograph data at the time of the casts. This comparison gives an
estimate of the quality of the thermosalinograph measurements. Of the ten
casts, two were identified as temperature outliers after comparing it
against the thermosalinograph data and removed from the analysis. The mean
difference between the CTD and the internal temperature sensor was
\sphinxhyphen{}0.247°C, with a standard deviation of \(\pm\) 0.067°C.


\section{Shipboard ADCP}
\label{\detokenize{4_section:shipboard-adcp}}

\subsection{WHOTS\sphinxhyphen{}19 Deployment Cruise}
\label{\detokenize{4_section:whots-19-deployment-cruise}}
\sphinxAtStartPar
Currents were measured for the cruise duration over the depth range of 30\sphinxhyphen{}700 m
with a 75 kHz RDI Ocean Surveyor (OS75) ADCP working in narrowband mode with a
vertical resolution of 16 m and broadband mode with a vertical resolution of 8
m. The system yielded good data {[}{]}
during operations near the WHOTS\sphinxhyphen{}18 and WHOTS\sphinxhyphen{}19 moorings. The broadband system
only recorded good data in the upper 200 m. The times of the datasets from the
OS75 kHz are shown in \hyperref[\detokenize{4_section:table-12}]{Table \ref{\detokenize{4_section:table-12}}}.


\begin{savenotes}\sphinxattablestart
\centering
\sphinxcapstartof{table}
\sphinxthecaptionisattop
\sphinxcaption{ADCP record times (UTC mm/dd/yy hh:mm:ss) for the Narrow Band 75 kHz ADCP during the WHOTS\sphinxhyphen{}19 cruise}\label{\detokenize{4_section:table-12}}
\sphinxaftertopcaption
\begin{tabulary}{\linewidth}[t]{|T|T|T|}
\hline
\sphinxstyletheadfamily 
\sphinxAtStartPar
\sphinxstylestrong{WHOTS\sphinxhyphen{}19}
&\sphinxstyletheadfamily 
\sphinxAtStartPar
\sphinxstylestrong{OS75nb}
&\sphinxstyletheadfamily 
\sphinxAtStartPar
\sphinxstylestrong{OS75bb}
\\
\hline
\sphinxAtStartPar
File starting time
&
\sphinxAtStartPar
10/04/19 19:38:54
&
\sphinxAtStartPar
10/04/19 19:38:54
\\
\hline
\sphinxAtStartPar
File ending time
&
\sphinxAtStartPar
10/12/19 20:14:30
&
\sphinxAtStartPar
10/12/19 20:14:30
\\
\hline
\end{tabulary}
\par
\sphinxattableend\end{savenotes}


\subsection{WHOTS\sphinxhyphen{}20 Deployment Cruise}
\label{\detokenize{4_section:whots-20-deployment-cruise}}
\sphinxAtStartPar
Currents were measured for the duration of the cruise over the depth range of
30\sphinxhyphen{}700 m with a 75 kHz RDI Ocean Surveyor (OS75) ADCP working in narrowband
mode with a vertical resolution of 16 m, and in broadband mode with vertical
resolution of 8 m. The system yielded good data (see
{[}\hyperlink{cite.references:id16}{Santiago\sphinxhyphen{}Mandujano \sphinxstyleemphasis{et al.}, 2022}{]}) during operations near the WHOTS\sphinxhyphen{}19 and
WHOTS\sphinxhyphen{}20 moorings. The broadband system only recorded good data in the upper
200 m. The times of the datasets from the OS75 kHz are shown in
\hyperref[\detokenize{4_section:table-13}]{Table \ref{\detokenize{4_section:table-13}}}.


\begin{savenotes}\sphinxattablestart
\centering
\sphinxcapstartof{table}
\sphinxthecaptionisattop
\sphinxcaption{ADCP record times (UTC mm/dd/yy hh:mm:ss) for the Narrow Band 75 kHz ADCP during the WHOTS\sphinxhyphen{}20 cruise}\label{\detokenize{4_section:table-13}}
\sphinxaftertopcaption
\begin{tabulary}{\linewidth}[t]{|T|T|T|}
\hline
\sphinxstyletheadfamily 
\sphinxAtStartPar
\sphinxstylestrong{WHOTS\sphinxhyphen{}20}
&\sphinxstyletheadfamily 
\sphinxAtStartPar
\sphinxstylestrong{OS75nb}
&\sphinxstyletheadfamily 
\sphinxAtStartPar
\sphinxstylestrong{OS75bb}
\\
\hline
\sphinxAtStartPar
File starting time
&
\sphinxAtStartPar
08/20/21 01:22:22
&
\sphinxAtStartPar
08/20/21 01:22:22
\\
\hline
\sphinxAtStartPar
File ending time
&
\sphinxAtStartPar
09/01/21 19:44:41
&
\sphinxAtStartPar
09/01/21 19:44:41
\\
\hline
\end{tabulary}
\par
\sphinxattableend\end{savenotes}


\chapter{Moored Instrument Observations}
\label{\detokenize{5_section:moored-instrument-observations}}\label{\detokenize{5_section::doc}}

\section{MicroCAT Data Processing Procedures}
\label{\detokenize{5_section:microcat-data-processing-procedures}}
\sphinxAtStartPar
Each moored MicroCAT temperature, conductivity, and pressure (when installed)
was calibrated at Sea\sphinxhyphen{}Bird before their deployment and after their recovery on
the dates shown in \hyperref[\detokenize{5_section:table-14}]{Table \ref{\detokenize{5_section:table-14}}}. The internally\sphinxhyphen{}recorded data from each
instrument were downloaded onboard the ship after the mooring recovery. The
nominally\sphinxhyphen{}calibrated data were plotted for a visual assessment of the data
quality. The data processing included checking the internal clock data against
external event times, pressure sensor drifts correction, temperature sensor
stability, and conductivity calibration against CTD data from casts conducted
near the mooring during HOT and WHOTS cruises. The detailed processing
procedures are described in this section.


\begin{savenotes}\sphinxattablestart
\centering
\sphinxcapstartof{table}
\sphinxthecaptionisattop
\sphinxcaption{WHOTS\sphinxhyphen{}19 MicroCAT temperature sensor calibration dates and sensor drift during deployments; \sphinxstyleemphasis{SN = Sea\sphinxhyphen{}Bird Serial Number; PDC = Pre\sphinxhyphen{}Deployment Calibration; PRC = Post\sphinxhyphen{}Recovery Calibration; TSA = Temperature Sensor’s Annual Drift during WHOTS\sphinxhyphen{}19 ; N. depth = Nominal deployment depth} }\label{\detokenize{5_section:table-14}}
\sphinxaftertopcaption
\begin{tabulary}{\linewidth}[t]{|T|T|T|T|T|}
\hline
\sphinxstyletheadfamily 
\sphinxAtStartPar
\sphinxstylestrong{N. depth (m)}
&\sphinxstyletheadfamily 
\sphinxAtStartPar
\sphinxstylestrong{SN}
&\sphinxstyletheadfamily 
\sphinxAtStartPar
\sphinxstylestrong{PDC}
&\sphinxstyletheadfamily 
\sphinxAtStartPar
\sphinxstylestrong{PRC}
&\sphinxstyletheadfamily 
\sphinxAtStartPar
\sphinxstylestrong{TSA(mili°C)}
\\
\hline
\sphinxAtStartPar
\sphinxstylestrong{7}
&
\sphinxAtStartPar
3617
&
\sphinxAtStartPar
21\sphinxhyphen{}Nov\sphinxhyphen{}18
&
\sphinxAtStartPar
17\sphinxhyphen{}Nov\sphinxhyphen{}21
&
\sphinxAtStartPar
\sphinxhyphen{}0.31
\\
\hline
\sphinxAtStartPar
\sphinxstylestrong{15}
&
\sphinxAtStartPar
6893
&
\sphinxAtStartPar
13\sphinxhyphen{}Dec\sphinxhyphen{}18
&
\sphinxAtStartPar
4\sphinxhyphen{}Nov\sphinxhyphen{}21
&
\sphinxAtStartPar
0.07
\\
\hline
\sphinxAtStartPar
\sphinxstylestrong{25}
&
\sphinxAtStartPar
6894
&
\sphinxAtStartPar
13\sphinxhyphen{}Dec\sphinxhyphen{}18
&
\sphinxAtStartPar
4\sphinxhyphen{}Nov\sphinxhyphen{}21
&
\sphinxAtStartPar
0.46
\\
\hline
\sphinxAtStartPar
\sphinxstylestrong{35}
&
\sphinxAtStartPar
6895
&
\sphinxAtStartPar
13\sphinxhyphen{}Dec\sphinxhyphen{}18
&
\sphinxAtStartPar
5\sphinxhyphen{}Nov\sphinxhyphen{}21
&
\sphinxAtStartPar
\sphinxhyphen{}0.33
\\
\hline
\sphinxAtStartPar
\sphinxstylestrong{40}
&
\sphinxAtStartPar
6896
&
\sphinxAtStartPar
12\sphinxhyphen{}Dec\sphinxhyphen{}18
&
\sphinxAtStartPar
5\sphinxhyphen{}Nov\sphinxhyphen{}21
&
\sphinxAtStartPar
\sphinxhyphen{}1.23
\\
\hline
\sphinxAtStartPar
\sphinxstylestrong{45}
&
\sphinxAtStartPar
6887
&
\sphinxAtStartPar
14\sphinxhyphen{}Dec\sphinxhyphen{}18
&
\sphinxAtStartPar
10\sphinxhyphen{}Nov\sphinxhyphen{}21
&
\sphinxAtStartPar
\sphinxhyphen{}0.14
\\
\hline
\sphinxAtStartPar
\sphinxstylestrong{50}
&
\sphinxAtStartPar
6897
&
\sphinxAtStartPar
12\sphinxhyphen{}Dec\sphinxhyphen{}18
&
\sphinxAtStartPar
4\sphinxhyphen{}Nov\sphinxhyphen{}21
&
\sphinxAtStartPar
0.6
\\
\hline
\sphinxAtStartPar
\sphinxstylestrong{55}
&
\sphinxAtStartPar
6898
&
\sphinxAtStartPar
14\sphinxhyphen{}Dec\sphinxhyphen{}18
&
\sphinxAtStartPar
4\sphinxhyphen{}Nov\sphinxhyphen{}21
&
\sphinxAtStartPar
0.02
\\
\hline
\sphinxAtStartPar
\sphinxstylestrong{65}
&
\sphinxAtStartPar
6899
&
\sphinxAtStartPar
13\sphinxhyphen{}Dec\sphinxhyphen{}18
&
\sphinxAtStartPar
4\sphinxhyphen{}Nov\sphinxhyphen{}21
&
\sphinxAtStartPar
0.34
\\
\hline
\sphinxAtStartPar
\sphinxstylestrong{75}
&
\sphinxAtStartPar
3618
&
\sphinxAtStartPar
18\sphinxhyphen{}Dec\sphinxhyphen{}18
&
\sphinxAtStartPar
4\sphinxhyphen{}Nov\sphinxhyphen{}21
&
\sphinxAtStartPar
0.46
\\
\hline
\sphinxAtStartPar
\sphinxstylestrong{85}
&
\sphinxAtStartPar
3634
&
\sphinxAtStartPar
13\sphinxhyphen{}Dec\sphinxhyphen{}18
&
\sphinxAtStartPar
4\sphinxhyphen{}Nov\sphinxhyphen{}21
&
\sphinxAtStartPar
1.36
\\
\hline
\sphinxAtStartPar
\sphinxstylestrong{95}
&
\sphinxAtStartPar
3670
&
\sphinxAtStartPar
14\sphinxhyphen{}Dec\sphinxhyphen{}18
&
\sphinxAtStartPar
10\sphinxhyphen{}Nov\sphinxhyphen{}21
&
\sphinxAtStartPar
\sphinxhyphen{}0.31
\\
\hline
\sphinxAtStartPar
\sphinxstylestrong{105}
&
\sphinxAtStartPar
6889
&
\sphinxAtStartPar
12\sphinxhyphen{}Dec\sphinxhyphen{}18
&
\sphinxAtStartPar
10\sphinxhyphen{}Nov\sphinxhyphen{}21
&
\sphinxAtStartPar
\sphinxhyphen{}0.05
\\
\hline
\sphinxAtStartPar
\sphinxstylestrong{120}
&
\sphinxAtStartPar
6890
&
\sphinxAtStartPar
12\sphinxhyphen{}Dec\sphinxhyphen{}18
&
\sphinxAtStartPar
10\sphinxhyphen{}Nov\sphinxhyphen{}21
&
\sphinxAtStartPar
\sphinxhyphen{}0.04
\\
\hline
\sphinxAtStartPar
\sphinxstylestrong{135}
&
\sphinxAtStartPar
6888
&
\sphinxAtStartPar
19\sphinxhyphen{}Dec\sphinxhyphen{}18
&
\sphinxAtStartPar
10\sphinxhyphen{}Nov\sphinxhyphen{}21
&
\sphinxAtStartPar
\sphinxhyphen{}0.54
\\
\hline
\sphinxAtStartPar
\sphinxstylestrong{155}
&
\sphinxAtStartPar
6891
&
\sphinxAtStartPar
19\sphinxhyphen{}Dec\sphinxhyphen{}18
&
\sphinxAtStartPar
10\sphinxhyphen{}Nov\sphinxhyphen{}21
&
\sphinxAtStartPar
\sphinxhyphen{}0.1
\\
\hline
\sphinxAtStartPar
\sphinxstylestrong{1875}
&
\sphinxAtStartPar
3639
&
\sphinxAtStartPar
15\sphinxhyphen{}Jun\sphinxhyphen{}16
&
\sphinxAtStartPar
27\sphinxhyphen{}Feb\sphinxhyphen{}22
&
\sphinxAtStartPar
\sphinxhyphen{}0.07
\\
\hline
\sphinxAtStartPar
\sphinxstylestrong{1875}
&
\sphinxAtStartPar
12242
&
\sphinxAtStartPar
25\sphinxhyphen{}May\sphinxhyphen{}14
&
\sphinxAtStartPar
25\sphinxhyphen{}Feb\sphinxhyphen{}22
&
\sphinxAtStartPar
\sphinxhyphen{}0.39
\\
\hline
\sphinxAtStartPar
\sphinxstylestrong{4713}
&
\sphinxAtStartPar
11391
&
\sphinxAtStartPar
7\sphinxhyphen{}Dec\sphinxhyphen{}13
&
\sphinxAtStartPar
10\sphinxhyphen{}Mar\sphinxhyphen{}22
&
\sphinxAtStartPar
\sphinxhyphen{}0.56
\\
\hline
\sphinxAtStartPar
\sphinxstylestrong{4713}
&
\sphinxAtStartPar
12241
&
\sphinxAtStartPar
23\sphinxhyphen{}May\sphinxhyphen{}14
&
\sphinxAtStartPar
25\sphinxhyphen{}Feb\sphinxhyphen{}22
&
\sphinxAtStartPar
\sphinxhyphen{}1.01
\\
\hline
\end{tabulary}
\par
\sphinxattableend\end{savenotes}


\subsection{Internal Clock Check and Missing Samples}
\label{\detokenize{5_section:internal-clock-check-and-missing-samples}}
\sphinxAtStartPar
Before the WHOTS\sphinxhyphen{}19 mooring deployment and after its recovery (before the data
logging was stopped), the MicroCATs temperature sensors were placed in contact
with an ice pack to create a spike in the data, to check for any problems with
their internal clocks, and for possible missing samples (\hyperref[\detokenize{3_section:table-8}]{Table \ref{\detokenize{3_section:table-8}}}).
However, it was found after recovery that all the instruments had stopped logging data
due to battery drainage, except for the deep instruments.
The cold spike before deployment was detected by a sudden decrease in temperature. For all the
instruments, the clock time of this event matched the time of the spike (within
the sampling interval of each instrument) correctly. No missing samples were
detected for any of the devices.


\subsection{Pressure Drift Correction and Pressure Variability}
\label{\detokenize{5_section:pressure-drift-correction-and-pressure-variability}}
\sphinxAtStartPar
Some MicroCATs used in the moorings were outfitted with pressure sensors (
\hyperref[\detokenize{3_section:mooring-subsurface}]{Table \ref{\detokenize{3_section:mooring-subsurface}}}). Biases were detected in the pressure sensors by
comparing the on\sphinxhyphen{}deck pressure readings (which should be zero for standard
atmospheric pressure at sea level of 1029 mbar) before deployment and after recovery.
\hyperref[\detokenize{5_section:table-15}]{Table \ref{\detokenize{5_section:table-15}}} shows the magnitude of the bias for each of the sensors
before and after deployment. To correct this offset, a linear fit between the
initial and final on\sphinxhyphen{}deck pressure offset as a function of time was obtained
and subtracted from each sensor. Only three of the deep instruments registered
on\sphinxhyphen{}deck pressure after recovery, all other instruments stopped recording data
before recovery due to battery drainage. For these last instruments only a\\
before\sphinxhyphen{}deployment pressure bias correction was applied.
\hyperref[\detokenize{5_section:figure5-1}]{Fig.\@ \ref{\detokenize{5_section:figure5-1}}} shows the linearly
corrected pressures measured by the MicroCATs located above 200 m during the
WHOTS\sphinxhyphen{}19 deployment. For all these sensors, the mean difference from the
nominal instrument pressure (based on the deployed depth) was less than 1.2
dbar. The standard deviation of the pressure for the duration of the record was
less than 1 dbar for all sensors, with the deeper sensors showing a slightly
larger standard deviation. The range of variability for all sensors was about \(\pm\)
3 dbar.

\sphinxAtStartPar
The causes of pressure variability can be several, including density variations
in the water column above the instrument; horizontal dynamic pressure (not only
due to the currents but also due to the motion of the mooring); mooring
position {[}\hyperlink{cite.references:id15}{Santiago\sphinxhyphen{}Mandujano \sphinxstyleemphasis{et al.}, 2007}{]}.


\begin{savenotes}\sphinxattablestart
\centering
\sphinxcapstartof{table}
\sphinxthecaptionisattop
\sphinxcaption{Pressure bias of MicroCATs with pressure sensors for WHOTS\sphinxhyphen{}19. All the instruments with a NA pressure bias ended recording before recovery. SN = Sea\sphinxhyphen{}bird Serial Number; BBD = Bias Before Deployment (dbar); BAR = Bias After Recovery (dbar)}\label{\detokenize{5_section:table-15}}
\sphinxaftertopcaption
\begin{tabulary}{\linewidth}[t]{|T|T|T|T|}
\hline
\sphinxstyletheadfamily 
\sphinxAtStartPar
\sphinxstylestrong{Depth (m)}
&\sphinxstyletheadfamily 
\sphinxAtStartPar
\sphinxstylestrong{SN}
&\sphinxstyletheadfamily 
\sphinxAtStartPar
\sphinxstylestrong{BBD(dbar)}
&\sphinxstyletheadfamily 
\sphinxAtStartPar
\sphinxstylestrong{BAR(dbar)}
\\
\hline
\sphinxAtStartPar
\sphinxstylestrong{45}
&
\sphinxAtStartPar
6887
&
\sphinxAtStartPar
0.07
&
\sphinxAtStartPar
NA
\\
\hline
\sphinxAtStartPar
\sphinxstylestrong{95}
&
\sphinxAtStartPar
3670
&
\sphinxAtStartPar
\sphinxhyphen{}1.2
&
\sphinxAtStartPar
NA
\\
\hline
\sphinxAtStartPar
\sphinxstylestrong{105}
&
\sphinxAtStartPar
6889
&
\sphinxAtStartPar
0.1
&
\sphinxAtStartPar
NA
\\
\hline
\sphinxAtStartPar
\sphinxstylestrong{120}
&
\sphinxAtStartPar
6890
&
\sphinxAtStartPar
0.11
&
\sphinxAtStartPar
NA
\\
\hline
\sphinxAtStartPar
\sphinxstylestrong{135}
&
\sphinxAtStartPar
6888
&
\sphinxAtStartPar
0.12
&
\sphinxAtStartPar
NA
\\
\hline
\sphinxAtStartPar
\sphinxstylestrong{155}
&
\sphinxAtStartPar
6891
&
\sphinxAtStartPar
0.07
&
\sphinxAtStartPar
NA
\\
\hline
\sphinxAtStartPar
\sphinxstylestrong{1875}
&
\sphinxAtStartPar
3639
&
\sphinxAtStartPar
\sphinxhyphen{}0.04
&
\sphinxAtStartPar
NA
\\
\hline
\sphinxAtStartPar
\sphinxstylestrong{1875}
&
\sphinxAtStartPar
12242
&
\sphinxAtStartPar
0.1
&
\sphinxAtStartPar
0.9
\\
\hline
\sphinxAtStartPar
\sphinxstylestrong{4713}
&
\sphinxAtStartPar
11391
&
\sphinxAtStartPar
0.5
&
\sphinxAtStartPar
2
\\
\hline
\sphinxAtStartPar
\sphinxstylestrong{4713}
&
\sphinxAtStartPar
12241
&
\sphinxAtStartPar
0.4
&
\sphinxAtStartPar
1.5
\\
\hline
\end{tabulary}
\par
\sphinxattableend\end{savenotes}

\begin{figure}[htbp]
\centering
\capstart

\noindent\sphinxincludegraphics[height=1000\sphinxpxdimen]{{w19pbias_a}.png}
\caption{Linearly corrected pressures from MicroCATs between 7 and 155 m during
WHOTS\sphinxhyphen{}19 deployment. The horizontal dashed line is the sensor’s nominal
pressure, based on deployed depth. The text on the left (right) side of the
figure indicates the mean (standard deviation) of the difference between each
instrument’s pressure and nominal pressure.}\label{\detokenize{5_section:figure5-1}}\end{figure}


\subsection{Temperature Sensor Stability}
\label{\detokenize{5_section:temperature-sensor-stability}}
\sphinxAtStartPar
The MicroCAT temperature sensors were calibrated at Sea\sphinxhyphen{}Bird before and after
each deployment, and their annual drift evaluations based on these calibrations
are shown in \hyperref[\detokenize{5_section:table-14}]{Table \ref{\detokenize{5_section:table-14}}}. These values turned out to be insignificant (
not higher than 0.002 °C) for all sensors. Comparisons between the MicroCAT and
CTD data from casts conducted near the mooring during HOT cruises confirmed
that the rest of the moored instruments’ temperature drift was insignificant. .
The two MicroCATs (SN 11391 and SN 12241) deployed near the bottom were drift
corrected. \hyperref[\detokenize{5_section:figure5-7}]{Fig.\@ \ref{\detokenize{5_section:figure5-7}}} (upper panel) shows the temperature differences
between both instruments before and after the correction. After the correction,
the temperature differences were in the \(\pm\)0.001 °C range.

\sphinxAtStartPar
Temperature comparisons between one of the WHOTS\sphinxhyphen{}19 near\sphinxhyphen{}surface MicroCAT
(SN 1834) and the four SBE\sphinxhyphen{}56 surface temperature sensors in the buoy hull
\hyperref[\detokenize{3_section:table-7}]{Table \ref{\detokenize{3_section:table-7}}} are shown in \hyperref[\detokenize{5_section:figure5-2}]{Fig.\@ \ref{\detokenize{5_section:figure5-2}}}. All the SBE\sphinxhyphen{}56 instruments
returned full records, and none of them show any obvious bias compared to the
Microcat measurements.

\begin{figure}[htbp]
\centering
\capstart

\noindent\sphinxincludegraphics[height=1000\sphinxpxdimen]{{w19tcompare_1}.png}
\caption{The temperature difference between MicroCAT SN 7212 at 1 m, and near\sphinxhyphen{}surface
temperature sensors SN 7212 (top panel), 7213 (second panel), 7214 (third
panel), and 7215 (bottom panel), during the WHOTS\sphinxhyphen{}19 deployment. The light blue
line is a 24\sphinxhyphen{}hour running mean of the differences.}\label{\detokenize{5_section:figure5-2}}\end{figure}

\sphinxAtStartPar
In addition to the Sea\sphinxhyphen{}Bird temperature sensors, there were additional
temperature sensors in the VMCMs (at 10 and 30 m) and in the ADCPs (at 47.5 m
and 125 m). Comparisons with the temperatures from adjacent MicroCATs were
conducted to evaluate the temperatures from those sensors.


\subsubsection{Comparisons with VMCM and ADCP temperature sensors}
\label{\detokenize{5_section:comparisons-with-vmcm-and-adcp-temperature-sensors}}
\sphinxAtStartPar
The upper panel of \hyperref[\detokenize{5_section:figure5-3}]{Fig.\@ \ref{\detokenize{5_section:figure5-3}}} shows the difference between the 10\sphinxhyphen{}m
VMCM and the 7\sphinxhyphen{}m MicroCAT temperatures during WHOTS\sphinxhyphen{}19, after adding a 0.0259°C
offset correction to the VMCM. The offset was the mean difference between
the uncorrected VMCM and the 7\sphinxhyphen{}m MicroCAT data. Also shown for comparison in
the middle panel of the figure are the corrected VMCM temperature differences
from the 15 m MicroCAT. The VMCM temperatures had a 0.04 °C offset in April
2020. The lower panel shows the temperature fluctuations in the differences
between the 7 and 15\sphinxhyphen{}m MicroCATs, which seem to be around zero.

\sphinxAtStartPar
Temperature differences between the 30\sphinxhyphen{}m VMCM and the temperatures from
adjacent MicroCATs at 25 and 35\sphinxhyphen{}m during WHOTS\sphinxhyphen{}19 are shown in
\hyperref[\detokenize{5_section:figure5-4}]{Fig.\@ \ref{\detokenize{5_section:figure5-4}}} after adding a 0.0147°C offset correction to the VMCM. The
offset was the mean difference between the uncorrected VMCM and the 25\sphinxhyphen{}m
MicroCAT data. For comparison, the differences between the MicroCATs
temperatures are also shown in the lower panel.

\sphinxAtStartPar
Temperature differences between the 47.5\sphinxhyphen{}m ADCP and the temperatures from
adjacent MicroCATs at 45 and 50\sphinxhyphen{}m during WHOTS\sphinxhyphen{}19 are shown in
\hyperref[\detokenize{5_section:figure5-5}]{Fig.\@ \ref{\detokenize{5_section:figure5-5}}}. The ADCP failed and stopped collecting data on January
21, 2020 (see {\hyperref[\detokenize{3_section:description-of-whots-19-mooring}]{\sphinxcrossref{\DUrole{std,std-ref}{Description of WHOTS\sphinxhyphen{}19 Mooring}}}}). For
comparison, the differences between the MicroCATs temperatures are also
shown in the lower panel.

\sphinxAtStartPar
Temperature differences between the 125\sphinxhyphen{}m ADCP and the temperatures from
adjacent MicroCATs at 120 and 135\sphinxhyphen{}m during WHOTS\sphinxhyphen{}19 are shown in
\hyperref[\detokenize{5_section:figure5-6}]{Fig.\@ \ref{\detokenize{5_section:figure5-6}}}. For comparison, the differences between the MicroCATs
temperatures are also shown in the lower panel. It is difficult to assess the
quality of the ADCP temperature from these comparisons. These sensors were
located at the top of the thermocline, where we expect to find substantial
temperature differences between adjacent sensors. However, an indication of the
ADCP temperatures’ quality is given in the upper panel plot, which shows
temperatures fluctuating closely around zero.

\begin{figure}[htbp]
\centering
\capstart

\noindent\sphinxincludegraphics[height=1000\sphinxpxdimen]{{w19tcompare_22}.png}
\caption{The temperature difference between the 7\sphinxhyphen{}m MicroCAT and the 10\sphinxhyphen{}m VMCM (upper
pane)l; between the 15\sphinxhyphen{}m MicroCAT and the 10\sphinxhyphen{}m VMCM (middle panel); and between
the 7\sphinxhyphen{}m and the 15\sphinxhyphen{}m MicroCATs (lower panel ) during the WHOTS\sphinxhyphen{}19 deployment.
The light blue line is a 24\sphinxhyphen{}hour running mean of the differences.}\label{\detokenize{5_section:figure5-3}}\end{figure}

\begin{figure}[htbp]
\centering
\capstart

\noindent\sphinxincludegraphics[height=1000\sphinxpxdimen]{{figures/microcats/w19tcompare_33}.png}
\caption{The temperature difference between the 25\sphinxhyphen{}m MicroCAT and the 30\sphinxhyphen{}m VMCM (upper
panel); between the 35\sphinxhyphen{}m MicroCAT and the 30\sphinxhyphen{}m VMCM (middle panel); and between
the 25\sphinxhyphen{}m and the 35\sphinxhyphen{}m MicroCATs (lower panel) during the WHOTS\sphinxhyphen{}19 deployment.
The light blue line is a 24\sphinxhyphen{}hour running mean of the differences.}\label{\detokenize{5_section:figure5-4}}\end{figure}

\begin{figure}[htbp]
\centering
\capstart

\noindent\sphinxincludegraphics[height=1000\sphinxpxdimen]{{w19tcompare_4}.png}
\caption{The temperature difference between the 45\sphinxhyphen{}m MicroCAT and the 47.5\sphinxhyphen{}m ADCP (upper
panel). (The ADCP stopped collecting data on 2020/1/21); between the 50\sphinxhyphen{}m
MicroCAT and the 47.5\sphinxhyphen{}m ADCP (middle panel); and between the 45\sphinxhyphen{}m and the 50\sphinxhyphen{}m
MicroCATs (lower panel) during the WHOTS\sphinxhyphen{}19 deployment. The light blue line is
a 24\sphinxhyphen{}hour running mean of the differences.}\label{\detokenize{5_section:figure5-5}}\end{figure}

\begin{figure}[htbp]
\centering
\capstart

\noindent\sphinxincludegraphics[height=1000\sphinxpxdimen]{{w19tcompare_5}.png}
\caption{The temperature difference between the 120\sphinxhyphen{}m MicroCAT and the 125\sphinxhyphen{}m ADCP (upper
panel); between the 135\sphinxhyphen{}m MicroCAT and the 125\sphinxhyphen{}m ADCP (middle panel); and
between the 120\sphinxhyphen{}m and the 135\sphinxhyphen{}m MicroCATs (lower panel) during the WHOTS\sphinxhyphen{}19
deployment. The light blue line is a 24\sphinxhyphen{}hour running mean of the differences.}\label{\detokenize{5_section:figure5-6}}\end{figure}


\subsection{Conductivity Calibration}
\label{\detokenize{5_section:conductivity-calibration}}
\sphinxAtStartPar
The results of the Sea\sphinxhyphen{}Bird post\sphinxhyphen{}recovery conductivity calibrations indicated
that some MicroCAT conductivity sensors experienced relatively large offsets
from their pre\sphinxhyphen{}deployment calibration. These were qualitatively confirmed by
comparing the mooring data against CTD data from casts conducted between 200 m
and 5 km from the mooring during HOT cruises. The conductivity offsets are not
apparent, and there may have been multiple causes ( see {[}\hyperlink{cite.references:id4}{Freitag \sphinxstyleemphasis{et al.}, 1999}{]}
for a similar experience with conductivity cells during COARE). For some
instruments, the offset was negative, caused perhaps by biofouling of the
conductivity cell. In contrast, for others, the offset was positive, for
reasons still unknown. A visual
inspection of the instruments after recovery did not show any apparent signs of
biofouling. There were no cell scourings reported in the post\sphinxhyphen{}recovery reviews
at Sea\sphinxhyphen{}Bird.

\sphinxAtStartPar
Corrections of the MicroCATs conductivity data were conducted by comparing them
against CTD data from profiles and yo\sphinxhyphen{}yo casts conducted near the mooring
during HOT cruises and during deployment/recovery cruises. Casts led between
200 and 1000 m from the mooring were given extra weight in the correction
compared to those conducted between 1 and 5 km away. Casts more than 5 km away
from the mooring were not used. Given that the CTD casts are conducted at least
200 m from the mooring, CTD and MicroCAT data’s alignment was done in density
rather than in\sphinxhyphen{}depth. For cases where the alignment in density was not possible
due to large conductivity offsets (causing unrealistic mooring density values),
the alignment was done in temperature space. A cubic least\sphinxhyphen{}squares fit (LSF) to
the CTD\sphinxhyphen{}MicroCAT differences against time was applied as a first approximation,
and the corresponding correction was applied.

\sphinxAtStartPar
Some sensors had large offsets and noticeable variability that could not
be explained by a cubic LSF (see below). For these sensors, a stepwise
correction was applied to match the data to the available CTD cast data and
then to use the differences between consecutive sensors to determine when the
sensor started to drift. For instance, during periods of weak stratification,
the conductivity difference between neighboring sensors A, B, and C could reach
near\sphinxhyphen{}zero values, in particular for instruments near the surface, which are the
ones most prone to suffer conductivity offsets. A sudden conductivity offset
observed during this period between sensors A and B, but not between sensors A
and C could indicate the beginning of an offset for sensor B.

\sphinxAtStartPar
Given that the most in\sphinxhyphen{}depth instruments on the mooring are less likely to be
affected by biofouling and consequent sudden conductivity drift, the deep
instruments served as an excellent reference to find any possible malfunction
in the shallower ones. Therefore, the conductivity from the deepest instruments
was corrected first, and the correction was continued sequentially upwards
toward the shallower ones.

\sphinxAtStartPar
As a quality control to the conductivity corrections, the buoyancy frequency
between neighboring instruments was calculated using finite differences. Over\sphinxhyphen{}
or under\sphinxhyphen{}corrected conductivities yielded instabilities in the water column (
negative buoyancy frequency) that were easy to detect and were not real when
lasting for several days. Based on this, the conductivity correction of the
corresponding sensors was revised.

\sphinxAtStartPar
Correction of the deep and the near\sphinxhyphen{}bottom MicroCATs’ conductivities were done following
similar procedures than for the shallow instruments, by comparing them
against CTD data from near\sphinxhyphen{}bottom profiles conducted during HOT cruises
(\hyperref[\detokenize{5_section:figure5-7}]{Fig.\@ \ref{\detokenize{5_section:figure5-7}}}, bottom panel). After correction, the salinity
differences between both instruments were in the \(\pm\)0.001 range.

\sphinxAtStartPar
Another characteristic of the offsets in the conductivity sensors is that their
development is not always linear in time. Their behavior can be highly variable
{[}\hyperlink{cite.references:id15}{Santiago\sphinxhyphen{}Mandujano \sphinxstyleemphasis{et al.}, 2007}{]}. The corrections applied to each of the
conductivity sensors during WHOTS\sphinxhyphen{}19 are shown in \hyperref[\detokenize{5_section:figure5-8}]{Fig.\@ \ref{\detokenize{5_section:figure5-8}}}
through \hyperref[\detokenize{5_section:figure5-15}]{Fig.\@ \ref{\detokenize{5_section:figure5-15}}}. Most of the instruments had a drift of less than
0.04 Siemens/m for the duration of the deployment (except for the near\sphinxhyphen{}surface
instrument SN 1834 which had a 0.08 S/m drift), corrected with a linear or
cubic least\sphinxhyphen{}squares fit. Many of the instruments deployed above 120 m showed a
negative drift starting a few months before the end of their record, apparently
due to the anti\sphinxhyphen{}foulant expiration.

\begin{figure}[htbp]
\centering
\capstart

\noindent\sphinxincludegraphics[height=1000\sphinxpxdimen]{{plt_w19_deep_corr}.png}
\caption{Temperature differences (top panel) and salinity differences (bottom panel)
between MicroCATs \#12241 and \#11391 during WHOTS\sphinxhyphen{}19. The blue (red) lines are
the differences before (after) correcting the data following the text’s
procedures.}\label{\detokenize{5_section:figure5-7}}\end{figure}

\begin{figure}[htbp]
\centering
\capstart

\noindent\sphinxincludegraphics[height=1000\sphinxpxdimen]{{w19mic_corr1}.jpg}
\caption{Conductivity sensor corrections for MicroCATs from 1 to 7 meters during
WHOTS\sphinxhyphen{}19.}\label{\detokenize{5_section:figure5-8}}\end{figure}

\begin{figure}[htbp]
\centering
\capstart

\noindent\sphinxincludegraphics[height=1000\sphinxpxdimen]{{w19mic_corr2}.jpg}
\caption{Conductivity sensor corrections for MicroCATs from 15 to 35 meters during
WHOTS\sphinxhyphen{}19}\label{\detokenize{5_section:figure5-9}}\end{figure}

\begin{figure}[htbp]
\centering
\capstart

\noindent\sphinxincludegraphics[height=1000\sphinxpxdimen]{{w19mic_corr3}.jpg}
\caption{Conductivity sensor corrections for MicroCATs from 40 to 50 meters during
WHOTS\sphinxhyphen{}19}\label{\detokenize{5_section:figure5-10}}\end{figure}

\begin{figure}[htbp]
\centering
\capstart

\noindent\sphinxincludegraphics[height=1000\sphinxpxdimen]{{w19mic_corr4}.jpg}
\caption{Conductivity sensor corrections for MicroCATs from 55 to 75 meters during
WHOTS\sphinxhyphen{}19.}\label{\detokenize{5_section:figure5-11}}\end{figure}

\begin{figure}[htbp]
\centering
\capstart

\noindent\sphinxincludegraphics[height=1000\sphinxpxdimen]{{w19mic_corr5}.jpg}
\caption{Conductivity sensor corrections for MicroCATs from 85 to 105 meters during
WHOTS\sphinxhyphen{}19.}\label{\detokenize{5_section:figure5-12}}\end{figure}

\begin{figure}[htbp]
\centering
\capstart

\noindent\sphinxincludegraphics[height=1000\sphinxpxdimen]{{w19mic_corr6}.jpg}
\caption{Conductivity sensor corrections for MicroCATs from 120 to 155 meters during
WHOTS\sphinxhyphen{}19}\label{\detokenize{5_section:figure5-13}}\end{figure}

\begin{figure}[htbp]
\centering
\capstart

\noindent\sphinxincludegraphics[height=1000\sphinxpxdimen]{{w19mic_corr7}.jpg}
\caption{Conductivity sensor corrections for MicroCATs at 1875 and 4713 meters
during WHOTS\sphinxhyphen{}19.}\label{\detokenize{5_section:figure5-14}}\end{figure}

\begin{figure}[htbp]
\centering
\capstart

\noindent\sphinxincludegraphics[height=1000\sphinxpxdimen]{{figures/microcats/w19mic_corr8}.jpg}
\caption{Conductivity sensor correction for MicroCAT at 4713 meters during WHOTS\sphinxhyphen{}19}\label{\detokenize{5_section:figure5-15}}\end{figure}


\section{Acoustic Doppler Current Profiler}
\label{\detokenize{5_section:acoustic-doppler-current-profiler}}
\sphinxAtStartPar
Two TRDI broadband Workhorse Sentinel ADCP’s were deployed on the WHOTS\sphinxhyphen{}19
mooring. A 600 kHz ADCP was deployed at 47.5 m depth in the upward\sphinxhyphen{}looking
configuration, and a 300 kHz ADCP was deployed at 125 m, also in the
upward\sphinxhyphen{}looking configuration. The instruments were installed in aluminum frames
and an external battery module to provide sufficient power for the intended
period of deployment. The four ADCP beams were angled at 20° from the vertical
line of the instrument. The 300 kHz ADCP was set to profile across 30 range
cells of 4 m with the first bin centered at 6.25m from the transducer. The 600
kHz ADCP was set to profile across 25 range cells of 4 m with the first bin
centered at 3.12m from the transducer. The specifications of the instrument are
shown in \hyperref[\detokenize{5_section:table-16}]{Table \ref{\detokenize{5_section:table-16}}}.


\begin{savenotes}\sphinxattablestart
\centering
\sphinxcapstartof{table}
\sphinxthecaptionisattop
\sphinxcaption{Specifications of the ADCP’s used for the WHOTS\sphinxhyphen{}19 mooring.}\label{\detokenize{5_section:table-16}}
\sphinxaftertopcaption
\begin{tabulary}{\linewidth}[t]{|T|T|T|T|}
\hline
\sphinxstyletheadfamily 
\sphinxAtStartPar
\sphinxstylestrong{Frequency (kHz)}
&\sphinxstyletheadfamily 
\sphinxAtStartPar
\sphinxstylestrong{Instrument}
&\sphinxstyletheadfamily 
\sphinxAtStartPar
\sphinxstylestrong{Model}
&\sphinxstyletheadfamily 
\sphinxAtStartPar
\sphinxstylestrong{Serial Number}
\\
\hline
\sphinxAtStartPar
\sphinxstylestrong{300}
&
\sphinxAtStartPar
TRDI Workhorse Sentinel
&
\sphinxAtStartPar
WHS300\sphinxhyphen{}I\sphinxhyphen{}UG86
&
\sphinxAtStartPar
4891
\\
\hline
\sphinxAtStartPar
\sphinxstylestrong{600}
&
\sphinxAtStartPar
TRDI Workhorse Sentinel
&
\sphinxAtStartPar
WHS600\sphinxhyphen{}I
&
\sphinxAtStartPar
1825
\\
\hline
\end{tabulary}
\par
\sphinxattableend\end{savenotes}


\subsection{Compass Calibrations}
\label{\detokenize{5_section:compass-calibrations}}

\subsubsection{Pre\sphinxhyphen{}Deployment}
\label{\detokenize{5_section:pre-deployment}}
\sphinxAtStartPar
Before the WHOTS\sphinxhyphen{}19 deployment, field calibration of the internal ADCP compass
was performed at the University of Hawaii’s soccer field at Manoa on September
2019, for 300 kHz and the 600 kHz instruments. Each instrument was mounted in
the deployment cage with the external battery module and was located away from
potential sources of magnetic field disturbances. The ADCP was mounted to a
turntable, aligned with the magnetic north using a surveyor’s compass. Using
the built\sphinxhyphen{}in RDI calibration procedure, the instrument was tilted in one
direction between 10 and 20 degrees and then rotated through 360 degrees at
less than 5° per second. The ADCP was then tilted in a different direction, and
a second rotation was made. Based on the results from the first two rotations,
calibration parameters are temporarily loaded, and the instrument, tilted in a
third direction, is rotated once more to check the calibration. Results from
each pre\sphinxhyphen{}deployment field calibration are shown in \hyperref[\detokenize{5_section:table-17}]{Table \ref{\detokenize{5_section:table-17}}} and
\hyperref[\detokenize{5_section:table-18}]{Table \ref{\detokenize{5_section:table-18}}} (\hyperref[\detokenize{5_section:figure5-16}]{Fig.\@ \ref{\detokenize{5_section:figure5-16}}} and \hyperref[\detokenize{5_section:figure5-17}]{Fig.\@ \ref{\detokenize{5_section:figure5-17}}}).


\begin{savenotes}\sphinxattablestart
\centering
\sphinxcapstartof{table}
\sphinxthecaptionisattop
\sphinxcaption{Results from the WHOTS\sphinxhyphen{}19 pre\sphinxhyphen{}deployment 300 kHz ADCP compass field calibration procedure. \sphinxstyleemphasis{SCE = Single Cycle Error (°); DCE = Double Cycle Error (°); LD\_SCE = Largest Double + Single Cycle Error (°); RMS\_RE = RMS of 3rd Order and Higher + Random Error (°); OE = Overall Error (°); PM\_STD = Pitch, Mean and St. Deviation (°); RM\_STD = Roll, Mean and St. Dev. (°)}}\label{\detokenize{5_section:table-17}}
\sphinxaftertopcaption
\begin{tabulary}{\linewidth}[t]{|T|T|T|T|T|T|T|T|}
\hline
\sphinxstyletheadfamily 
\sphinxAtStartPar
\sphinxstylestrong{(SN 4891)}
&\sphinxstyletheadfamily 
\sphinxAtStartPar
\sphinxstylestrong{SCE}
&\sphinxstyletheadfamily 
\sphinxAtStartPar
\sphinxstylestrong{DCE}
&\sphinxstyletheadfamily 
\sphinxAtStartPar
\sphinxstylestrong{LD\_SCE}
&\sphinxstyletheadfamily 
\sphinxAtStartPar
\sphinxstylestrong{RMS\_RE}
&\sphinxstyletheadfamily 
\sphinxAtStartPar
\sphinxstylestrong{OE}
&\sphinxstyletheadfamily 
\sphinxAtStartPar
\sphinxstylestrong{PM\_STD}
&\sphinxstyletheadfamily 
\sphinxAtStartPar
\sphinxstylestrong{RM\_STD}
\\
\hline
\sphinxAtStartPar
\sphinxstylestrong{Before}
&
\sphinxAtStartPar
2.41
&
\sphinxAtStartPar
0.47
&
\sphinxAtStartPar
2.88
&
\sphinxAtStartPar
0.23
&
\sphinxAtStartPar
2.39
&
\sphinxAtStartPar
1.64 \(\pm\)0.43
&
\sphinxAtStartPar
0.1\(\pm\)0.42
\\
\hline
\sphinxAtStartPar
\sphinxstylestrong{After}
&
\sphinxAtStartPar
0.78
&
\sphinxAtStartPar
0.07
&
\sphinxAtStartPar
0.85
&
\sphinxAtStartPar
0.37
&
\sphinxAtStartPar
0.79
&
\sphinxAtStartPar
\sphinxhyphen{}17.08 \(\pm\)0.44
&
\sphinxAtStartPar
\sphinxhyphen{}0.01\(\pm\)0.81
\\
\hline
\end{tabulary}
\par
\sphinxattableend\end{savenotes}


\begin{savenotes}\sphinxattablestart
\centering
\sphinxcapstartof{table}
\sphinxthecaptionisattop
\sphinxcaption{Results from the WHOTS\sphinxhyphen{}19 pre\sphinxhyphen{}deployment 600 kHz ADCP compass field calibration procedure. See acronyms on Table 5.4}\label{\detokenize{5_section:table-18}}
\sphinxaftertopcaption
\begin{tabulary}{\linewidth}[t]{|T|T|T|T|T|T|T|T|}
\hline
\sphinxstyletheadfamily 
\sphinxAtStartPar
\sphinxstylestrong{(SN 1825)}
&\sphinxstyletheadfamily 
\sphinxAtStartPar
\sphinxstylestrong{SCE}
&\sphinxstyletheadfamily 
\sphinxAtStartPar
\sphinxstylestrong{DCE}
&\sphinxstyletheadfamily 
\sphinxAtStartPar
\sphinxstylestrong{LD\_SCE}
&\sphinxstyletheadfamily 
\sphinxAtStartPar
\sphinxstylestrong{RMS\_RE}
&\sphinxstyletheadfamily 
\sphinxAtStartPar
\sphinxstylestrong{OE}
&\sphinxstyletheadfamily 
\sphinxAtStartPar
\sphinxstylestrong{PM\_STD}
&\sphinxstyletheadfamily 
\sphinxAtStartPar
\sphinxstylestrong{RM\_STD}
\\
\hline
\sphinxAtStartPar
\sphinxstylestrong{Before}
&
\sphinxAtStartPar
1.44
&
\sphinxAtStartPar
0.27
&
\sphinxAtStartPar
1.71
&
\sphinxAtStartPar
0.13
&
\sphinxAtStartPar
1.52
&
\sphinxAtStartPar
1.42 \(\pm\)0.35
&
\sphinxAtStartPar
0.47 \(\pm\)0.34
\\
\hline
\sphinxAtStartPar
\sphinxstylestrong{After}
&
\sphinxAtStartPar
0.15
&
\sphinxAtStartPar
0.27
&
\sphinxAtStartPar
0.42
&
\sphinxAtStartPar
0.21
&
\sphinxAtStartPar
0.37
&
\sphinxAtStartPar
\sphinxhyphen{}17.76 \(\pm\)0.33
&
\sphinxAtStartPar
\sphinxhyphen{}0.85\(\pm\)0.69
\\
\hline
\end{tabulary}
\par
\sphinxattableend\end{savenotes}


\subsubsection{Post\sphinxhyphen{}Deployment}
\label{\detokenize{5_section:post-deployment}}
\sphinxAtStartPar
After the WHOTS\sphinxhyphen{}19 mooring was recovered, the ADCP compass’s performance was
tested at the University of Hawai’i’s soccer field at Manoa on September 9,
2021, with an identical compass calibration procedure as during the
pre\sphinxhyphen{}deployment calibration. Results from the WHOTS\sphinxhyphen{}19 post\sphinxhyphen{}deployment ADCP
compass field calibration procedure are listed in \hyperref[\detokenize{5_section:table-19}]{Table \ref{\detokenize{5_section:table-19}}} and
\hyperref[\detokenize{5_section:table-20}]{Table \ref{\detokenize{5_section:table-20}}} (\hyperref[\detokenize{5_section:figure5-16}]{Fig.\@ \ref{\detokenize{5_section:figure5-16}}} and \hyperref[\detokenize{5_section:figure5-17}]{Fig.\@ \ref{\detokenize{5_section:figure5-17}}}).


\begin{savenotes}\sphinxattablestart
\centering
\sphinxcapstartof{table}
\sphinxthecaptionisattop
\sphinxcaption{Results from the WHOTS\sphinxhyphen{}19 post\sphinxhyphen{}deployment 300kHz ADCP compass field calibration procedure. See acronyms on Table 5.4 }\label{\detokenize{5_section:table-19}}
\sphinxaftertopcaption
\begin{tabulary}{\linewidth}[t]{|T|T|T|T|T|T|T|T|}
\hline
\sphinxstyletheadfamily 
\sphinxAtStartPar
\sphinxstylestrong{(SN 4891)}
&\sphinxstyletheadfamily 
\sphinxAtStartPar
\sphinxstylestrong{SCE}
&\sphinxstyletheadfamily 
\sphinxAtStartPar
\sphinxstylestrong{DCE}
&\sphinxstyletheadfamily 
\sphinxAtStartPar
\sphinxstylestrong{LD\_SCE}
&\sphinxstyletheadfamily 
\sphinxAtStartPar
\sphinxstylestrong{RMS\_RE}
&\sphinxstyletheadfamily 
\sphinxAtStartPar
\sphinxstylestrong{OE}
&\sphinxstyletheadfamily 
\sphinxAtStartPar
\sphinxstylestrong{PM\_STD}
&\sphinxstyletheadfamily 
\sphinxAtStartPar
\sphinxstylestrong{RM\_STD}
\\
\hline
\sphinxAtStartPar
\sphinxstylestrong{After}
&
\sphinxAtStartPar
2.00
&
\sphinxAtStartPar
0.05
&
\sphinxAtStartPar
2.05
&
\sphinxAtStartPar
0.15
&
\sphinxAtStartPar
2.00
&
\sphinxAtStartPar
\sphinxhyphen{}0.03 \(\pm\)0.40
&
\sphinxAtStartPar
0.13\(\pm\)0.49
\\
\hline
\end{tabulary}
\par
\sphinxattableend\end{savenotes}


\begin{savenotes}\sphinxattablestart
\centering
\sphinxcapstartof{table}
\sphinxthecaptionisattop
\sphinxcaption{Results from the WHOTS\sphinxhyphen{}19 post\sphinxhyphen{}deployment 600kHz ADCP compass field calibration procedure. See acronyms on Table 5.4 }\label{\detokenize{5_section:table-20}}
\sphinxaftertopcaption
\begin{tabulary}{\linewidth}[t]{|T|T|T|T|T|T|T|T|}
\hline
\sphinxstyletheadfamily 
\sphinxAtStartPar
\sphinxstylestrong{(SN 1825)}
&\sphinxstyletheadfamily 
\sphinxAtStartPar
\sphinxstylestrong{SCE}
&\sphinxstyletheadfamily 
\sphinxAtStartPar
\sphinxstylestrong{DCE}
&\sphinxstyletheadfamily 
\sphinxAtStartPar
\sphinxstylestrong{LD\_SCE}
&\sphinxstyletheadfamily 
\sphinxAtStartPar
\sphinxstylestrong{RMS\_RE}
&\sphinxstyletheadfamily 
\sphinxAtStartPar
\sphinxstylestrong{OE}
&\sphinxstyletheadfamily 
\sphinxAtStartPar
\sphinxstylestrong{PM\_STD}
&\sphinxstyletheadfamily 
\sphinxAtStartPar
\sphinxstylestrong{RM\_STD}
\\
\hline
\sphinxAtStartPar
\sphinxstylestrong{After}
&
\sphinxAtStartPar
0.99
&
\sphinxAtStartPar
0.23
&
\sphinxAtStartPar
1.21
&
\sphinxAtStartPar
0.19
&
\sphinxAtStartPar
1.07
&
\sphinxAtStartPar
0.10 \(\pm\)0.43
&
\sphinxAtStartPar
0.47\(\pm\)0.41
\\
\hline
\end{tabulary}
\par
\sphinxattableend\end{savenotes}

\begin{figure}[htbp]
\centering
\capstart

\noindent\sphinxincludegraphics[height=600\sphinxpxdimen]{{figures/adcp_moored/adcp_whot19cmpserr_sn4891}.png}
\caption{Results of the post\sphinxhyphen{}cruise compass calibration, conducted September 9, 2021,
on ADCP SN 4891 at the University of Hawai’i at Manoa.}\label{\detokenize{5_section:figure5-16}}\end{figure}

\begin{figure}[htbp]
\centering
\capstart

\noindent\sphinxincludegraphics[height=600\sphinxpxdimen]{{figures/adcp_moored/adcp_whot19cmpserr_sn1825}.png}
\caption{Results of the post\sphinxhyphen{}cruise compass calibration, conducted September 9, 2021,
on ADCP SN 1875 at the University of Hawai’i at Manoa.}\label{\detokenize{5_section:figure5-17}}\end{figure}


\subsection{ADCP Configurations}
\label{\detokenize{5_section:adcp-configurations}}
\sphinxAtStartPar
Individual configurations for the two ADCP’s on the WHOTS\sphinxhyphen{}19 mooring are
detailed in {\hyperref[\detokenize{appendices:whots-19-300-khz-serial-4891}]{\sphinxcrossref{\DUrole{std,std-ref}{WHOTS\sphinxhyphen{}19 300 kHz \sphinxhyphen{} Serial 4891}}}}, and
{\hyperref[\detokenize{appendices:whots-19-600-khz-serial-1825}]{\sphinxcrossref{\DUrole{std,std-ref}{WHOTS\sphinxhyphen{}19 600 kHz \sphinxhyphen{} Serial 1825}}}}. The salient differences for
each of the ADCP’s are summarized below.


\subsubsection{300 kHz (SN/4891 \sphinxhyphen{} 125m)}
\label{\detokenize{5_section:khz-sn-4891-125m}}
\sphinxAtStartPar
The ADCP, set to a beam frequency of 300 kHz, was configured in a burst
sampling mode consisting of 40 pings per ensemble to resolve low\sphinxhyphen{}frequency wave
orbital motions. The interval between each ping was 4 seconds, so the ensemble
length was 160 seconds. The interval between ensembles was 10 minutes. Data
were recorded in earth coordinates, with a heading bias of 9.54° E due to
magnetic declination. False targets, usually fish, were screened by setting the
threshold maximum to 70 counts. Velocity data were rejected if the difference
in echo intensity among the four beams exceeded this threshold.


\subsubsection{600 kHz (SN/1825\sphinxhyphen{} 47.5m)}
\label{\detokenize{5_section:khz-sn-1825-47-5m}}
\sphinxAtStartPar
The ADCP, set to a beam frequency of 600 kHz, was configured in a burst
sampling mode consisting of 80 pings per ensemble. The interval between each
ping was 2 seconds, so the ensemble length was also 160 seconds. The interval
between ensembles was 10 minutes. Data were recorded in earth coordinates with
a heading bias of 9.54° E. The threshold maximum was also set to 70 counts.
Velocity data were rejected if the difference in echo intensity among the four
beams exceeded this threshold.


\subsection{ADCP data processing procedures}
\label{\detokenize{5_section:adcp-data-processing-procedures}}
\sphinxAtStartPar
Binary files output from the ADCP were read and converted to MATLAB™ binary
files using scripts developed by
\sphinxhref{https://currents.soest.hawaii.edu}{Eric Firing’s ADCP lab}. The beginning of
the raw data files was truncated to a time after the mooring anchor was
released to allow time for the anchor to reach the seabed and for the mooring
motions that follow the anchor’s impact on the seafloor to dissipate. The
pitch, roll, and ADCP temperature were examined to pick reasonable times that
ensured good data quality without unnecessarily discarding too much data
(\hyperref[\detokenize{5_section:figure5-18}]{Fig.\@ \ref{\detokenize{5_section:figure5-18}}}, \hyperref[\detokenize{5_section:figure5-19}]{Fig.\@ \ref{\detokenize{5_section:figure5-19}}}). Truncation at the end of the data
files was chosen to be the ensemble before the acoustic release signal was sent
to avoid contamination due to the instrument’s ascent. The times of the first
ensemble from the raw data, deployments, and recovery time, along with the
truncated records of both deployments, are shown in \hyperref[\detokenize{5_section:table-21}]{Table \ref{\detokenize{5_section:table-21}}}.

\begin{figure}[htbp]
\centering
\capstart

\noindent\sphinxincludegraphics[height=500\sphinxpxdimen]{{300_rawt_plt}.png}
\caption{Temperature record from the 300 kHz ADCP during WHOTS\sphinxhyphen{}19 mooring (top panel).
The bottom panel shows the beginning and end of the record, with the green
vertical line representing the in\sphinxhyphen{}water time during deployment and out\sphinxhyphen{}of\sphinxhyphen{}water
recovery time. The red line represents the anchor release and acoustic release
trigger for deployment and recovery, respectively.}\label{\detokenize{5_section:figure5-18}}\end{figure}

\begin{figure}[htbp]
\centering
\capstart

\noindent\sphinxincludegraphics[height=500\sphinxpxdimen]{{600_rawt_plt}.png}
\caption{Same as \hyperref[\detokenize{5_section:figure5-18}]{Fig.\@ \ref{\detokenize{5_section:figure5-18}}}, but for the 600 kHz ADCP.}\label{\detokenize{5_section:figure5-19}}\end{figure}


\begin{savenotes}\sphinxattablestart
\centering
\sphinxcapstartof{table}
\sphinxthecaptionisattop
\sphinxcaption{ADCP record times (UTC mm/dd/yyyy, hh:mm:ss) during WHOTS\sphinxhyphen{}19 deployment}\label{\detokenize{5_section:table-21}}
\sphinxaftertopcaption
\begin{tabulary}{\linewidth}[t]{|T|T|T|}
\hline
\sphinxstyletheadfamily 
\sphinxAtStartPar
\sphinxstylestrong{Activities}
&\sphinxstyletheadfamily 
\sphinxAtStartPar
\sphinxstylestrong{300 kHz}
&\sphinxstyletheadfamily 
\sphinxAtStartPar
\sphinxstylestrong{600 kHz}
\\
\hline
\sphinxAtStartPar
\sphinxstylestrong{Raw file start}
&
\sphinxAtStartPar
10/4/2019,  00:00:00
&
\sphinxAtStartPar
10/4/2019,  00:00:00
\\
\hline
\sphinxAtStartPar
\sphinxstylestrong{Raw file end}
&
\sphinxAtStartPar
7/6/2021,  14:39:59
&
\sphinxAtStartPar
1/21/2020,  20:49:59
\\
\hline
\sphinxAtStartPar
\sphinxstylestrong{ADCP In water}
&
\sphinxAtStartPar
10/5/2019,  20:04:00
&
\sphinxAtStartPar
10/5/2019,  19:43:00
\\
\hline
\sphinxAtStartPar
\sphinxstylestrong{Anchor over}
&
\sphinxAtStartPar
10/06/2019,  02:12:00
&
\sphinxAtStartPar
10/06/2019,  02:12:00
\\
\hline
\sphinxAtStartPar
\sphinxstylestrong{Anchor release fired}
&
\sphinxAtStartPar
08/28/2021,  17:52:00
&
\sphinxAtStartPar
08/28/2021,  17:52:00
\\
\hline
\sphinxAtStartPar
\sphinxstylestrong{ADCP on deck}
&
\sphinxAtStartPar
08/29/2021,  01:44:00
&
\sphinxAtStartPar
8/29/2021,  02:12:00
\\
\hline
\end{tabulary}
\par
\sphinxattableend\end{savenotes}


\subsubsection{ADCP Clock Drift}
\label{\detokenize{5_section:adcp-clock-drift}}
\sphinxAtStartPar
Upon recovery, a spike is normally produced in the ADCP data by gently rubbing
each instrument’s transducer by hand for 20 seconds (see \hyperref[\detokenize{3_section:table-10}]{Table \ref{\detokenize{3_section:table-10}}}) to
compare the ADCP clocks with the ship’s time server. Unfortunately, the clock
on both ADCPs could not be evaluated because the instrument stopped working
before recovery. Past deployments of the ADCP’s suggest a 3\sphinxhyphen{}minute difference
is not unusual. No drift corrections were made. However, this drift may be
significant if the data are used for time\sphinxhyphen{}dependent analysis, such as tidal or
spectrum analysis. A drift correction needs to be applied in those cases.


\subsubsection{Heading Bias}
\label{\detokenize{5_section:heading-bias}}
\sphinxAtStartPar
As mentioned in the ADCP configuration section, the data were recorded in the
earth coordinates. A heading bias, the angle between magnetic north and true
north, can be included in the setup to obtain output data in true\sphinxhyphen{}earth
coordinates. Magnetic variation was obtained from the
\sphinxhref{https://www.ngdc.noaa.gov/geomag/calculators/magcalc.shtml\#declination}{National Geophysical Data Center ‘Geomag’ calculator}
. A constant value is acceptable for a yearlong deployment because the change
in declination is small, approximately \sphinxhyphen{}0.02°\(year^{-1}\) at the
WHOTS
location. A heading bias of 9.54° was entered in the setup of the WHOTS\sphinxhyphen{}19
ADCP’s.


\subsubsection{Speed of sound}
\label{\detokenize{5_section:speed-of-sound}}
\sphinxAtStartPar
Due to the constant proportionality between the Doppler shift and water speed,
the speed of sound needs only be measured at the transducer head
{[}\hyperlink{cite.references:id3}{Firing, 1991}{]}. The sound speed used by the ADCP is calculated using a
constant value of salinity (35) and the temperature recorded by the transducer
temperature sensor of the ADCP. Using CTD profiles close to the mooring during
HOT cruises, HOT\sphinxhyphen{}316 to 332, and from the WHOTS deployment/recovery cruises,
the mean salinity at 125 dbar was 34.95 while the mean salinity at 47.5 dbar
was 34.87. The mean ADCP temperature at 125 dbar was 21.38 °C and 25.86 °C at
47.5dbar
(\hyperref[\detokenize{5_section:figure5-18}]{Fig.\@ \ref{\detokenize{5_section:figure5-18}}}, \hyperref[\detokenize{5_section:figure5-19}]{Fig.\@ \ref{\detokenize{5_section:figure5-19}}}, and \hyperref[\detokenize{5_section:figure5-20}]{Fig.\@ \ref{\detokenize{5_section:figure5-20}}}).The mean
sound velocity at 47.5 and 125 dbar was \(1537.04 ms^{-1}\) and
\(1527.12 ms^{-1}\), respectively.

\begin{figure}[htbp]
\centering
\capstart

\noindent\sphinxincludegraphics[height=800\sphinxpxdimen]{{figures/adcp_moored/wh19_CTD_sv_profile}.png}
\caption{Sound speed profile (top panel) during the deployment of the WHOTS\sphinxhyphen{}19 mooring
from 2 dbar CTD data taken during regular HOT cruises and CTD profiles taken
during the WHOTS\sphinxhyphen{}19 and \sphinxhyphen{}20) deployment cruises (individual casts marked with a
red diamond). The bottom left panels show the sound velocity at a depth of the
ADCP’s (47.5 m and 125 m), with the mean sound velocity indicated with a
dashed black line. The lower right panels show the temperature and salinity
at each ADCP depth for the time series, with the mean temperatures
indicated with blue lines and mean salinity indicated with red lines.}\label{\detokenize{5_section:figure5-20}}\end{figure}


\subsubsection{Quality Control}
\label{\detokenize{5_section:quality-control}}
\sphinxAtStartPar
Quality control of the ADCP data involved the thorough examination of the
velocity, instrument orientation, and diagnostic fields to develop the basis of
the QC flagging procedures. Details of the methods used can be found in the
WHOTS Data Report 1 {[}\hyperlink{cite.references:id15}{Santiago\sphinxhyphen{}Mandujano \sphinxstyleemphasis{et al.}, 2007}{]}. The following QC
procedures were applied to the WHOTS\sphinxhyphen{}19 deployment of ADCP data.
\begin{enumerate}
\sphinxsetlistlabels{\arabic}{enumi}{enumii}{}{.}%
\item {} 
\sphinxAtStartPar
The first bin (closest to the transducer) is sometimes corrupted due to what
is known as ringing. A period of time is needed for the sound energy
produced during a transducer’s transmit pulse to dissipate before the ADCP
can adequately receive the returned echoes. This “blanking interval” is used
to prevent useless data from being recorded. If it is too short, signal
returns can be contaminated by the lingering noise from the transducer. The
blanking interval is expressed as a distance. The default value of 1.76 m
was used for the 300 kHz ADCP, whereas an interval of 0.88 m was used for
the 600 kHz ADCP. As a result, bin one was flagged and replaced with \sphinxcode{\sphinxupquote{ Not a Number (NaN)}} in the quality\sphinxhyphen{}controlled dataset (\hyperref[\detokenize{5_section:figure5-21}]{Fig.\@ \ref{\detokenize{5_section:figure5-21}}}).

\begin{figure}[htbp]
\centering
\capstart

\noindent\sphinxincludegraphics[height=600\sphinxpxdimen]{{wh19_ringing}.png}
\caption{Eastward velocity component for the 300 kHz (top panel) and the 600 kHz (bottom
panel) ADCPs are showing the incoherence between depth bins 1 (red), 2 (green),
and 3 (blue).}\label{\detokenize{5_section:figure5-21}}\end{figure}

\item {} 
\sphinxAtStartPar
For an upward\sphinxhyphen{}looking ADCP with a beam angle of 20° within range of the sea
surface, the upper 6\% of the depth range is contaminated with sidelobe
interference {[}\hyperlink{cite.references:id27}{Teledyne RD Instruments, 2011}{]}. This contamination results
from the much stronger signal reflection from the sea surface than from
scatters, overwhelming the sidelobe suppression of the transducer. Data
quality is quantified using echo intensity, a measure of the backscattered
echo’s strength for each depth cell. With distance from the transducer
sensor, echo intensity is expected to decrease. Sharp increases in echo
intensity indicate contamination from surface reflection. Most of the data
within the upper four bins (\textasciitilde{}14\% of the vertical range) were flagged. These
top four bins range from about 15 m up to the sea surface.

\item {} 
\sphinxAtStartPar
The Janus configuration of four beams (along with instrument orientation)
is used to resolve currents into their component earth\sphinxhyphen{}referenced
velocities, providing a second estimate of the vertical velocity. The scaled
difference between these estimates is defined as the error velocity, and it
is useful for assessing data quality. Error velocities with an absolute
magnitude more significant than \(0.15 ms^{-1}\) (value comparable to
the standard deviation of observed horizontal velocities) were flagged and
removed.

\item {} 
\sphinxAtStartPar
An indication of data quality for each ensemble is given by the “percent
good” data indicator, which accompanies each beam for each bin. The use of
the percent good indicator is determined by the coordinate transformation
mode used during the data collection. For profiles transformed into earth
coordinates, the percent good field shows the percentage of pings that could
be used to create the earth coordinate velocities. The percent good fields
show the percentage of data made using 4 and 3 beam solutions in each depth
cell within an ensemble and the percentage that was rejected due to failing
one of the criteria set during the instrument setup (see
{\hyperref[\detokenize{appendices:whots-19-300-khz-serial-4891}]{\sphinxcrossref{\DUrole{std,std-ref}{WHOTS\sphinxhyphen{}19 300 kHz \sphinxhyphen{} Serial 4891}}}}). Data were flagged when
data in each depth cell within an ensemble made from 3 or 4 beam solutions
was 20\% or less.

\item {} 
\sphinxAtStartPar
Data were rejected using correlation magnitude, which is the pulse\sphinxhyphen{}to\sphinxhyphen{}pulse
correlation (in ping returns) for each depth cell. Correlation magnitude
represents how the shape of the received signal corresponds to the outgoing
signal for each ping. If at least three of the beams exhibited a correlation
magnitude more significant than 64 counts for a given bin, the profile could
be transformed into earth coordinates. Low correlation magnitudes may
indicate sudden changes in particle density or sudden changes in ADCP tilt.
More research is needed at this time into relationships between ADCP tilt
and correlation magnitude. If any beam had a correlation magnitude of 20
counts or less, that data point was flagged.

\item {} 
\sphinxAtStartPar
Histograms of raw vertical velocity data and partially cleaned data from the
ADCP (\hyperref[\detokenize{5_section:figure5-22}]{Fig.\@ \ref{\detokenize{5_section:figure5-22}}} and \hyperref[\detokenize{5_section:figure5-23}]{Fig.\@ \ref{\detokenize{5_section:figure5-23}}}) and the WHOTS Data
Report 1 {[}\hyperlink{cite.references:id15}{Santiago\sphinxhyphen{}Mandujano \sphinxstyleemphasis{et al.}, 2007}{]} showed vertical velocities larger
than expected, some exceeding \(1 m s^{-1}\). Recall that the
instruments’ burst sampling (4\sphinxhyphen{}second intervals for the 300 kHz and 2\sphinxhyphen{}second
intervals for the 600 kHz, for 160 seconds every 10 minutes) was designed to
minimize aliasing by occasional large ocean swell orbital motions
{\hyperref[\detokenize{3_section:description-of-whots-19-mooring}]{\sphinxcrossref{\DUrole{std,std-ref}{Description of WHOTS\sphinxhyphen{}19 Mooring}}}} , and therefore are not
the source of these speeds in the data. These significant vertical speeds
are possibly fish swimming in the beams based on the histograms of the
partially cleaned data; depth cells with an absolute value of vertical
velocity greater than \(0.3 ms^{-1}\) were flagged.

\begin{figure}[htbp]
\centering
\capstart

\noindent\sphinxincludegraphics[height=600\sphinxpxdimen]{{wh19_300_vv_hist}.png}
\caption{Histogram of the vertical velocity of the 300 kHz ADCP for raw data (top panel)
and enlarged for clarity (upper middle panel), and partial quality controlled
data (lower middle panel) and enlarged for clarity (bottom).}\label{\detokenize{5_section:figure5-22}}\end{figure}

\begin{figure}[htbp]
\centering
\capstart

\noindent\sphinxincludegraphics[height=600\sphinxpxdimen]{{wh19_600_vv_hist}.png}
\caption{Same as \hyperref[\detokenize{5_section:figure5-22}]{Fig.\@ \ref{\detokenize{5_section:figure5-22}}}, but for 600kHz ADCP.}\label{\detokenize{5_section:figure5-23}}\end{figure}

\item {} 
\sphinxAtStartPar
A quality control routine known as ‘edgers’ identifies outliers in surface
bins using a five\sphinxhyphen{}point median differencing method. The median velocity from
surface bins was calculated for each ensemble, and then a five\sphinxhyphen{}point running
median of the surface bin median was calculated. This last median was then
compared to individual velocity observations in the surface bins, and those
differing by greater than \(0.48 ms^{-1}\) were flagged.

\item {} 
\sphinxAtStartPar
A 5\sphinxhyphen{}pole low pass Butterworth filter with a cutoff frequency of
\(0.25 \frac{cycles}{hour}\) was used upon the time\sphinxhyphen{}series’ length to
isolate low\sphinxhyphen{}frequency flow for each bin independently. The low\sphinxhyphen{}frequency
flow is then subtracted, giving a time series of high\sphinxhyphen{}frequency velocity
component fluctuations for each bin. Data points were considered outliers
when their values exceeded four standard deviations from the mean (for each
bin) and were removed.

\item {} 
\sphinxAtStartPar
A median residual filter used a 7\sphinxhyphen{}point (70 minutes) median differencing
method to define velocity fluctuations. A 7\sphinxhyphen{}point running median is
calculated for each bin independently, and the result is subtracted out,
giving time series of variations relative to the running median. Outliers
higher than four standard deviations from the mean of the 7 points are
flagged and removed for each bin.

\item {} 
\sphinxAtStartPar
Meticulous verification of all the quality control routines was performed
through visual inspections of the quality\sphinxhyphen{}controlled velocity data. Two
methods were utilized; time\sphinxhyphen{}series of u and v components for multiple bins
were evaluated, and individual vertical profiles. The time\sphinxhyphen{}series
methodology involved inspecting u and v components separately, five bins at
a time, over 600 ensembles (100 hours). Any instance showing one bin
behaving erratically from the other four bins was investigated further. If
it seemed that there could be no reasonable rationale for the erratic
points from the identified bin, the points were flagged. The intent of the
inspection of vertical profiles of u and v components was to find entire
profiles that were not aligned with neighboring profiles. Thirty u and v
profiles were stacked at a time and were visually inspected for any
anomalous data.

\end{enumerate}


\section{Vector Measuring Current Meter (VMCM)}
\label{\detokenize{5_section:vector-measuring-current-meter-vmcm}}
\sphinxAtStartPar
Vector measuring current meters (VMCM) were deployed on the WHOTS\sphinxhyphen{}19 mooring at
depths of 10 m and 30 m, serial numbers SN 2042 and 2032, respectively. VMCM
data were processed by the WHOI/UOP group, and the record times are shown
in \hyperref[\detokenize{5_section:table-22}]{Table \ref{\detokenize{5_section:table-22}}}.


\begin{savenotes}\sphinxattablestart
\centering
\sphinxcapstartof{table}
\sphinxthecaptionisattop
\sphinxcaption{Record times (UTC mm/dd/yy hh:mm) for the VMCMs at 10 m and 30 m during the WHOTS\sphinxhyphen{}19 deployment}\label{\detokenize{5_section:table-22}}
\sphinxaftertopcaption
\begin{tabulary}{\linewidth}[t]{|T|T|T|}
\hline
\sphinxstyletheadfamily 
\sphinxAtStartPar
\sphinxstylestrong{Time Over}
&\sphinxstyletheadfamily 
\sphinxAtStartPar
\sphinxstylestrong{VMCM (SN 2042)}
&\sphinxstyletheadfamily 
\sphinxAtStartPar
\sphinxstylestrong{VMCM (SN 2032)}
\\
\hline
\sphinxAtStartPar
\sphinxstylestrong{Deployment}
&
\sphinxAtStartPar
10/5/19 19:11
&
\sphinxAtStartPar
10/5/19 18:51
\\
\hline
\sphinxAtStartPar
\sphinxstylestrong{Recovery}
&
\sphinxAtStartPar
8/29/21 3:17
&
\sphinxAtStartPar
8/29/21 3:27
\\
\hline
\end{tabulary}
\par
\sphinxattableend\end{savenotes}

\sphinxAtStartPar
Daily (24 hours) moving averages of quality controlled 600 kHz ADCP data are
compared to VMCM data interpolated to the ADCP ensemble times in the top panels
of \hyperref[\detokenize{5_section:figure5-24}]{Fig.\@ \ref{\detokenize{5_section:figure5-24}}} through \hyperref[\detokenize{5_section:figure5-27}]{Fig.\@ \ref{\detokenize{5_section:figure5-27}}}, and the difference is
shown in the middle panels. The absolute value of the mean difference plus or
minus one standard deviation is shown at the top of the middle panel.
Velocities are not compared if greater than 80\% of the ADCP data within a
24\sphinxhyphen{}hour average was flagged. The absolute value of mean differences for all
deployments and both velocity components varied between 2 and 3.5
\(cm s^{-1}\), with standard deviations between 1.8 and 2.7
\(cm s^{-1}\). The VMCM data does not appear to degrade over time for any
deployment. Propeller fouling would dampen measured VMCM velocity magnitudes,
but a decrease in VMCM velocity magnitude than ADCP velocity magnitude with
time is not observed.

\begin{figure}[htbp]
\centering
\capstart

\noindent\sphinxincludegraphics[height=1000\sphinxpxdimen]{{wh19_NGVM_30_U}.png}
\caption{A comparison of 30 m VMCM and ADCP U velocity for WHOTS\sphinxhyphen{}19. The top panel shows
24\sphinxhyphen{}hour moving averages of VMCM zonal (U) velocity at 30 m depth (red) and ADCP
U velocity from the nearest depth bin to 30 m (30.22 m). The middle panel shows
the U velocity difference, and the bottom panel shows the percentage of ADCP
data within the moving average not flagged by quality control methods.}\label{\detokenize{5_section:figure5-24}}\end{figure}

\begin{figure}[htbp]
\centering
\capstart

\noindent\sphinxincludegraphics[height=1000\sphinxpxdimen]{{wh19_NGVM_30_V}.png}
\caption{Same as in \hyperref[\detokenize{5_section:figure5-24}]{Fig.\@ \ref{\detokenize{5_section:figure5-24}}} but for the meridional (V) velocity component.}\label{\detokenize{5_section:figure5-25}}\end{figure}

\begin{figure}[htbp]
\centering
\capstart

\noindent\sphinxincludegraphics[height=1000\sphinxpxdimen]{{wh19_NGVM_10_U}.png}
\caption{Same as in \hyperref[\detokenize{5_section:figure5-24}]{Fig.\@ \ref{\detokenize{5_section:figure5-24}}} but for the 10 m VMCM.}\label{\detokenize{5_section:figure5-26}}\end{figure}

\begin{figure}[htbp]
\centering
\capstart

\noindent\sphinxincludegraphics[height=1000\sphinxpxdimen]{{wh19_NGVM_10_V}.png}
\caption{Same as \hyperref[\detokenize{5_section:figure5-26}]{Fig.\@ \ref{\detokenize{5_section:figure5-26}}}, but for the meridional (V) velocity component.}\label{\detokenize{5_section:figure5-27}}\end{figure}


\section{Global Positioning System Receiver}
\label{\detokenize{5_section:global-positioning-system-receiver}}
\sphinxAtStartPar
Xeos Global Positioning System receiver (Melo\sphinxhyphen{}\sphinxcode{\sphinxupquote{IMEI:300034012129060}}) and
(Rover\sphinxhyphen{}\sphinxcode{\sphinxupquote{IMEI:300434063359170}}) were attached to the buoy’s tower
top during the WHOTS\sphinxhyphen{}19 deployment
({\hyperref[\detokenize{3_section:description-of-whots-19-mooring}]{\sphinxcrossref{\DUrole{std,std-ref}{Description of WHOTS\sphinxhyphen{}19 Mooring}}}}). Data returns from the
receiver were high (\hyperref[\detokenize{5_section:table-23}]{Table \ref{\detokenize{5_section:table-23}}}). There was no ARGOS receiver for
WHOTS\sphinxhyphen{}19.


\begin{savenotes}\sphinxattablestart
\centering
\sphinxcapstartof{table}
\sphinxthecaptionisattop
\sphinxcaption{GPS record times (UTC mm/dd/yy hh:mm) during WHOTS\sphinxhyphen{}19}\label{\detokenize{5_section:table-23}}
\sphinxaftertopcaption
\begin{tabulary}{\linewidth}[t]{|T|T|T|}
\hline
\sphinxstyletheadfamily 
\sphinxAtStartPar
\sphinxstylestrong{Raw file}
&\sphinxstyletheadfamily 
\sphinxAtStartPar
\sphinxstylestrong{Xeos GPS (Melo)}
&\sphinxstyletheadfamily 
\sphinxAtStartPar
\sphinxstylestrong{Xeos GPS (Rover)}
\\
\hline
\sphinxAtStartPar
\sphinxstylestrong{Start Time}
&
\sphinxAtStartPar
10/6/19 03:07
&
\sphinxAtStartPar
7/30/19 21:01
\\
\hline
\sphinxAtStartPar
\sphinxstylestrong{End Time}
&
\sphinxAtStartPar
3/28/20 04:43
&
\sphinxAtStartPar
8/30/21 12:00
\\
\hline
\end{tabulary}
\par
\sphinxattableend\end{savenotes}


\chapter{Results}
\label{\detokenize{6_section:results}}\label{\detokenize{6_section::doc}}
\sphinxAtStartPar
During the WHOTS\sphinxhyphen{}19 cruise (WHOTS\sphinxhyphen{}19 mooring deployment, 17 June 2023), a
weakening front north of the Hawaiian Islands acted to weaken trades across the
region. This ridge to the north was eroding due to a weakening cold front north
of the ridge, which led to a downward trend in winds. The front continued to
weaken and sink southward through the weekend, keeping the winds light and
variable. Regular trades returned by mid\sphinxhyphen{}week. On October 10, winds
in the morning increased to 25 kt gusting to 35 kt, and there was a brief rain
event in the morning. Waves during the week were also low, with a small swell
from the north arriving mid\sphinxhyphen{}week.

\sphinxAtStartPar
Near\sphinxhyphen{}surface currents were almost 1 kt northward during transit to Station
ALOHA, turning southward upon arrival to Station ALOHA. Eventually, the current
shifted due north again, became somewhat weaker (about 0.5 kt), and remained
for approximately five days. There was a nearly low sea level just to the east
of Station ALOHA, not quite a fully formed eddy; it was reflected by northward
flow to the east and southward flow to the west. A combination of internal
semi\sphinxhyphen{}diurnal and diurnal tides and near\sphinxhyphen{}inertial oscillations was noticeable,
especially in vertical shear.

\sphinxAtStartPar
Conditions during the WHOTS\sphinxhyphen{}19 deployment on October 5 were favorable, with
light NNE winds of \textasciitilde{} 5 kt increasing to up to 10 kt by the end of the
deployment. There were clear skies and no precipitation in the region, and 1.0
to 1.5 m waves from the east, with a strong surface current towards ENE.

\sphinxAtStartPar
CTD casts conducted near the WHOTS\sphinxhyphen{}19 buoy (Station 52) after deployment
(\hyperref[\detokenize{6_section:figure6-4}]{Fig.\@ \ref{\detokenize{6_section:figure6-4}}} and \hyperref[\detokenize{6_section:figure6-5}]{Fig.\@ \ref{\detokenize{6_section:figure6-5}}}) displayed a subsurface
salinity maximum between 70 and 80 dbar and a mixed layer 40 to 60 dbar deep.

\sphinxAtStartPar
During the WHOTS\sphinxhyphen{}20 cruise (WHOTS\sphinxhyphen{}19 mooring recovery, 02 June 2024), a high\sphinxhyphen{}pressure ridge far north of the Hawaiian Islands maintained a
tight enough pressure gradient down across the region to produce moderate to
locally strong trades. As this high slowly moved northeast away from the area
and subtly weakened the gradient, trades gradually weakened. There was no
measurable precipitation during the mooring recovery, and surface
currents were less than 1 kt.

\sphinxAtStartPar
CTD casts conducted near the WHOTS\sphinxhyphen{}19 buoy (Station 52) before recovery
(\hyperref[\detokenize{6_section:figure6-10}]{Fig.\@ \ref{\detokenize{6_section:figure6-10}}}) displayed a subsurface salinity maximum between 150 and
170 dbar and a mixed layer 60 dbar deep.

\sphinxAtStartPar
The temperature MicroCAT records during the WHOTS\sphinxhyphen{}19 deployment
(\hyperref[\detokenize{6_section:wh19-temp-1-4-png}]{Fig.\@ \ref{\detokenize{6_section:wh19-temp-1-4-png}}} through \hyperref[\detokenize{6_section:wh19-temp-21-png}]{Fig.\@ \ref{\detokenize{6_section:wh19-temp-21-png}}}) show
noticeable seasonal variability in the upper 100 m. A temperature decrease in
October\sphinxhyphen{}November 2019 was evident in the instruments below 65 m. The salinity
records (\hyperref[\detokenize{6_section:wh19-sali-1-4-png}]{Fig.\@ \ref{\detokenize{6_section:wh19-sali-1-4-png}}} through \hyperref[\detokenize{6_section:wh19-sali-21-png}]{Fig.\@ \ref{\detokenize{6_section:wh19-sali-21-png}}}) do
not show an apparent seasonal cycle, but a salinity increase was recorded
during October\sphinxhyphen{}November 2019, by the instruments between 40 and 85 m,
coinciding with the temperature decrease. This increase was followed by a
period of low salinity (less than 35 on average above 120 m) throughout
2020\sphinxhyphen{}2021, with extreme values (nearly 34.4) above 120 m in November\sphinxhyphen{}December
2019, and above 75 m in July\sphinxhyphen{}September 2020.

\sphinxAtStartPar
\hyperref[\detokenize{6_section:w1-19-contts-png}]{Fig.\@ \ref{\detokenize{6_section:w1-19-contts-png}}} through \hyperref[\detokenize{6_section:w1-19-cont-s-png}]{Fig.\@ \ref{\detokenize{6_section:w1-19-cont-s-png}}} show contours of
the WHOTS\sphinxhyphen{}19 MicroCAT data in context with data from the previous 15
deployments. The seasonal cycle is evident in the temperature record, with
record temperatures (higher than 26°C) in the summer of 2004, and again in
2014, 2015, 2017, 2019, and 2020. Salinities in the subsurface salinity maximum
were relatively low during the first 6 years of the record, only to increase
drastically after 2008 through 2015, with some lower salinity episodes in
mid\sphinxhyphen{}2011 and early 2012. The salinity maximum extended to near the surface in
early 2010, 2011, late2012\sphinxhyphen{}early 2013, and February\sphinxhyphen{}March 2013. Salinities in
the salinity minimum decreased after 2015, showing low salinities above 100 m
in 2016, 2017, 2018, and reaching record low values (34.4) in July\sphinxhyphen{}August 2019
and July\sphinxhyphen{}September 2020. When plotted in \(\sigma\theta\) coordinates
(\hyperref[\detokenize{6_section:w1-19-cont-s-png}]{Fig.\@ \ref{\detokenize{6_section:w1-19-cont-s-png}}}), the salinity maximum seems to be centered roughly
between 24 and 24.5 \(\sigma\theta\).

\sphinxAtStartPar
Records from the WHOTS\sphinxhyphen{}19 MicroCATs (\hyperref[\detokenize{6_section:plt-w19-aco-all-png}]{Fig.\@ \ref{\detokenize{6_section:plt-w19-aco-all-png}}}) deployed
near the bottom of the mooring (4713 m) detected temperature and salinity
changes related to episodic ‘cold events’ apparently caused by bottom water
moving between abyssal basins {[}\hyperlink{cite.references:id12}{Lukas \sphinxstyleemphasis{et al.}, 2001}{]}. These events are being
monitored by instruments at the ALOHA Cabled Observatory (ACO) {[}\hyperlink{cite.references:id10}{Howe \sphinxstyleemphasis{et al.}, 2011}{]}
, a deep water observatory located at the bottom of Station ALOHA (about 6
nautical miles north from the WHOTS\sphinxhyphen{}19 anchor), since June 2011.
\hyperref[\detokenize{6_section:plt-w19-aco-all-png}]{Fig.\@ \ref{\detokenize{6_section:plt-w19-aco-all-png}}} shows temperature and salinity records from
the WHOTS\sphinxhyphen{}19 MicroCATs superimposed on the ACO data. The MicroCAT data
agreed with the temperature decrease and the salinity variability
registered by ACO instruments during cold events in January, March and
December 2020, and a minor events in August 2020 and September 2021.

\sphinxAtStartPar
\hyperref[\detokenize{6_section:whots-19-u-subplot-png}]{Fig.\@ \ref{\detokenize{6_section:whots-19-u-subplot-png}}} through \hyperref[\detokenize{6_section:whots-19-w-subplot-png}]{Fig.\@ \ref{\detokenize{6_section:whots-19-w-subplot-png}}} shows
the time series of the zonal, meridional, and vertical currents recorded with
the moored ADCPs during the WHOTS\sphinxhyphen{}19 deployment.
\hyperref[\detokenize{6_section:wh1-19-adcp-uv-cont-png}]{Fig.\@ \ref{\detokenize{6_section:wh1-19-adcp-uv-cont-png}}}, through \hyperref[\detokenize{6_section:wh1-19-adcp-w-cont-png}]{Fig.\@ \ref{\detokenize{6_section:wh1-19-adcp-w-cont-png}}},
shows the ADCP current components’ contours in context with data from the
previous deployments. Despite the gaps in the data, an apparent variability is
seen in the zonal and meridional currents, apparently caused by passing eddies.
There have been periods of intermittent positive or negative zonal currents on
top of this variability, for instance, during 2007\sphinxhyphen{}2008. The contours of the
vertical current component \hyperref[\detokenize{6_section:wh1-19-adcp-w-cont-png}]{Fig.\@ \ref{\detokenize{6_section:wh1-19-adcp-w-cont-png}}} show a transition
in the magnitude of the contours near 47 m, indicating that the 300 kHz ADCP
located at 126 m moves more vertically than the 600 kHz ADCP located at 47.5 m.

\sphinxAtStartPar
A comparison between the moored ADCP data and the shipboard ADCP data obtained
during the WHOTS\sphinxhyphen{}19 cruise is shown in
\hyperref[\detokenize{6_section:whots19deploy-adcp-contour1-png}]{Fig.\@ \ref{\detokenize{6_section:whots19deploy-adcp-contour1-png}}}, and
\hyperref[\detokenize{6_section:whots19deploy-adcp-contour2-png}]{Fig.\@ \ref{\detokenize{6_section:whots19deploy-adcp-contour2-png}}}, and a similar comparison during the
WHOTS\sphinxhyphen{}20 cruise is shown in \hyperref[\detokenize{6_section:whots19recover-adcp-contour1-png}]{Fig.\@ \ref{\detokenize{6_section:whots19recover-adcp-contour1-png}}} and
\hyperref[\detokenize{6_section:whots19recover-adcp-contour2-png}]{Fig.\@ \ref{\detokenize{6_section:whots19recover-adcp-contour2-png}}}. Some differences were seen,
especially in the zonal component, maybe due to the mooring motion, which was
not removed from the data. Comparisons between the available shipboard ADCP
from HOT\sphinxhyphen{}316 to \sphinxhyphen{}332 cruises and the mooring data are shown in
\hyperref[\detokenize{6_section:wh19-moor-ship-adcp-comp-1-png}]{Fig.\@ \ref{\detokenize{6_section:wh19-moor-ship-adcp-comp-1-png}}} through
\hyperref[\detokenize{6_section:wh19-moor-ship-adcp-comp-4-png}]{Fig.\@ \ref{\detokenize{6_section:wh19-moor-ship-adcp-comp-4-png}}}.

\sphinxAtStartPar
The Xeos\sphinxhyphen{}GPS receiver registered the WHOTS\sphinxhyphen{}19 buoy motion, and its positions
are plotted in \hyperref[\detokenize{6_section:wh19xeos-pos-png}]{Fig.\@ \ref{\detokenize{6_section:wh19xeos-pos-png}}}. The buoy was located west of the
anchor for most of the deployment, except from around June to November 2020,
when it was east. The power spectrum of these data (\hyperref[\detokenize{6_section:wh19xeos-pos-png}]{Fig.\@ \ref{\detokenize{6_section:wh19xeos-pos-png}}})
shows extra energy at the inertial period (\textasciitilde{}31 hr). Combining the buoy motion
with the tilt (a combination of pitch and roll) from the ADCP data
(\hyperref[\detokenize{6_section:wh19-adcp-tilt-png}]{Fig.\@ \ref{\detokenize{6_section:wh19-adcp-tilt-png}}}) showed that the tilt increased as the buoy
distance from the anchor WHOTS\sphinxhyphen{}19 increased. This was expected since the
inclination of the cable increases as the buoy moves away from the anchor.


\section{CTD Profiling Data}
\label{\detokenize{6_section:ctd-profiling-data}}
\sphinxAtStartPar
Profiles of temperature, salinity, and potential density (\(\sigma\theta\))
from the casts obtained during the WHOTS\sphinxhyphen{}19 deployment cruise are presented in
\hyperref[\detokenize{6_section:figure6-1}]{Fig.\@ \ref{\detokenize{6_section:figure6-1}}} through \hyperref[\detokenize{6_section:figure6-5}]{Fig.\@ \ref{\detokenize{6_section:figure6-5}}}, together with the results of
bottle determination of salinity. \hyperref[\detokenize{6_section:figure6-6}]{Fig.\@ \ref{\detokenize{6_section:figure6-6}}} through
\hyperref[\detokenize{6_section:figure6-10}]{Fig.\@ \ref{\detokenize{6_section:figure6-10}}} shows the results of the CTD profiles during the WHOTS\sphinxhyphen{}20
cruise.

\begin{figure}[htbp]
\centering
\capstart

\noindent\sphinxincludegraphics[height=800\sphinxpxdimen]{{figures/ctd/1.whots_19/s20c1_s50c1}.png}
\caption{{[}Upper left panel{]} Profiles of CTD temperature, salinity, and potential
density (\(\sigma\theta\)) as a function of pressure, including discrete
bottle salinity samples (when available) for station 20 cast 1 during the
WHOTS\sphinxhyphen{}19 cruise. {[}Upper right panel{]} Profiles of CTD salinity as a function
of potential temperature, including discrete bottle salinity samples (when
available) for station 20 cast 1 during the WHOTS\sphinxhyphen{}19 cruise. {[}Lower left
panel{]} Same as in the upper left panel, but for station 50 cast 1. {[}Lower
right panel{]} Same as in the upper right panel, but station 50 cast 1.}\label{\detokenize{6_section:figure6-1}}\end{figure}

\begin{figure}[htbp]
\centering
\capstart

\noindent\sphinxincludegraphics[height=1000\sphinxpxdimen]{{figures/ctd/1.whots_19/s50c2_s50c3}.png}
\caption{{[}Upper panels{]} Same as in \hyperref[\detokenize{6_section:figure6-1}]{Fig.\@ \ref{\detokenize{6_section:figure6-1}}}, but for station 50, cast
2. {[}Lower panels{]} Same as \hyperref[\detokenize{6_section:figure6-1}]{Fig.\@ \ref{\detokenize{6_section:figure6-1}}}, but for station 50, cast 3.}\label{\detokenize{6_section:figure6-2}}\end{figure}

\begin{figure}[htbp]
\centering
\capstart

\noindent\sphinxincludegraphics[height=1000\sphinxpxdimen]{{figures/ctd/1.whots_19/s50c4_s50c5}.png}
\caption{{[}Upper panels{]} Same as in \hyperref[\detokenize{6_section:figure6-1}]{Fig.\@ \ref{\detokenize{6_section:figure6-1}}}, but for station 50, cast 4.
{[}Lower panels{]} Same as in \hyperref[\detokenize{6_section:figure6-1}]{Fig.\@ \ref{\detokenize{6_section:figure6-1}}}, but for station 50 cast 5.}\label{\detokenize{6_section:figure6-3}}\end{figure}

\begin{figure}[htbp]
\centering
\capstart

\noindent\sphinxincludegraphics[height=1000\sphinxpxdimen]{{figures/ctd/1.whots_19/s52c1_s52c2}.png}
\caption{{[}Upper panels{]} Same as in \hyperref[\detokenize{6_section:figure6-1}]{Fig.\@ \ref{\detokenize{6_section:figure6-1}}}, but for station 52, cast 1. {[}Lower
panels{]} Same as in \hyperref[\detokenize{6_section:figure6-1}]{Fig.\@ \ref{\detokenize{6_section:figure6-1}}}, but for station 52, cast 2.}\label{\detokenize{6_section:figure6-4}}\end{figure}

\begin{figure}[htbp]
\centering
\capstart

\noindent\sphinxincludegraphics[height=1000\sphinxpxdimen]{{figures/ctd/1.whots_19/s52c3_s52c4}.png}
\caption{Upper panels{]} Same as in \hyperref[\detokenize{6_section:figure6-1}]{Fig.\@ \ref{\detokenize{6_section:figure6-1}}}, but for station 52, cast 3.{[}Lower
panels{]} Same as in \hyperref[\detokenize{6_section:figure6-1}]{Fig.\@ \ref{\detokenize{6_section:figure6-1}}}, but for station 52, cast 4.}\label{\detokenize{6_section:figure6-5}}\end{figure}

\begin{figure}[htbp]
\centering
\capstart

\noindent\sphinxincludegraphics[height=800\sphinxpxdimen]{{figures/ctd/2.whots_20/s2c1_s20c1}.png}
\caption{{[}Upper left panel{]} Profiles of CTD temperature, salinity, and potential
density (\(\sigma\theta\)) as a function of pressure, including discrete
bottle salinity samples (when available) for station 2 cast 1 during the
WHOTS\sphinxhyphen{}20 cruise. {[}Upper right panel{]} Profiles of CTD salinity as a function
of potential temperature, including discrete bottle salinity samples (when
available) for station 2 cast 1 during the WHOTS\sphinxhyphen{}20 cruise. {[}Lower left
panel{]} Same as in the upper left panel, but for station 20 cast 1. {[}Lower
right panel{]} Same as in the upper right panel, but station 20 cast 1.}\label{\detokenize{6_section:figure6-6}}\end{figure}

\begin{figure}[htbp]
\centering
\capstart

\noindent\sphinxincludegraphics[height=1000\sphinxpxdimen]{{figures/ctd/2.whots_20/s50c1_s50c2}.png}
\caption{Upper panels{]} Same as in \hyperref[\detokenize{6_section:figure6-6}]{Fig.\@ \ref{\detokenize{6_section:figure6-6}}}, but for station 50, cast
1.{[}Lower panels{]} Same as in \hyperref[\detokenize{6_section:figure6-6}]{Fig.\@ \ref{\detokenize{6_section:figure6-6}}}, but for station 50, cast 2.}\label{\detokenize{6_section:figure6-7}}\end{figure}

\begin{figure}[htbp]
\centering
\capstart

\noindent\sphinxincludegraphics[height=1000\sphinxpxdimen]{{figures/ctd/2.whots_20/s50c3_s50c4}.png}
\caption{Upper panels{]} Same as in \hyperref[\detokenize{6_section:figure6-6}]{Fig.\@ \ref{\detokenize{6_section:figure6-6}}}, but for station 50, cast
3.{[}Lower panels{]} Same as in \hyperref[\detokenize{6_section:figure6-6}]{Fig.\@ \ref{\detokenize{6_section:figure6-6}}}, but for station 50, cast 4.}\label{\detokenize{6_section:figure6-8}}\end{figure}

\begin{figure}[htbp]
\centering
\capstart

\noindent\sphinxincludegraphics[height=1000\sphinxpxdimen]{{figures/ctd/2.whots_20/s50c5_s50c6}.png}
\caption{Upper panels{]} Same as in \hyperref[\detokenize{6_section:figure6-6}]{Fig.\@ \ref{\detokenize{6_section:figure6-6}}}, but for station 50, cast
5.{[}Lower panels{]} Same as in \hyperref[\detokenize{6_section:figure6-6}]{Fig.\@ \ref{\detokenize{6_section:figure6-6}}}, but for station 50, cast 6.}\label{\detokenize{6_section:figure6-9}}\end{figure}

\begin{figure}[htbp]
\centering
\capstart

\noindent\sphinxincludegraphics[height=1000\sphinxpxdimen]{{s52c1_s52c2}.png}
\caption{Upper panels{]} Same as in \hyperref[\detokenize{6_section:figure6-6}]{Fig.\@ \ref{\detokenize{6_section:figure6-6}}}, but for station 52, cast 1.
{[}Lower panels{]} Same as in \hyperref[\detokenize{6_section:figure6-6}]{Fig.\@ \ref{\detokenize{6_section:figure6-6}}}, but for station 52, cast 2.}\label{\detokenize{6_section:figure6-10}}\end{figure}


\section{Thermosalinograph Data}
\label{\detokenize{6_section:thermosalinograph-data}}
\sphinxAtStartPar
Underway measurements of near\sphinxhyphen{}surface temperature and salinity from the
thermosalinograph (TSG) system onboard the R/V Oscar Sette cruise are presented
in \hyperref[\detokenize{6_section:ac39thsl-final-png}]{Fig.\@ \ref{\detokenize{6_section:ac39thsl-final-png}}} and navigational data is shown in
\hyperref[\detokenize{6_section:ac39nav-final-png}]{Fig.\@ \ref{\detokenize{6_section:ac39nav-final-png}}} for the WHOTS\sphinxhyphen{}19 cruise. The WHOTS\sphinxhyphen{}19 underway
seawater system that feeds the TSG failed on October 11, 2019, due to air
going into the plumbing, causing the pumps to stop working during
deteriorated weather conditions. TSG and navigational data during the
WHOTS\sphinxhyphen{}20 cruise, onboard the R/V Oscar Sette, are presented in \{numref\}
\sphinxcode{\sphinxupquote{ac40thsl\_final.png}} and \hyperref[\detokenize{6_section:ac40nav-final-png}]{Fig.\@ \ref{\detokenize{6_section:ac40nav-final-png}}}, respectively. The
data between August 25 and 27, 2021 are particularly bad because it was
during transit back to Oahu to disembark a crew member with medical
problems, and the flow through the system was stopped during that time.

\begin{figure}[htbp]
\centering
\capstart

\noindent\sphinxincludegraphics[height=600\sphinxpxdimen]{{figures/thermosal/ac39thsl_final}.png}
\caption{Final processed temperature (upper panel), salinity (middle panel), and
potential density (\(\sigma\theta\)) (lower panel) data from the continuous
underway system onboard the R/V Hi’ialakai during the WHOTS\sphinxhyphen{}19 cruise.
Temperature and salinity taken from 6\sphinxhyphen{}dbar CTD data (circles) and salinity
bottle sample data (crosses) are superimposed. The dashed vertical red line
indicates the period of occupation of Station ALOHA and the WHOTS site.}\label{\detokenize{6_section:ac39thsl-final-png}}\end{figure}

\begin{figure}[htbp]
\centering
\capstart

\noindent\sphinxincludegraphics[height=600\sphinxpxdimen]{{figures/thermosal/ac39nav_final}.png}
\caption{Timeseries of latitude (upper panel), longitude (middle panel), and ship’s
speed (lower panel) during the WHOTS\sphinxhyphen{}19 cruise.}\label{\detokenize{6_section:ac39nav-final-png}}\end{figure}

\begin{figure}[htbp]
\centering
\capstart

\noindent\sphinxincludegraphics[height=600\sphinxpxdimen]{{figures/thermosal/ac40thsl_final}.png}
\caption{Final processed temperature (upper panel), salinity (middle panel), and
potential density (\(\sigma\theta\)) (lower panel) data from the continuous underway system
onboard the R/V Oscar Sette during the WHOTS\sphinxhyphen{}20 cruise. Temperature and
salinity were taken from 6\sphinxhyphen{}dbar CTD data (circles), and salinity bottle sample
data (crosses) are superimposed. The dashed vertical red line indicates the
period of occupation of Station ALOHA and the WHOTS site.}\label{\detokenize{6_section:ac40thsl-final-png}}\end{figure}

\begin{figure}[htbp]
\centering
\capstart

\noindent\sphinxincludegraphics[height=600\sphinxpxdimen]{{figures/thermosal/ac40nav_final}.png}
\caption{Timeseries of latitude (upper panel), longitude (middle panel), and ship’s
speed (lower panel) during the WHOTS\sphinxhyphen{}20 cruise.}\label{\detokenize{6_section:ac40nav-final-png}}\end{figure}


\section{MicroCAT Data}
\label{\detokenize{6_section:microcat-data}}
\sphinxAtStartPar
The temperatures measured by MicroCATs during the mooring deployment for
WHOTS\sphinxhyphen{}19 are presented in \hyperref[\detokenize{6_section:wh19-temp-1-4-png}]{Fig.\@ \ref{\detokenize{6_section:wh19-temp-1-4-png}}} through
\hyperref[\detokenize{6_section:wh19-temp-21-png}]{Fig.\@ \ref{\detokenize{6_section:wh19-temp-21-png}}} for each of the depths where the instruments
were located. The salinities are plotted in \hyperref[\detokenize{6_section:wh19-sali-1-4-png}]{Fig.\@ \ref{\detokenize{6_section:wh19-sali-1-4-png}}}
through \hyperref[\detokenize{6_section:wh19-sali-21-png}]{Fig.\@ \ref{\detokenize{6_section:wh19-sali-21-png}}}. The potential densities (\(\sigma\theta\)) are
plotted in \hyperref[\detokenize{6_section:wh19-sigma-1-4-png}]{Fig.\@ \ref{\detokenize{6_section:wh19-sigma-1-4-png}}} through \hyperref[\detokenize{6_section:wh19-sigma-21-png}]{Fig.\@ \ref{\detokenize{6_section:wh19-sigma-21-png}}}.

\sphinxAtStartPar
Contoured plots of temperature and salinity as a function of depth for the
deployments WHOTS\sphinxhyphen{}1 through \sphinxhyphen{}16 are presented in \hyperref[\detokenize{6_section:w1-19-contts-png}]{Fig.\@ \ref{\detokenize{6_section:w1-19-contts-png}}},
and contoured plots of potential density (\(\sigma\theta\)) as a function of depth are in
\hyperref[\detokenize{6_section:w1-19-contst-png}]{Fig.\@ \ref{\detokenize{6_section:w1-19-contst-png}}}, and of salinity as a function of \(\sigma\theta\)
are in \hyperref[\detokenize{6_section:w1-19-cont-s-png}]{Fig.\@ \ref{\detokenize{6_section:w1-19-cont-s-png}}}.

\sphinxAtStartPar
The potential temperature (\(\theta\)) and salinity measured by the deep MicroCATs
during the mooring deployment are shown in \hyperref[\detokenize{6_section:plt-w19-aco-all-png}]{Fig.\@ \ref{\detokenize{6_section:plt-w19-aco-all-png}}}. Also
shown in the plot are the \(\theta\) and salinity data obtained with a MicroCAT (SBE\sphinxhyphen{}37)
installed in the ALOHA Cabled Observatory, about six nautical miles north from
the WHOTS\sphinxhyphen{}19 anchor. The instrument is located 2 m above the bottom.

\begin{figure}[htbp]
\centering
\capstart

\noindent\sphinxincludegraphics[height=1000\sphinxpxdimen]{{wh19_Temp_1_4}.png}
\caption{Temperatures from MicroCATs during WHOTS\sphinxhyphen{}19 deployment at 1.5, 7, 15, and 25 m.}\label{\detokenize{6_section:wh19-temp-1-4-png}}\end{figure}

\begin{figure}[htbp]
\centering
\capstart

\noindent\sphinxincludegraphics[height=1000\sphinxpxdimen]{{wh19_Temp_5_8}.png}
\caption{Same as in \hyperref[\detokenize{6_section:wh19-temp-1-4-png}]{Fig.\@ \ref{\detokenize{6_section:wh19-temp-1-4-png}}}, but at 35, 40, 45, and 50 m.}\label{\detokenize{6_section:wh19-temp-5-8-png}}\end{figure}

\begin{figure}[htbp]
\centering
\capstart

\noindent\sphinxincludegraphics[height=1000\sphinxpxdimen]{{wh19_Temp_9_12}.png}
\caption{Same as in \hyperref[\detokenize{6_section:wh19-temp-1-4-png}]{Fig.\@ \ref{\detokenize{6_section:wh19-temp-1-4-png}}}, but at 55, 65, 75, and 85 m.}\label{\detokenize{6_section:wh19-temp-9-12-png}}\end{figure}

\begin{figure}[htbp]
\centering
\capstart

\noindent\sphinxincludegraphics[height=1000\sphinxpxdimen]{{figures/microcats/wh19_Temp_13_19}.png}
\caption{Same as in \hyperref[\detokenize{6_section:wh19-temp-1-4-png}]{Fig.\@ \ref{\detokenize{6_section:wh19-temp-1-4-png}}}, but at 95, 105, 120, and 135 m.}\label{\detokenize{6_section:wh19-temp-13-16-png}}\end{figure}

\begin{figure}[htbp]
\centering
\capstart

\noindent\sphinxincludegraphics[height=1000\sphinxpxdimen]{{figures/microcats/wh19_Temp_17_20}.png}
\caption{Same as in \hyperref[\detokenize{6_section:wh19-temp-1-4-png}]{Fig.\@ \ref{\detokenize{6_section:wh19-temp-1-4-png}}}, but at 155, 1875, and 4713 m.}\label{\detokenize{6_section:wh19-temp-17-20-png}}\end{figure}

\begin{figure}[htbp]
\centering
\capstart

\noindent\sphinxincludegraphics[height=1000\sphinxpxdimen]{{figures/microcats/wh19_Temp_21}.png}
\caption{Same as in \hyperref[\detokenize{6_section:wh19-temp-1-4-png}]{Fig.\@ \ref{\detokenize{6_section:wh19-temp-1-4-png}}}, but at 4713 m}\label{\detokenize{6_section:wh19-temp-21-png}}\end{figure}

\begin{figure}[htbp]
\centering
\capstart

\noindent\sphinxincludegraphics[height=1000\sphinxpxdimen]{{wh19_Sali_1_4}.png}
\caption{Salinities from MicroCATs during WHOTS\sphinxhyphen{}19 deployment at 1.5, 7, 15, and 25 m}\label{\detokenize{6_section:wh19-sali-1-4-png}}\end{figure}

\begin{figure}[htbp]
\centering
\capstart

\noindent\sphinxincludegraphics[height=1000\sphinxpxdimen]{{wh19_Sali_5_8}.png}
\caption{Same as in \hyperref[\detokenize{6_section:wh19-sali-1-4-png}]{Fig.\@ \ref{\detokenize{6_section:wh19-sali-1-4-png}}}, but at 35, 40, 45, and 50 m.}\label{\detokenize{6_section:wh19-sali-5-8-png}}\end{figure}

\begin{figure}[htbp]
\centering
\capstart

\noindent\sphinxincludegraphics[height=1000\sphinxpxdimen]{{wh19_Sali_9_12}.png}
\caption{Same as in \hyperref[\detokenize{6_section:wh19-sali-1-4-png}]{Fig.\@ \ref{\detokenize{6_section:wh19-sali-1-4-png}}}, but at 55, 65, 75, and 85 m}\label{\detokenize{6_section:wh19-sali-9-12-png}}\end{figure}

\begin{figure}[htbp]
\centering
\capstart

\noindent\sphinxincludegraphics[height=1000\sphinxpxdimen]{{wh19_Sali_13_16}.png}
\caption{Same as in \hyperref[\detokenize{6_section:wh19-sali-1-4-png}]{Fig.\@ \ref{\detokenize{6_section:wh19-sali-1-4-png}}}, but at 95, 105, 120, and 135 m.}\label{\detokenize{6_section:wh19-sali-13-16-png}}\end{figure}

\begin{figure}[htbp]
\centering
\capstart

\noindent\sphinxincludegraphics[height=1000\sphinxpxdimen]{{figures/microcats/wh19_Sali_17_20}.png}
\caption{Same as in \hyperref[\detokenize{6_section:wh19-sali-1-4-png}]{Fig.\@ \ref{\detokenize{6_section:wh19-sali-1-4-png}}}, but at 155, 1875, and 4713 m.}\label{\detokenize{6_section:wh19-sali-17-20-png}}\end{figure}

\begin{figure}[htbp]
\centering
\capstart

\noindent\sphinxincludegraphics[height=1000\sphinxpxdimen]{{figures/microcats/wh19_Sali_21}.png}
\caption{Same as in \hyperref[\detokenize{6_section:wh19-sali-1-4-png}]{Fig.\@ \ref{\detokenize{6_section:wh19-sali-1-4-png}}}, but at 4713 m.}\label{\detokenize{6_section:wh19-sali-21-png}}\end{figure}

\begin{figure}[htbp]
\centering
\capstart

\noindent\sphinxincludegraphics[height=1000\sphinxpxdimen]{{wh19_Sigma_1_4}.png}
\caption{Potential densities (\(\sigma\theta\)) from MicroCATs during WHOTS\sphinxhyphen{}19 deployment at 1.5, 7,
15, and 25 m.}\label{\detokenize{6_section:wh19-sigma-1-4-png}}\end{figure}

\begin{figure}[htbp]
\centering
\capstart

\noindent\sphinxincludegraphics[height=1000\sphinxpxdimen]{{wh19_Sigma_5_8}.png}
\caption{Same as in \hyperref[\detokenize{6_section:wh19-sigma-1-4-png}]{Fig.\@ \ref{\detokenize{6_section:wh19-sigma-1-4-png}}}, but at 35, 40, 45, and 50 m.}\label{\detokenize{6_section:wh19-sigma-5-8-png}}\end{figure}

\begin{figure}[htbp]
\centering
\capstart

\noindent\sphinxincludegraphics[height=1000\sphinxpxdimen]{{wh19_Sigma_9_12}.png}
\caption{Same as in \hyperref[\detokenize{6_section:wh19-sigma-1-4-png}]{Fig.\@ \ref{\detokenize{6_section:wh19-sigma-1-4-png}}}, but at 55, 65, 75, and 85 m.}\label{\detokenize{6_section:wh19-sigma-9-12-png}}\end{figure}

\begin{figure}[htbp]
\centering
\capstart

\noindent\sphinxincludegraphics[height=1000\sphinxpxdimen]{{wh19_Sigma_13_16}.png}
\caption{Same as in \hyperref[\detokenize{6_section:wh19-sigma-1-4-png}]{Fig.\@ \ref{\detokenize{6_section:wh19-sigma-1-4-png}}}, but at 95, 105, 120, and 135 m.}\label{\detokenize{6_section:wh19-sigma-13-16-png}}\end{figure}

\begin{figure}[htbp]
\centering
\capstart

\noindent\sphinxincludegraphics[height=1000\sphinxpxdimen]{{figures/microcats/wh19_Sigma_17_20}.png}
\caption{Same as in \hyperref[\detokenize{6_section:wh19-sigma-1-4-png}]{Fig.\@ \ref{\detokenize{6_section:wh19-sigma-1-4-png}}}, but at 155, 1875, and 4713 m.}\label{\detokenize{6_section:wh19-sigma-17-20-png}}\end{figure}

\begin{figure}[htbp]
\centering
\capstart

\noindent\sphinxincludegraphics[height=1000\sphinxpxdimen]{{figures/microcats/wh19_Sigma_21}.png}
\caption{Same as in \hyperref[\detokenize{6_section:wh19-sigma-1-4-png}]{Fig.\@ \ref{\detokenize{6_section:wh19-sigma-1-4-png}}}, but at 4713 m.}\label{\detokenize{6_section:wh19-sigma-21-png}}\end{figure}

\begin{figure}[htbp]
\centering
\capstart

\noindent\sphinxincludegraphics[height=1000\sphinxpxdimen]{{w1_19_contTS}.png}
\caption{Contour plots of temperature (upper panel) and salinity (lower panel) versus
depth from SeaCATs/MicroCATs during WHOTS\sphinxhyphen{}1 through WHOTS\sphinxhyphen{}19 deployments. The
shaded areas indicate missing data. The diamonds along the right axis indicate
the depths of the instrument.}\label{\detokenize{6_section:w1-19-contts-png}}\end{figure}

\begin{figure}[htbp]
\centering
\capstart

\noindent\sphinxincludegraphics[height=1000\sphinxpxdimen]{{w1_19_contSt}.png}
\caption{Contour plots of potential density (\(\sigma\theta\)), versus depth from SeaCATs/MicroCATs
during WHOTS\sphinxhyphen{}1 through WHOTS\sphinxhyphen{}19 deployments. The shaded areas indicate
missing data. The diamonds along the right axis in the upper figure
indicate the depths of the instrument.}\label{\detokenize{6_section:w1-19-contst-png}}\end{figure}

\begin{figure}[htbp]
\centering
\capstart

\noindent\sphinxincludegraphics[height=1000\sphinxpxdimen]{{w1_19_cont_S}.png}
\caption{Contour plots of salinity versus \(\sigma\theta\) from SeaCATs/MicroCATs during WHOTS\sphinxhyphen{}1
through WHOTS\sphinxhyphen{}19 deployments.}\label{\detokenize{6_section:w1-19-cont-s-png}}\end{figure}

\begin{figure}[htbp]
\centering
\capstart

\noindent\sphinxincludegraphics[height=1000\sphinxpxdimen]{{figures/microcats/plt_w19_aco_all}.png}
\caption{Potential temperature (upper panel) and salinity (lower panel) time\sphinxhyphen{}series from
the ALOHA Cabled Observatory (ACO) sensors and the WHOTS\sphinxhyphen{}19 MicroCATs 11391 and
12241.}\label{\detokenize{6_section:plt-w19-aco-all-png}}\end{figure}


\section{Moored ADCP Data}
\label{\detokenize{6_section:moored-adcp-data}}
\sphinxAtStartPar
Contoured plots of smoothed horizontal (east and north component) and vertical
velocity as a function of depth during the mooring deployments 1 through 19 are
presented in \hyperref[\detokenize{6_section:wh1-19-adcp-uv-cont-png}]{Fig.\@ \ref{\detokenize{6_section:wh1-19-adcp-uv-cont-png}}} and
\hyperref[\detokenize{6_section:wh1-19-adcp-w-cont-png}]{Fig.\@ \ref{\detokenize{6_section:wh1-19-adcp-w-cont-png}}}. A staggered time\sphinxhyphen{}series of smoothed
horizontal and vertical velocities are shown in
\hyperref[\detokenize{6_section:whots-19-u-subplot-png}]{Fig.\@ \ref{\detokenize{6_section:whots-19-u-subplot-png}}} through \hyperref[\detokenize{6_section:whots-19-w-subplot-png}]{Fig.\@ \ref{\detokenize{6_section:whots-19-w-subplot-png}}}.
Smoothing was performed by applying a daily running mean to the data and then
interpolating it on an hourly grid.

\sphinxAtStartPar
Contours of east and north velocity components from the Ship Oscar Sette Ocean
Surveyor broadband 75 kHz shipboard ADCP, and the moored 300 kHz ADCP from the
WHOTS\sphinxhyphen{}19 deployment as a function of time and depth, during the WHOTS\sphinxhyphen{}19
cruise, are shown in \hyperref[\detokenize{6_section:whots19deploy-adcp-contour1-png}]{Fig.\@ \ref{\detokenize{6_section:whots19deploy-adcp-contour1-png}}} and
\hyperref[\detokenize{6_section:whots19deploy-adcp-contour2-png}]{Fig.\@ \ref{\detokenize{6_section:whots19deploy-adcp-contour2-png}}}.

\begin{figure}[htbp]
\centering
\capstart

\noindent\sphinxincludegraphics[width=1000\sphinxpxdimen]{{figures/adcp_moored/wh1_19_adcp_uv_cont}.png}
\caption{Contour plot of east velocity component (\(m s^{-1}\)) versus depth and
time from the moored ADCPs from the WHOTS\sphinxhyphen{}1 through \sphinxhyphen{}19 deployments (upper panel).
Contour plot of north velocity component (\(m s^{-1}\)) (lower panel).}\label{\detokenize{6_section:wh1-19-adcp-uv-cont-png}}\end{figure}

\begin{figure}[htbp]
\centering
\capstart

\noindent\sphinxincludegraphics[width=1000\sphinxpxdimen]{{figures/adcp_moored/wh1_19_adcp_w_cont}.png}
\caption{Contour plot of vertical velocity component (\(m s^{-1}\)) versus depth
and time from the moored ADCPs from the WHOTS\sphinxhyphen{}1 through \sphinxhyphen{}19 deployments.}\label{\detokenize{6_section:wh1-19-adcp-w-cont-png}}\end{figure}

\begin{figure}[htbp]
\centering
\capstart

\noindent\sphinxincludegraphics[width=1000\sphinxpxdimen]{{WHOTS-19_u_subplot}.png}
\caption{Staggered time\sphinxhyphen{}series of east velocity component (\(m s^{-1}\))
for each bin of the 600 kHz (upper panel) and 300 kHz (lower panel) moored
ADCPs during WHOTS\sphinxhyphen{}19. The time\sphinxhyphen{}series are offset upwards by
0.5 \(m s^{-1}\); each bin’s depth is on the right.}\label{\detokenize{6_section:whots-19-u-subplot-png}}\end{figure}

\begin{figure}[htbp]
\centering
\capstart

\noindent\sphinxincludegraphics[width=1000\sphinxpxdimen]{{WHOTS-19_v_subplot}.png}
\caption{Same as \hyperref[\detokenize{6_section:whots-19-u-subplot-png}]{Fig.\@ \ref{\detokenize{6_section:whots-19-u-subplot-png}}} but for north velocity component}\label{\detokenize{6_section:whots-19-v-subplot-png}}\end{figure}

\begin{figure}[htbp]
\centering
\capstart

\noindent\sphinxincludegraphics[width=1000\sphinxpxdimen]{{WHOTS-19_w_subplot}.png}
\caption{Same as \hyperref[\detokenize{6_section:whots-19-u-subplot-png}]{Fig.\@ \ref{\detokenize{6_section:whots-19-u-subplot-png}}} but for north velocity component
but for vertical velocity component.}\label{\detokenize{6_section:whots-19-w-subplot-png}}\end{figure}


\section{Moored and Shipboard ADCP comparisons}
\label{\detokenize{6_section:moored-and-shipboard-adcp-comparisons}}
\sphinxAtStartPar
Contours of zonal and meridional current components from the Oscar Sette’s
Ocean Surveyor broadband 75 kHz shipboard ADCP, and the moored 300 kHz ADCP
from the WHOTS\sphinxhyphen{}19 deployment as a function of time and depth, during the
WHOTS\sphinxhyphen{}19 cruise, are shown in \hyperref[\detokenize{6_section:whots19deploy-adcp-contour1-png}]{Fig.\@ \ref{\detokenize{6_section:whots19deploy-adcp-contour1-png}}}. and
\hyperref[\detokenize{6_section:whots19deploy-adcp-contour2-png}]{Fig.\@ \ref{\detokenize{6_section:whots19deploy-adcp-contour2-png}}}. Similar comparisons during the
WHOTS\sphinxhyphen{}20 cruise are in \hyperref[\detokenize{6_section:whots19recover-adcp-contour1-png}]{Fig.\@ \ref{\detokenize{6_section:whots19recover-adcp-contour1-png}}}. and
\hyperref[\detokenize{6_section:whots19recover-adcp-contour2-png}]{Fig.\@ \ref{\detokenize{6_section:whots19recover-adcp-contour2-png}}}.

\begin{figure}[htbp]
\centering
\capstart

\noindent\sphinxincludegraphics[width=1000\sphinxpxdimen]{{figures/shipboard_adcp/whots19deploy_adcp_contour1}.png}
\caption{The contour of zonal currents (\(m s^{-1}\)) from the Ship Oscar Sette
Ocean Surveyor narrowband 75 kHz shipboard ADCP (upper panel), and the moored 300
kHz ADCP from the WHOTS\sphinxhyphen{}19 mooring (bottom panel) as a function of time and
depth, during the WHOTS\sphinxhyphen{}19 cruise. Times when the CTD rosette was in the
water are identified between solid and dashed black lines.}\label{\detokenize{6_section:whots19deploy-adcp-contour1-png}}\end{figure}

\begin{figure}[htbp]
\centering
\capstart

\noindent\sphinxincludegraphics[width=1000\sphinxpxdimen]{{figures/shipboard_adcp/whots19deploy_adcp_contour2}.png}
\caption{The contour of meridional currents (\(m s^{-1}\)) from the Ship Oscar Sette
Ocean Surveyor narrowband 75 kHz shipboard ADCP (upper panel), and the
moored 300 kHz ADCP from the WHOTS\sphinxhyphen{}19 mooring (bottom panel) as a function
of time and depth, during the WHOTS\sphinxhyphen{}19 cruise. Times when the CTD rosette
was in the water are identified between solid and dashed black lines.}\label{\detokenize{6_section:whots19deploy-adcp-contour2-png}}\end{figure}

\begin{figure}[htbp]
\centering
\capstart

\noindent\sphinxincludegraphics[width=1000\sphinxpxdimen]{{whots19recover_adcp_contour1}.png}
\caption{The contour of zonal currents (\(m s^{-1}\)) from the Ship Oscar Sette
Ocean Surveyor narrowband 75 kHz shipboard ADCP (upper panel), and the
moored 300 kHz ADCP from the WHOTS\sphinxhyphen{}19 mooring (bottom panel) as a function
of time and depth, during the WHOTS\sphinxhyphen{}20 cruise. Times when the CTD rosette
was in the water are identified between solid and dashed black lines.}\label{\detokenize{6_section:whots19recover-adcp-contour1-png}}\end{figure}

\begin{figure}[htbp]
\centering
\capstart

\noindent\sphinxincludegraphics[width=1000\sphinxpxdimen]{{whots19recover_adcp_contour2}.png}
\caption{Contours of meridional currents (\(m s^{-1}\)) from the Ship Oscar Sette Ocean
Surveyor narrowband 75 kHz shipboard ADCP (upper panel), and the moored 300 kHz
ADCP from the WHOTS\sphinxhyphen{}19 mooring (lower panel) as a function of time and depth,
during the WHOTS\sphinxhyphen{}20 cruise. Times when the CTD/rosette was in the water are
identified between the solid and dashed black lines.}\label{\detokenize{6_section:whots19recover-adcp-contour2-png}}\end{figure}

\sphinxAtStartPar
Comparisons between quality\sphinxhyphen{}controlled moored ADCPs during the WHOTS\sphinxhyphen{}19
deployment and available shipboard ADCP obtained during regular HOT cruises 316
to 332, and during the mooring deployment (WHOTS\sphinxhyphen{}19) and recovery (WHOTS\sphinxhyphen{}20)
cruises are shown in \hyperref[\detokenize{6_section:wh19-moor-ship-adcp-comp-1-png}]{Fig.\@ \ref{\detokenize{6_section:wh19-moor-ship-adcp-comp-1-png}}} and
\hyperref[\detokenize{6_section:wh19-moor-ship-adcp-comp-3-png}]{Fig.\@ \ref{\detokenize{6_section:wh19-moor-ship-adcp-comp-3-png}}} for the 300 kHz ADCP. Median and mean
velocity profiles were computed when HOT CTD casts were being conducted near
the WHOTS mooring specifically intended to calibrate moored instrumentation
(see {\hyperref[\detokenize{5_section:conductivity-calibration}]{\sphinxcrossref{\DUrole{std,std-ref}{Conductivity Calibration}}}}). The HOT shipboard profiles
were taken when the ship was stationary, within 1 km of the mooring, and within
4 hours before the start and 4 hours after the end of the CTD cast conducted
near the WHOTS mooring.

\sphinxAtStartPar
HOT\sphinxhyphen{}316 was conducted on the R/V Oceanus and used data from TRDI Workhorse 300
kHz ADCP (wh300) with 2 m bin size, and averaging ensembles every 2 minutes;
and from a TRDI Ocean Surveyor 75 kHz operating in broadband mode (os75nb) with
16 m bin size, with 5\sphinxhyphen{}minute ensemble interpolated to the profile resolution of
the shipboard ADCP, and ensemble mean, and median profiles were obtained for
each data set to compute differences and correlation coefficients between them.

\sphinxAtStartPar
HOT cruises conducted on the R/V Kilo Moana (HOT\sphinxhyphen{}317 to HOT\sphinxhyphen{}332) used data from
a TRDI Workhorse 300 kHz ADCP (wh300) with 2 m bin size and averaging
ensembles every 2 minutes; HOT\sphinxhyphen{}317 to HOT\sphinxhyphen{}319 and HOT\sphinxhyphen{}326 to HOT\sphinxhyphen{}332
used data from a TRDI Ocean Surveyor 38 kHz operating in broadband mode
(os38bb) with 12 m bin size, with 5\sphinxhyphen{}minute ensemble interpolated to the profile
resolution of the shipboard ADCP. HOT\sphinxhyphen{}317 to HOT\sphinxhyphen{}319 and HOT\sphinxhyphen{}326 to HOT\sphinxhyphen{}332
also used data from a TRDI Ocean Surveyor 38 kHz operating in narrowband mode
(os38nb) with 24 m bin size, with 5\sphinxhyphen{}minute ensemble. HOT\sphinxhyphen{}324 only used data
from the wh300 and os38nb (only three beams were working) instruments. HOT\sphinxhyphen{}326
also displayed issues with the os38 instrument.

\sphinxAtStartPar
Comparisons between the 300 kHz and the shipboard ADCP were available for
HOT\sphinxhyphen{}316, HOT\sphinxhyphen{}318 to \sphinxhyphen{}325, HOT\sphinxhyphen{}327 to \sphinxhyphen{}329
(\hyperref[\detokenize{6_section:wh19-moor-ship-adcp-comp-1-png}]{Fig.\@ \ref{\detokenize{6_section:wh19-moor-ship-adcp-comp-1-png}}}), and HOT\sphinxhyphen{}331 to \sphinxhyphen{}332
(\hyperref[\detokenize{6_section:wh19-moor-ship-adcp-comp-3-png}]{Fig.\@ \ref{\detokenize{6_section:wh19-moor-ship-adcp-comp-3-png}}}); data from all others HOT cruises
were excluded due to a lack of comparable data.

\sphinxAtStartPar
Comparisons between the moored 600 kHz and the shipboard ADCP were only
available for HOT\sphinxhyphen{}316 and HOT\sphinxhyphen{}318 due to a mechanical issue with 600 kHz
ADCP on January 21, 2020 (\hyperref[\detokenize{6_section:wh19-moor-ship-adcp-comp-2-png}]{Fig.\@ \ref{\detokenize{6_section:wh19-moor-ship-adcp-comp-2-png}}} and
\hyperref[\detokenize{6_section:wh19-moor-ship-adcp-comp-4-png}]{Fig.\@ \ref{\detokenize{6_section:wh19-moor-ship-adcp-comp-4-png}}}).

\begin{figure}[htbp]
\centering
\capstart

\noindent\sphinxincludegraphics[width=1000\sphinxpxdimen]{{figures/adcp_moored/wh19_moor_ship_ADCP_comp_1}.png}
\caption{Mean current profiles during shipboard ADCP (cyan: zonal, magenta: meridional)
versus moored 300 kHz ADCP (blue: zonal, red: meridional) intercomparisons from
HOT\sphinxhyphen{}316 through HOT\sphinxhyphen{}329. Moored minus shipboard ADCP differences shown in
dotted lines (blue: zonal, red: meridional)}\label{\detokenize{6_section:wh19-moor-ship-adcp-comp-1-png}}\end{figure}

\begin{figure}[htbp]
\centering
\capstart

\noindent\sphinxincludegraphics[width=1000\sphinxpxdimen]{{figures/adcp_moored/wh19_moor_ship_ADCP_comp_2}.png}
\caption{Mean current profiles during shipboard ADCP (cyan: zonal, magenta: meridional)
versus moored 600 kHz ADCP (blue: zonal, red: meridional) intercomparisons from
HOT\sphinxhyphen{}316 through HOT\sphinxhyphen{}329. Moored minus shipboard ADCP differences shown in
dotted lines (blue: zonal, red: meridional)}\label{\detokenize{6_section:wh19-moor-ship-adcp-comp-2-png}}\end{figure}

\begin{figure}[htbp]
\centering
\capstart

\noindent\sphinxincludegraphics[width=1000\sphinxpxdimen]{{figures/adcp_moored/wh19_moor_ship_ADCP_comp_3}.png}
\caption{Same as \hyperref[\detokenize{6_section:wh19-moor-ship-adcp-comp-1-png}]{Fig.\@ \ref{\detokenize{6_section:wh19-moor-ship-adcp-comp-1-png}}} but from HOT\sphinxhyphen{}331 through
HOT\sphinxhyphen{}332 and from WHOTS\sphinxhyphen{}19, and WHOTS\sphinxhyphen{}20 cruises.}\label{\detokenize{6_section:wh19-moor-ship-adcp-comp-3-png}}\end{figure}

\begin{figure}[htbp]
\centering
\capstart

\noindent\sphinxincludegraphics[width=1000\sphinxpxdimen]{{figures/adcp_moored/wh19_moor_ship_ADCP_comp_4}.png}
\caption{Same as \hyperref[\detokenize{6_section:wh19-moor-ship-adcp-comp-2-png}]{Fig.\@ \ref{\detokenize{6_section:wh19-moor-ship-adcp-comp-2-png}}} but from HOT\sphinxhyphen{}331 through
HOT\sphinxhyphen{}332 and from WHOTS\sphinxhyphen{}19, and WHOTS\sphinxhyphen{}20 cruises.}\label{\detokenize{6_section:wh19-moor-ship-adcp-comp-4-png}}\end{figure}


\section{Next Generation Vector Measuring Current Meter Data (VMCM)}
\label{\detokenize{6_section:next-generation-vector-measuring-current-meter-data-vmcm}}
\sphinxAtStartPar
Time\sphinxhyphen{}series of daily mean horizontal velocity components for the VMCM current
meters deployed during WHOTS\sphinxhyphen{}19 at 10 m and 30 m are presented in
\hyperref[\detokenize{6_section:whots19vmcm-plot-png}]{Fig.\@ \ref{\detokenize{6_section:whots19vmcm-plot-png}}}.

\begin{figure}[htbp]
\centering
\capstart

\noindent\sphinxincludegraphics[width=1000\sphinxpxdimen]{{whots19vmcm_plot}.png}
\caption{Horizontal velocity data (\(m s^{-1}\)) during WHOTS\sphinxhyphen{}19 from the VMCMs at
10 m depth (first and second panel) and at 30 m depth (third and fourth panel)}\label{\detokenize{6_section:whots19vmcm-plot-png}}\end{figure}


\section{GPS Data}
\label{\detokenize{6_section:gps-data}}
\sphinxAtStartPar
Time\sphinxhyphen{}series of latitude and longitude of the WHOTS\sphinxhyphen{}19 buoy from GPS data are
presented in \hyperref[\detokenize{6_section:wh19xeos-pos-png}]{Fig.\@ \ref{\detokenize{6_section:wh19xeos-pos-png}}}, and spectra of the time\sphinxhyphen{}series are
shown in \hyperref[\detokenize{6_section:wh19gps-spec-dpng-png}]{Fig.\@ \ref{\detokenize{6_section:wh19gps-spec-dpng-png}}}.

\begin{figure}[htbp]
\centering
\capstart

\noindent\sphinxincludegraphics[width=1000\sphinxpxdimen]{{wh19xeos_pos}.png}
\caption{GPS Latitude (upper panel) and longitude (lower panel) time series from the
WHOTS\sphinxhyphen{}19 deployment.}\label{\detokenize{6_section:wh19xeos-pos-png}}\end{figure}

\begin{figure}[htbp]
\centering
\capstart

\noindent\sphinxincludegraphics[width=1000\sphinxpxdimen]{{wh19gps_spec_dpng}.png}
\caption{The power spectrum of latitude (upper panel) and longitude (lower panel) for
the WHOTS\sphinxhyphen{}19.}\label{\detokenize{6_section:wh19gps-spec-dpng-png}}\end{figure}


\section{Mooring Motion}
\label{\detokenize{6_section:mooring-motion}}
\sphinxAtStartPar
The position of the mooring with respect to its anchor was determined from the
GPS positions. Additional information on the mooring motion was provided by the
ADCP data of pitch, roll, and heading, shown in this section.

\sphinxAtStartPar
\hyperref[\detokenize{6_section:wh19-adcp-tilt-png}]{Fig.\@ \ref{\detokenize{6_section:wh19-adcp-tilt-png}}} shows the ADCP data of the instrument’s tilt (a
combination of the pitch and roll), plotted against the buoy’s distance from
its anchor (derived from GPS positions), for both WHOTS\sphinxhyphen{}19 ADCP’s. The plot’s
red line is a quadratic fit to the median tilt calculated every 0.2 km distance
bins. The figure shows that during both deployments, the ADCP tilt increased as
the anchor’s distance increased. This tilting was caused by the mooring line’s
deviation from its vertical position as the anchor pulled it. The tilting of
the line also caused the rising of the instruments attached to the line. The
600 kHz ADCP failed in January 2021.

\begin{figure}[htbp]
\centering
\capstart

\noindent\sphinxincludegraphics[width=1000\sphinxpxdimen]{{wh19_adcp_tilt}.png}
\caption{Scatter plots of ADCP tilt and distance of the buoy to its anchor for the 300
kHz (left panel) and the 600 kHz ADCP deployments (right panel, blue circles).
The red line is a quadratic fit to the median tilt calculated every 0.2 km
distance bins.}\label{\detokenize{6_section:wh19-adcp-tilt-png}}\end{figure}


\chapter{Appendix}
\label{\detokenize{appendices:appendix}}\label{\detokenize{appendices::doc}}

\section{WHOTS\sphinxhyphen{}19 300 kHz \sphinxhyphen{} Serial 4891}
\label{\detokenize{appendices:whots-19-300-khz-serial-4891}}

\section{WHOTS\sphinxhyphen{}19 600 kHz \sphinxhyphen{} Serial 1825}
\label{\detokenize{appendices:whots-19-600-khz-serial-1825}}
\begingroup
\renewcommand\chapter[1]{\endgroup}
\phantomsection


\chapter{References}
\label{\detokenize{references:references}}\label{\detokenize{references:refs}}\label{\detokenize{references::doc}}
\begin{sphinxthebibliography}{10}
\bibitem[1]{references:id14}
\sphinxAtStartPar
Albert J. Plueddemann, Robert A. Weller, Roger Lukas, Jeffrey Lord, Paul R. Bouchard, and M. Alexander Walsh. Whoi hawaii ocean timeseries station (whots) : whots\sphinxhyphen{}2 mooring turnaround cruise report. Technical Report, Woods Hole Oceanographic Institution, 2006. URL: \sphinxurl{https://hdl.handle.net/1912/1074}, \sphinxhref{https://doi.org/10.1575/1912/1074}{doi:10.1575/1912/1074}.
\bibitem[2]{references:id22}
\sphinxAtStartPar
Sean P. Whelan, Robert A. Weller, Roger Lukas, Frank Bradley, Jeffrey Lord, Jason C. Smith, Frank B. Bahr, Paul Lethaby, and Jeffrey Snyder. Whoi hawaii ocean timeseries station (whots) : whots\sphinxhyphen{}3 mooring turnaround cruise report. Technical Report, Woods Hole Oceanographic Institution, 2007. URL: \sphinxurl{https://hdl.handle.net/1912/1825}, \sphinxhref{https://doi.org/10.1575/1912/1825}{doi:10.1575/1912/1825}.
\bibitem[3]{references:id23}
\sphinxAtStartPar
Sean P. Whelan, Albert J. Plueddemann, Roger Lukas, Jeffrey Lord, Paul Lethaby, Jason C. Smith, Frank B. Bahr, Nancy R. Galbraith, and Christopher L. Sabine. Whoi hawaii ocean timeseries station (whots): whots\sphinxhyphen{}4 2007 mooring turnaround cruise report. Technical Report, Woods Hole Oceanographic Institution, 1 2008. \sphinxhref{https://doi.org/https://doi.org/10.1575/1912/2504}{doi:https://doi.org/10.1575/1912/2504}.
\bibitem[4]{references:id25}
\sphinxAtStartPar
Sean P. Whelan, Fernando Santiago\sphinxhyphen{}Mandujano, Frank Bradley, Albert J. Plueddemann, Ludovic Barista, James R. Ryder, Roger Lukas, Paul Lethaby, Jefrey Snyder, Christopher L. Sabine, Diane Stanitski, Anita D. Rapp, Christopher W. Fairall, Sergio Pezoa, Nancy R. Galbraith, Jeffrey Lord, and Frank B. Bahr. Whoi hawaii ocean timeseries station (whots): whots\sphinxhyphen{}6 2009 mooring turnaround cruise report. Technical Report, Woods Hole Oceanographic Institution, 2 2010. \sphinxhref{https://doi.org/https://doi.org/10.1575/1912/3458}{doi:https://doi.org/10.1575/1912/3458}.
\bibitem[5]{references:id17}
\sphinxAtStartPar
Fernando Santiago\sphinxhyphen{}Mandujano, Fernando Carvalho Pacheco, Dan Fitzgerald, Kelsey Maloney, Tully Rohrer, Camile Adksion, and James Potemra. Uh contributions to whots\sphinxhyphen{}19 cruise report. Technical Report, School of Ocean and Earth Science and Technology, University of Hawaii, April 2024. URL: \sphinxurl{https://www.soest.hawaii.edu/whots/proc\_reports/WHOTS19\_Cruise\_Report.pdf}.
\bibitem[6]{references:id18}
\sphinxAtStartPar
Fernando Santiago\sphinxhyphen{}Mandujano, Dan Fitzgerald, Kelsey Maloney, Tully Rohrer, Merriitt Shepherd, Prajna Jandial, and James Potemra. Uh contributions to whots\sphinxhyphen{}20 cruise report. Technical Report, School of Ocean and Earth Science and Technology, University of Hawaii, July 2024. URL: \sphinxurl{https://www.soest.hawaii.edu/whots/proc\_reports/WHOTS20\_Cruise\_Report.pdf}.
\bibitem[7]{references:id9}
\sphinxAtStartPar
David S. Hosom, Robert A. Weller, Richard E. Payne, and Kenneth E. Prada. The imet (improved meteorology) ship and buoy systems. \sphinxstyleemphasis{Journal of Atmospheric and Oceanic Technology}, 12:527\textendash{}540, 6 1995. \sphinxhref{https://doi.org/10.1175/1520-0426(1995)012\textless{}0527:timsab\textgreater{}2.0.co;2}{doi:10.1175/1520\sphinxhyphen{}0426(1995)012\textless{}0527:timsab\textgreater{}2.0.co;2}.
\bibitem[8]{references:id2}
\sphinxAtStartPar
Keir Colbo and Robert A. Weller. Accuracy of the imet sensor package in the subtropics. \sphinxstyleemphasis{Journal of Atmospheric and Oceanic Technology}, 26:1867\textendash{}1890, 9 2009. \sphinxhref{https://doi.org/10.1175/2009JTECHO667.1}{doi:10.1175/2009JTECHO667.1}.
\bibitem[9]{references:id11}
\sphinxAtStartPar
A. Birol Kara, Peter A. Rochford, and Harley E. Hurlburt. Mixed layer depth variability and barrier layer formation over the north pacific ocean. \sphinxstyleemphasis{Journal of Geophysical Research: Oceans}, 105:16783\textendash{}16801, 7 2000. \sphinxhref{https://doi.org/10.1029/2000jc900071}{doi:10.1029/2000jc900071}.
\bibitem[10]{references:id15}
\sphinxAtStartPar
Fernando Santiago\sphinxhyphen{}Mandujano, Paul Lethaby, Roger Lukas, Jefrey Snyder, Robert A. Weller, Albert J. Plueddemann, Jeffrey Lord, Sean Whelan, Paul Bouchard, and Nan Galbraith. Hydrographic observations at the woods hole oceanographic institution (whoi) hawaii ocean timeseries (hot) site (whots): 2004\sphinxhyphen{}2006. Technical Report, School of Ocean and Earth Science and Technology, University of Hawaii, 2007. URL: \sphinxurl{http://uop.whoi.edu/currentprojects/WHOTS/docs/UH\_Whots\_data\_report\_1.pdf}.
\bibitem[11]{references:id20}
\sphinxAtStartPar
Luis Tupas, Fernando Santiago\sphinxhyphen{}Mandujano, Dale Hebel, Roger Lukas, David Karl, and Eric Firing. Hawaii ocean time\sphinxhyphen{}series data report 4, 1992. Technical Report, School of Ocean and Earth Science and Technology, University of Hawaii, 1993. URL: \sphinxurl{https://hahana.soest.hawaii.edu/hot/reports/rep\_y4.pdf}.
\bibitem[12]{references:id13}
\sphinxAtStartPar
W. Brechner Owens and Robert C. Millard. A new algorithm for ctd oxygen calibration. \sphinxstyleemphasis{Journal of Physical Oceanography}, 15(5):621 \textendash{} 631, 1985. URL: \sphinxurl{https://journals.ametsoc.org/view/journals/phoc/15/5/1520-0485\_1985\_015\_0621\_anafco\_2\_0\_co\_2.xml}, \sphinxhref{https://doi.org/10.1175/1520-0485(1985)015\textless{}0621:ANAFCO\textgreater{}2.0.CO;2}{doi:10.1175/1520\sphinxhyphen{}0485(1985)015\textless{}0621:ANAFCO\textgreater{}2.0.CO;2}.
\bibitem[13]{references:id19}
\sphinxAtStartPar
Luis Tupas, Fernando Santiago\sphinxhyphen{}Mandujando, Dale Hebel, Craig Nosse, Lance Fujieki, Eric Firing, Roger Lukas, David Karl, Christopher Winn, Robert Bidigare, Michael Landry, and Mai Lopez. Hawaii ocean time\sphinxhyphen{}series data report 8, 1996. Technical Report, School of Ocean and Earth Science and Technology, University of Hawaii, 1996. URL: \sphinxurl{https://hahana.soest.hawaii.edu/hot/reports/rep\_y8.pdf}.
\bibitem[14]{references:id16}
\sphinxAtStartPar
Fernando Santiago\sphinxhyphen{}Mandujano, Fernando Carvalho Pacheco, Dan Fitzgerald, Kelsey Maloney, Tully Rohrer, James Harris III, Noah Howins, and James Potemra. Uh contributions to whots\sphinxhyphen{}18 cruise report. Technical Report, School of Ocean and Earth Science and Technology, University of Hawaii, 2022. URL: \sphinxurl{https://www.soest.hawaii.edu/whots/proc\_reports/WHOTS18\_Cruise\_Report.pdf}.
\bibitem[15]{references:id4}
\sphinxAtStartPar
Howard Paul Freitag, M E McCarty, Craig Nosse, Roger Lukas, Michael J. McPhaden, and Meghan F. Cronin. Coare seacat data: calibrations and quality control procedures. \sphinxstyleemphasis{NOAA Technical Memorandum ERL PMEL\sphinxhyphen{}115}, pages 1\textendash{}89, 1999. URL: \sphinxurl{https://repository.library.noaa.gov/view/noaa/10999}.
\bibitem[16]{references:id3}
\sphinxAtStartPar
Eric Firing. Acoustic doppler current profiling measurements and navigation. \sphinxstyleemphasis{WOCE Hydrographic Operations and Methods. WOCE Operations Manual, WHP Office Report WHPO 91\sphinxhyphen{}1, WOCE Report No. 68/91}, pages 1\textendash{}24, 1991. URL: \sphinxurl{https://www.nodc.noaa.gov/archive/arc0013/0001873/1.1/data/1-data/publications/WOCE/ADCP.pdf}.
\bibitem[17]{references:id27}
\sphinxAtStartPar
Teledyne RD Instruments. \sphinxstyleemphasis{Acoustic Doppler Current Profiler Principles of Operation A Practical Primer}. 1 2011. URL: \sphinxurl{http://www.teledynemarine.com/Documents/Brand\%20Support/RD\%20INSTRUMENTS/Technical\%20Resources/Manuals\%20and\%20Guides/General\%20Interest/BBPRIME.pdf}.
\bibitem[18]{references:id12}
\sphinxAtStartPar
Roger Lukas, Fernando Santiago\sphinxhyphen{}Mandujano, Frederick Bingham, and Arnold Mantyla. Cold bottom water events observed in the hawaii ocean time\sphinxhyphen{}series: implications for vertical mixing. \sphinxstyleemphasis{Deep Sea Research Part I: Oceanographic Research Papers}, 48:995\textendash{}1021, 2001. URL: \sphinxurl{https://www.sciencedirect.com/science/article/pii/S0967063700000789}, \sphinxhref{https://doi.org/https://doi.org/10.1016/S0967-0637(00)00078-9}{doi:https://doi.org/10.1016/S0967\sphinxhyphen{}0637(00)00078\sphinxhyphen{}9}.
\bibitem[19]{references:id10}
\sphinxAtStartPar
Bruce M Howe, Roger Lukas, Fred Duennebier, and David Karl. Aloha cabled observatory installation. In \sphinxstyleemphasis{OCEANS\textquotesingle{}11 MTS/IEEE KONA}, 1\textendash{}11. Waikoloa, HI, USA, 2011. IEEE. \sphinxhref{https://doi.org/10.23919/OCEANS.2011.6107301}{doi:10.23919/OCEANS.2011.6107301}.
\end{sphinxthebibliography}



\renewcommand{\indexname}{Index}
\printindex
\end{document}